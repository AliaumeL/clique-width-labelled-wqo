
\section{Bounded Clique Width}
\label{sec:hereditary-classes}


In this section, we will aim at leveraging the previous results 
to tackle classes of \kl{bounded clique-width}. Our first goal is to 
prove our \cref{main:theorem} regarding the 
images of finite trees under \kl{MSO interpretations}. There are two difficulties here:
first, we only dealt with \kl{simple MSO interpretations} so far, and second,
we have not yet given any decision procedure to check whether the image of
a given \kl{simple MSO interpretation} is \kl{2-well-quasi-ordered}. Let us first
tackle the second difficulty.

\begin{definition}[Perfect Bough]
  \label{def:perfect-bough}
  Let $\aTree = C[B]$ be a tree with a \kl{bough} $B$ of level $k$.
  We say that $B$ is a \intro{perfect bough} there exists a compatible
  \kl{bough} $H$ of level $k$ and 
  a map $h \colon \someInterp(C[B]) \to \someInterp(C[H])$
  \begin{itemize}
      \item $h$ is an embedding of graphs,
      \item $h$ is the identity map on leaves belonging $C[\square]$
      \item the \kl{bough type} of every leaf $x$ in $H$ is the same as 
        the \kl{bough type} of $h(x)$ in $B$.
      \item there exists a \kl{block} in $H$ that is left untouched 
          by $h$.
  \end{itemize}
  An \intro{imperfect bough} is a \kl{bough} that is not a \kl{perfect bough}.
\end{definition}

\begin{lemma}
  \label{lem:perfect-boughs-wqo}
  The image of $\someInterp$ is \kl{$2$-well-quasi-ordered} 
  if and only if
  the image of $\someInterp$ is \kl{$\forall$-well-quasi-ordered} 
  if and only if 
  there exists a bound on the \kl{dimension} of \kl{imperfect boughs}.
\end{lemma}
\begin{proof}
  First, if we have a bound on the \kl{dimension} of \kl{imperfect boughs},
  then we also have a bound on the \kl{dimension} of \kl{bad boughs},
  because every \kl{bad bough} is an \kl{imperfect bough}.
  Hence, by \cref{lem:good-boughs-wqo}, the image of $\someInterp$ is
  \kl{$\forall$-well-quasi-ordered}.

  Conversely, assume that there exist \kl{imperfect boughs} of arbitrarily
  large \kl{dimension}. Then, by an argument similar to the one of 
  \cref{lem:bad-bough-antichain}, we can construct an infinite
  labelled \kl{antichain} in the image of $\someInterp$. The only change is that 
  one also adds the \kl{bough type} of the leaves as additional labels.
\end{proof}

The important advantage of \cref{lem:perfect-boughs-wqo} over
\cref{lem:good-boughs-wqo} is that \kl{imperfect boughs} are easily
recognizable.

\begin{lemma}
  \label{lem:recognizing-perfect-boughs}
  There exists an $\MSO$ formula that recognizes \kl{imperfect boughs}.
\end{lemma}
\begin{proof}[Proof Sketch]
  The proof follows the same pattern as 
  \cite[Lemma 19]{LOPEZ24}. The main idea is that if a \kl{bough} $B$ is
  \kl{perfect}, then the only thing that matters for a given leaf $x$ in $B$
  is its \kl{bough type} and whether it is mapped on the left or right of the
  ``untouched block'' in the \kl{bough} $H$. Hence, one can guess for every leaf 
  in $B$ where it is mapped (left or right of the untouched block),
  and verify that the mapping preserves edges using only \kl{MSO} formulas.
  If such a mapping exists, then one can reconstruct a bough $H$
  by triplicating $B$, that is, $H = B B B$, and mapping leaves 
  to their corresponding copy in $H$. The middle copy will be untouched, 
  witnessing that $B$ is a \kl{perfect bough}.
  This reasoning was made formal for \kl{linear clique-width} in
  \cite[Lemma 16]{LOPEZ24}, and the exact same proof works here.
\end{proof}

\begin{corollary}
  One can decide whether there are \kl{imperfect boughs} of arbitrarily large
  \kl{dimension}.
\end{corollary}
\begin{proof}
  One can build a cost-mso formula on trees that outputs $0$ if the tree 
  is not an \kl{imperfect bough}, and outputs its \kl{dimension} otherwise.
  By \cite{COLOD10}, one can decide whether this cost-mso formula is
  bounded on the class of finite trees, hence the result.
\end{proof}


We are now ready to prove our \cref{main:theorem}.
\begin{proof}
  Let $\someInterp$ be an \kl{MSO interpretation}.
  We can get rid of the domain formula and selection formula 
  by standard arguments without changing the \kl{2-well-quasi-ordering}
  status of the image of $\someInterp$ (see \cite{LOPEZ24}).
  Hence, we can assume that $\someInterp$ is a \kl{simple MSO interpretation}.
  Furthermore, the transformation from an \kl{simple MSO interpretation}
  to a \kl{monoid interpretation} from trees
  described in \cref{sec:ramseyan} is effective.
  Finally, by \cref{lem:perfect-boughs-wqo}
  the image of $\someInterp$ is \kl{$2$-well-quasi-ordered}
  if and only if it is \kl{$\forall$-well-quasi-ordered}, and 
  by \cref{lem:recognizing-perfect-boughs},
  one can decide whether the image of $\someInterp$ is \kl{$2$-well-quasi-ordered}. 

  It remains to discuss the fact that one can add a total ordering on the graphs 
  without changing the \kl{2-well-quasi-ordering} status of the image of $\someInterp$.
  This is because we prove that the image of $\someInterp$ is \kl{$2$-well-quasi-ordered}
  using the \kl{marked gap-embedding} ordering on \kl{marked nested trees}
  representing the graphs, which already includes a total ordering on the leaves
  (the topological ordering).
\end{proof}

Let now turn our attention to hereditary classes to obtain our 
\cref{main:corollary}.
\begin{proof}
  By a standard argument, if the class $\someInterp$ is hereditary
  and \kl{$2$-well-quasi-ordered}, then it is defined by finitely many
  forbidden induced subgraphs (see \cite{LOPEZ24} for details).
  Hence, if $\Cls$ has \kl{bounded clique-width} and is hereditary,
  one can assume that $\Cls$ is the image of an \kl{MSO interpretation}
  $\someInterp$ from finite trees, and the result follows from 
  \cref{main:theorem}.
\end{proof}


Let us now turn our attetion to \cref{thm:characterisations},
and in particular the link between \kl{2-well-quasi-ordering} and
\kl{bounded linear clique-width}. The following lemma shows that
the existence of \kl{imperfect boughs} of arbitrarily large \kl{dimension}
can be witnessed using only \kl{blocks} from a finite set.

\begin{lemma}
  \label{lem:hereditary-perfect-boughs}
  Assume that there are \kl{imperfect boughs} of arbitrarily large \kl{dimension}.
  Then, there exists a finite set of blocks $\mathbb{F}$ such that 
  one can build an \kl{imperfect bough} of arbitrarily large \kl{dimension}
  using only \kl{blocks} from $\mathbb{F}$.
\end{lemma}

As an immediate consequence of \cref{lem:perfect-boughs-wqo} and
\cref{lem:hereditary-perfect-boughs}, we see that 

\begin{corollary}
  \label{cor:hereditary-perfect-boughs-wqo}
  Assume that the class $\someInterp$ is hereditary.
  Then, the image of $\someInterp$ is \kl{$2$-well-quasi-ordered}
  if and only if 
  every class of \kl{bounded linear clique-width} in $\someInterp$
  is \kl{$2$-well-quasi-ordered}.
\end{corollary}

