\section{From Interpretations to Nested Trees}
\label{sec:ramseyan}

As a first step towards proving \cref{main-theorem:thm}, we are going to
transform a \kl{simple $\MSO$-interpretation} from trees to graphs into a more
combinatorial object that will make the interpretation \emph{quantifier-free}.
To that end, we will chain two standard transformations from automata theory:
the first one is to transform an \kl{$\MSO$-interpretation} into a so-called
\kl{monoid interpretation}, and the second one is to factorise the trees using
\kl{forward Ramseyan splits} (nested trees) \cite{COLC07}. Finally, we will
show that the natural ordering on \kl{nested trees} (the \kl{gap embedding}) is
\emph{almost} such that the resulting interpretation is order preserving from
\kl{nested trees} to graphs.

\AP  Let us fix for the rest of the section a finite alphabet $\Sigma$ and a
\kl{simple $\MSO$-interpretation} $\someInterp$ from $\Trees{\Sigma}{}$ to
finite undirected graphs defined by a formula $\varphi(x,y)$ that defines the
edges of the graphs.



\subsection{From Interpretation to Monoids}

\AP It follows from standard arguments that to a formula $\varphi(x,y)$ over
$\Trees{\Sigma}{}$ one can associate a finite monoid $M$ and a morphism $\mu
\colon \Sigma^* \to M$ such that for any tree $T$ and any pair of leaves $x
\treesibleq y$ in $T$, whether $T \models \varphi(x,y)$ is entirely determined
by the values of $\mu(T[z \colon x])$, $\mu(T[z \colon y])$ and
$\mu(T[\treeRoot \colon z])$ where $z = \lca(x,y)$ is the \kl{least common
ancestor} of $x$ and $y$. We refer to the book on Tree Automata Techniques and
Applications for a comprehensive overview of the connection between MSO on
trees, and tree automata \cite{TATA08}.


\AP As a consequence, one can collect in a subset $P \subseteq M^3$ all the
triples $(a,b,c)$ such that this triple corresponds to a satisfying assignment
of the formula $\varphi(x,y)$. This is illustrated in
\cref{interpretation-to-monoid:fig}. Given two nodes $x \treelt[T] y$ in a tree
$T$, we will write $\tlbl{T}{x}{y}$ for the product of the edge labels on the
path from $x$ to $y$ in $T$.
Adding the function $\tlbl{T}{x}{y}$ to the signature of trees, and assuming 
that $\lca$ is also part of the signature, we can rewrite the formula
$\varphi(x,y)$ as a quantifier-free formula as follows:
\begin{equation}
    \label{monoid-interpretation:eq}
    (\tlbl{T}{\treeRoot}{\lca(x,y)},
     \tlbl{T}{\lca(x,y)}{x}, \tlbl{T}{\lca(x,y)}{y}) \in P
\end{equation}

\begin{figure}
    \centering
    \begin{tikzpicture}[
        leaf/.style={
            color=Prune
        },
        lca/.style={
            color=A1
        },
        edge/.style={
            color=A2
        },
        root/.style={
            color=Prune
        },
        gedge/.style={
            color=A4
        },
        ]
        \node[root] (root) at (0,1) {$\treeRoot$};
        \node[leaf] (x) at (-1,-1) {$x$};
        \node[leaf] (y) at (1,-1) {$y$};
        \node[lca] (lca) at (0,0) {$\lca(x,y)$};
        \coordinate (tl) at ($(x.south west)+(-0.2,0)$);
        \coordinate (tr) at ($(y.south east)+(0.2,0) $);
        \coordinate (t)  at ($(root.north)+(0, 1)$);
        \draw[edge,->] (root) to node[midway, left]        {$a$}  (lca);
        \draw[edge,->] (lca)  to node[midway, above  left] {$b$} (x);
        \draw[edge,->] (lca)  to node[midway, above right] {$c$} (y);
        \draw (tl) -- (tr) -- (t) -- cycle;

        \node (maps-to) at (2,0) {$\mapsto$};

        \begin{scope}[xshift=2cm]
            \node[leaf] (nx) at (2,-1) {$x$};
            \node[leaf] (ny) at (2,2) {$y$};
            \draw[gedge] (nx) to node[midway, above, rotate=90] {$(a,b,c) \in P$?} (ny);
        \end{scope}
    \end{tikzpicture}
    \caption{The interpretation of a tree using a monoid and an accepting part $P \subseteq M^3$.}
    \label{interpretation-to-monoid:fig}
\end{figure}

The main idea driving this paper is that understanding when a class of graphs
is \kl{$\forall$-well-quasi-ordered} can be reduced to finding a suitable
\kl{well-quasi-order} on trees such that the interpretation $\someInterp$ is
\emph{order-preserving} from trees to graphs, leveraging the following standard
\cref{fact:surjective-wqo}.

\begin{fact}
  \label{fact:surjective-wqo}
  Let $(X, \leq_X)$ and $(Y, \leq_Y)$ be two quasi-orders,
  and $f \colon X \to Y$ be a surjective order-preserving map.
  If $(X, \leq_X)$ is \kl{well-quasi-ordered},
  then $(Y, \leq_Y)$ is \kl{well-quasi-ordered}.
\end{fact}

\AP One candidate ordering on trees is the \intro{composition ordering} defined
as follows. Given two trees $\aTree_1$ and $\aTree_2$ labelled over a monoid
$M$, we say that $\aTree_1$ is less than $\aTree_2$ in the \kl{composition
ordering}, written $\aTree_1 \cmpleq \aTree_2$, if there exists a map $h \colon
\aTree_1 \to \aTree_2$ that maps leaves to leaves, respects the relation
$\lca$, $\treesibleq$, and the function $\tlbl{\cdot}{\cdot}{\cdot}$. This
ordering is naturally extended to node-labelled trees by asking that for every
node $x$ in $\aTree_1$, the label of $x$ is less or equal than the label of
$h(x)$ in $\aTree_2$. It is straightforward to check that the interpretation
$\someInterp$ is order-preserving from trees equipped with the \kl{composition
ordering} to graphs equipped with the \kl{induced subgraph} ordering. Hence, to
prove that the image of $\someInterp$ is \kl{$\forall$-well-quasi-ordered}, it
suffices to prove that the class of trees is \kl{well-quasi-ordered} under the
\kl{composition ordering} when labelled using a \kl{well-quasi-ordered} set
$(X, \leq_X)$.


\AP  Historically, this  was (although not explicitly) the approach taken by
\cite{DRT10}. Unfortunately, the composition ordering on trees is more often
than not way more strict than the \kl{induced subgraph} ordering. It was
already observed that the composition ordering on trees is
\kl{well-quasi-ordered} if and only if \emph{for every} $P \subseteq M^3$, the
class of graphs obtained from trees by considering only the triples in $P$ is
\kl{labelled-well-quasi-ordered} \cite[Theorem 24]{LOPEZ24}. To understand what happens for
a precise choice of $P$, we need to have a finer understanding of the way trees
can be decomposed.


\subsection{Forward Ramseyan Splits}

\def\t{\aTree} \AP The combinatorial ingredient allowing us to  further
decompose the trees will come from an adaptation of the classical Simon
Factorisation Theorem for semigroups \cite{SIMO90}, adapted to trees by
Colcombet \cite{COLC07}. The rest of this section is devoted to explaining how
this factorisation works, and how it can be used to define a more suitable
ordering on trees.

\AP A \intro{split of height $N$} of a tree $\t$ is a mapping $\spt$ from the
nodes of $T$ to $\set{1, \dots, N}$. Given a split $\spt$ and two nodes $x
\treelt[\t] y$, we define $\spt(x \colon y)$ to be the minimal value of
$\spt(z)$ for $x \treelt[\t] z \treelt[\t] y$, and $N+1$ otherwise. Two
elements $x \treelt[\t] y$ such that $\spt(x) = \spt(y) = k$ are
\intro{$k$-neighbours} if $\spt(x \colon y) \geq k$. Note that a \kl{split}
induces some kind of hierarchical structure on the tree based on the
\kl{$k$-neighbourhoods}. We illustrate the situation on a branch of a tree in
\cref{split-on-branch:fig}.

\begin{figure}
  \centering 
  \begin{tikzpicture}
  %
  % s : 3 2 1 2 2 1 1 3 2 1 3
  %     ---------------------
  %       -----------  ----
  %         -     --      - 
    \node (s) at (0,0) {$\spt \colon$};
    \foreach \i/\s in {1/1, 2/2, 3/3, 4/2, 5/2, 6/3, 7/3, 8/1, 9/2, 10/3, 11/1} {
      % x : \i * 0.5 
      % y : (3 -\s) * 0.5
      \pgfmathsetmacro{\x}{\i * 0.7}
      \pgfmathsetmacro{\y}{0}
      \node (x\i) at (\x,\y) {$\s$};
    };

    % draw edges of the branch
    \foreach \i in {1,...,10} {
      \pgfmathsetmacro{\xone}{\i}
      \pgfmathsetmacro{\xtwo}{int(\i + 1)}
      \draw[->] (x\xone) -- (x\xtwo);
    }

    % Draw 3-neighbourhoods
    % 1 / 8 / 11
    \draw[A4] (x1) to[bend left=70] (x8);
    \draw[A4] (x8) to[bend left=70] (x11);
    % Draw 2-neighbourhoods
    \draw[A5] (x2) to[bend left=50] (x4);
    \draw[A5] (x4) to[bend left=50] (x5);
    % Draw 1-neighbourhoods
    \draw[A3] (x6) to[bend left=30] (x7);
  \end{tikzpicture}

  \caption{A split $\spt$ on a branch of a tree. The arcs 
  in \textcolor{A4}{dark blue} connect $1$-neighbours, 
  the arcs in \textcolor{A5}{light blue} connect $2$-neighbours,
  and the arcs in \textcolor{A3}{light purple} connect $3$-neighbours. Full neighbourhoods
  are obtained by considering the connected components induced by the (reflxive closure) of
  arcs of each color.}
  \label{split-on-branch:fig}
\end{figure}


\AP A split $\spt$ is \intro{forward Ramseyan} if for every $k \in \set{1,
\dots, N}$ and every $x, y, x', y'$ in the same class of \kl{$k$-neighbourhood}
with $x \treelt[\t] y$ and $x' \treelt[\t] y'$, we have: 
\begin{equation}
  \label{fake-idempotent:eq} 
  \tlbl{\t}{x}{y} = \tlbl{\t}{x}{y} \cdot \tlbl{\t}{x'}{y'} \quad . 
\end{equation} 

\AP In particular, $\tlbl{\t}{x}{y}$ is always an \intro{idempotent}:
$\tlbl{\t}{x}{y} \cdot \tlbl{\t}{x}{y} = \tlbl{\t}{x}{y}$. However, note that
$\tlbl{\t}{x}{y}$ and $\tlbl{\t}{x'}{y'}$ may be different \kl{idempotents}.

\AP Let us illustrate how one can use \kl{forward Ramseyan splits} to compute
efficiently the value of $\tlbl{\t}{x}{y}$ for any pair of nodes $x \treelt[\t]
y$. There are essentially three cases to consider. We say that two nodes $x$
and $y$ are \intro{connected at level $k$} if $\spt(x:y) > k$. We say that two
nodes $x$ and $y$ are \intro{one-separated at level $k$} if $\spt(x:y) = k$ and
there exists a unique node $z$ such that $x \treelt[\t] z \treelt[\t] y$ and
$\spt(z) = k$. Finally, we say that $x$ and $y$ are \intro{independent at level
$k$} if $\spt(x:y) = k$ and there exists three nodes $z_1, z_2, z_3$ such that
$x \treelt[\t] z_1 \treelt[\t] z_2 \treeleq[\t] z_3 \treelt[\t] y$ and
$\spt(z_1) = \spt(z_2) = \spt(z_3) = k$, and $\spt(x:z_1) > k$, $\spt(z_1:z_2)
> k$, and $\spt(z_3:y) > k$. In the case of \kl{independence at level $k$}, we
can use the property of \kl{forward Ramseyan splits} to replace the middle part
of the path from $x$ to $y$ by a shorter path, as illustrated in
\cref{fast-computation:fig}. This is the core of the following lemma allowing
for a fast (first-order computable) computation of the value $\tlbl{\t}{x}{y}$
given a \kl{forward Ramseyan split} \cite[Lemma 3]{COLC07}.

\begin{lemma}[{\cite[Lemma 3]{COLC07}}]
  \label{lem:fast-computation}
  Let $\aTree$ be a tree equipped with a \kl{forward Ramseyan split} $\spt$ of
  height $N$. Let $\sigma$ be the map that associates to a node 
  of the tree the label (in $M$) of its (unique) incoming edge in $\aTree$ if it exists.
  Let us define 
  $\llbracket x:y \rrbracket_k$ for $x \treelt[\t] y$ by (reverse) induction on $k \in \{1, \ldots, N+1\}$ as follows:
  \begin{align*}
    \llbracket x: y \rrbracket_{N+1} 
                                 & = \sigma(y) & \\
    \llbracket x: y \rrbracket_k &= 
                    \llbracket x : y \rrbracket_{k+1} \quad \text{if \kl{connected}} \\
    \llbracket x: y \rrbracket_k 
                      &= \llbracket x : z \rrbracket_{k+1} \cdot
      \llbracket z : y \rrbracket_{k+1} \quad \text{\kl{one-separated} by } z \\
    \llbracket x: y \rrbracket_k 
      &= \llbracket x : z_1 \rrbracket_{k+1}
      \cdot
      \llbracket z_1 : z_2 \rrbracket_{k+1}
      \cdot 
      \llbracket z_3 : y \rrbracket_{k+1} \\ & \quad 
      \text{if } z_1, z_2, z_3 \text{ witness \kl{independence} at level } k
  \end{align*}
  Then, for every $x \treelt[\t] y$, we have $\tlbl{\t}{x}{y} =
  \llbracket x:y \rrbracket_1$.
\end{lemma}

\begin{figure}
    \centering
    \begin{tikzpicture}[xscale=0.75, yscale=0.7]
        \node (s) at (-0.5,-1) {$\spt \colon $};
        \node (x) at (0,0) {$x$};
        \node[A4] (z1) at (2,0) {$z_1$};
        \node[A4] (z2) at (4,0) {$z_2$};
        \node (z3) at (6,0) {$\cdots$};
        \node[A4] (z4) at (8,0) {$z_3$};
        \node (y) at (10,0) {$y$};

        \node[A4] at (2,-1) {$k$};
        \node[A4] at (4,-1) {$k$};
        \node[A4] at (8,-1) {$k$};
        \node (xz1) at (1,-1) {$> k$};
        \node (z1z2) at (3,-1) {$> k$};
        \node (z4y) at (9,-1) {$> k$};
        \node (z2z3) at (6,-1) {$\geq k$};

        \draw (0.2,-0.5) rectangle (9.8,-1.5);
        \draw (1.7,-0.5) -- (1.7, -1.5);
        \draw (2.3,-0.5) -- (2.3, -1.5);

        \draw (3.7,-0.5) -- (3.7, -1.5);
        \draw (4.3,-0.5) -- (4.3, -1.5);

        \draw (7.7,-0.5) -- (7.7, -1.5);
        \draw (8.3,-0.5) -- (8.3, -1.5);


        \draw[->] (x) -- 
        node[midway, above] {$\tlbl{\aTree}{x}{z_1}$}
        (z1);
        \draw[->] (z1) --
        node[midway, above] {$\tlbl{\t}{z_1}{z_2}$}
        (z2);
        \draw[->] (z2) -- (z3);
        \draw[->] (z3) -- (z4);
        \draw[->] (z4) -- 
        node[midway, above] {$\tlbl{\t}{z_3}{y}$}
        (y);

        \draw[->,A2] (z1) -- (2, 1.5) -- (8, 1.5) -- (z4);
        \node at (5, 2) {$\tlbl{\t}{z_1}{z_2} = \tlbl{\t}{z_1}{z_3}$};
    \end{tikzpicture}
    \caption{Fast computation of the value $\tlbl{\t}{x}{y}$ provided a \kl{forward Ramseyan split},
    using the fact that $x$ and $y$ are \kl{independent at level $k$}.}
    \label{fast-computation:fig}
\end{figure}

\AP The main theorem of \cite{COLC07} is that for every finite monoid $M$,
there exists a finite depth $N$ such that for every tree $T$ labelled using
$M$, there exists a forward Ramseyan split of height $N$ for $T$. We can
therefore assume that our trees are always equipped with a forward Ramseyan
split of height $N$. 
