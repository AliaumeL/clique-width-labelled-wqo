\clearpage
\section{From Interpretations to Nested Trees}
\label{sec:ramseyan}

As a first step towards proving \cref{main:theorem}, we are going to
transform a \kl{simple $\MSO$-interpretation} from trees to graphs into a more
combinatorial object that will make the interpretation \emph{quantifier-free}.
To that end, we will chain two standard transformations from automata theory:
the first one is to transform an \kl{$\MSO$-interpretation} into a so-called
\kl{monoid interpretation}, and the second one is to factorise the trees using
\kl{forward Ramseyan splits} (nested trees) \cite{COLC07}. Finally, we will
show that the natural ordering on \kl{nested trees} (the \kl{gap embedding}) is
\emph{almost} such that the resulting interpretation is order preserving from
\kl{nested trees} to graphs.

\AP  Let us fix for the rest of the section a finite alphabet $\Sigma$ and a
\kl{simple $\MSO$-interpretation} $\someInterp$ from $\Trees{\Sigma}{}$ to
finite undirected graphs defined by a formula $\varphi(x,y)$ that defines the
edges of the graphs.



\subsection{From Interpretation to Monoids}

\AP It follows from standard arguments that to a formula $\varphi(x,y)$ over
$\Trees{\Sigma}{}$ one can associate a finite monoid $M$ and a morphism $\mu
\colon \Sigma^* \to M$ such that for any tree $T$ and any pair of leaves $x
\treesibleq y$ in $T$, whether $T \models \varphi(x,y)$ is entirely determined
by the values of $\mu(T[z \colon x])$, $\mu(T[z \colon y])$ and
$\mu(T[\treeRoot \colon z])$ where $z = \lca(x,y)$ is the \kl{least common
ancestor} of $x$ and $y$. We refer to the book on Tree Automata Techniques and
Applications for a comprehensive overview of the connection between MSO on
trees, and tree automata \cite{TATA08}.
In the rest of the section, we will fix such a monoid $M$ and morphism $\mu$. 
To simplify notations, we will 
write $\tlbl{T}{x}{y}$ for $\mu(T[x \colon y])$, that is, the 
product of the labels of the edges on the path from $x$ to $y$ in $T$.

\AP As a consequence, one can collect in a subset $P \subseteq M^3$ all the
triples $(a,b,c)$ such that this triple corresponds to a satisfying assignment
of the formula $\varphi(x,y)$. This is illustrated in
\cref{interpretation-to-monoid:fig}.
We define a \intro{monoid interpretation} from trees
labelled over $M$ to finite undirected graphs to be a tuple 
$(\Sigma, \mu, M, P)$ where $\Sigma$ is a finite alphabet,
$\mu \colon \Sigma^* \to M$ is a morphism to a finite monoid $M$,
and $P \subseteq M^3$ is a subset of triples of elements of $M$.
The interpretation works as follows: given a tree $T$ whose edges are labelled
by elements of $\Sigma$, the vertices of the interpreted graph are the leaves of
$T$, and there is an edge between two leaves $x \treesibleq y$ if and only if
the triple $(\tlbl{T}{\treeRoot}{\lca(x,y)},
\tlbl{T}{\lca(x,y)}{x}, \tlbl{T}{\lca(x,y)}{y})$ is in $P$.

Adding the function $\tlbl{T}{x}{y}$ to the signature of trees, and assuming 
that $\lca$ is also part of the signature, we can rewrite the formula
$\varphi(x,y)$ as a quantifier-free formula as follows:
\begin{equation}
    \label{monoid-interpretation:eq}
    (\tlbl{T}{\treeRoot}{\lca(x,y)},
     \tlbl{T}{\lca(x,y)}{x}, \tlbl{T}{\lca(x,y)}{y}) \in P
\end{equation}

\begin{figure}
    \centering
    \begin{tikzpicture}[
        leaf/.style={
            color=Prune
        },
        lca/.style={
            color=A1
        },
        edge/.style={
            color=A2
        },
        root/.style={
            color=Prune
        },
        gedge/.style={
            color=A4
        },
        ]
        \node[root] (root) at (0,1) {$\treeRoot$};
        \node[leaf] (x) at (-1,-1) {$x$};
        \node[leaf] (y) at (1,-1) {$y$};
        \node[lca] (lca) at (0,0) {$\lca(x,y)$};
        \coordinate (tl) at ($(x.south west)+(-0.2,0)$);
        \coordinate (tr) at ($(y.south east)+(0.2,0) $);
        \coordinate (t)  at ($(root.north)+(0, 1)$);
        \draw[edge,->] (root) to node[midway, left]        {$a$}  (lca);
        \draw[edge,->] (lca)  to node[midway, above  left] {$b$} (x);
        \draw[edge,->] (lca)  to node[midway, above right] {$c$} (y);
        \draw (tl) -- (tr) -- (t) -- cycle;

        \node (maps-to) at (2,0) {$\mapsto$};

        \begin{scope}[xshift=2cm]
            \node[leaf] (nx) at (2,-1) {$x$};
            \node[leaf] (ny) at (2,2) {$y$};
            \draw[gedge] (nx) to node[midway, above, rotate=90] {$(a,b,c) \in P$?} (ny);
        \end{scope}
    \end{tikzpicture}
    \caption{The interpretation of a tree using a monoid and an accepting part $P \subseteq M^3$.}
    \label{interpretation-to-monoid:fig}
\end{figure}

The main idea driving this paper is that understanding when a class of graphs
is \kl{$\forall$-well-quasi-ordered} can be reduced to finding a suitable
\kl{well-quasi-order} on trees such that the interpretation $\someInterp$ is
\emph{order-preserving} from trees to graphs, leveraging the following standard
\cref{fact:surjective-wqo}.

\begin{fact}
  \label{fact:surjective-wqo}
  Let $(X, \leq_X)$ and $(Y, \leq_Y)$ be two quasi-orders,
  and $f \colon X \to Y$ be a surjective order-preserving map.
  If $(X, \leq_X)$ is \kl{well-quasi-ordered},
  then $(Y, \leq_Y)$ is \kl{well-quasi-ordered}.
\end{fact}

\AP One candidate ordering on trees is the \intro{composition ordering} defined
as follows. Given two trees $\aTree_1$ and $\aTree_2$ labelled over a monoid
$M$, we say that $\aTree_1$ is less than $\aTree_2$ in the \kl{composition
ordering}, written $\aTree_1 \cmpleq \aTree_2$, if there exists a map $h \colon
\aTree_1 \to \aTree_2$ that maps leaves to leaves, respects the relation
$\lca$, $\treesibleq$, and the function $\tlbl{}{\cdot}{\cdot}$. This
ordering is naturally extended to node-labelled trees by asking that for every
node $x$ in $\aTree_1$, the label of $x$ is less or equal than the label of
$h(x)$ in $\aTree_2$. It is straightforward to check that the interpretation
$\someInterp$ is order-preserving from trees equipped with the \kl{composition
ordering} to graphs equipped with the \kl{induced subgraph} ordering. Hence, to
prove that the image of $\someInterp$ is \kl{$\forall$-well-quasi-ordered}, it
suffices to prove that the class of trees is \kl{well-quasi-ordered} under the
\kl{composition ordering} when labelled using a \kl{well-quasi-ordered} set
$(X, \leq_X)$.


\AP  Historically, this  was (although not explicitly) the approach taken by
\cite{DING92,DRT10}. Unfortunately, the composition ordering on trees is more often
than not way more strict than the \kl{induced subgraph} ordering. It was
already observed that the composition ordering on trees is
\kl{well-quasi-ordered} if and only if \emph{for every} $P \subseteq M^3$, the
class of graphs obtained from trees by considering only the triples in $P$ is
\kl{labelled-well-quasi-ordered} \cite[Theorem 24]{LOPEZ24}. To understand what happens for
a precise choice of $P$, we need to have a finer understanding of the way trees
can be decomposed.


\subsection{Forward Ramseyan Splits}

\def\t{\aTree} \AP The combinatorial ingredient allowing us to  further
decompose the trees will come from an adaptation of the classical Simon
Factorisation Theorem for semigroups \cite{SIMO90}, adapted to trees by
Colcombet \cite{COLC07}. The rest of this section is devoted to explaining how
this factorisation works, and how it can be used to define a more suitable
ordering on trees.

\AP A \intro{split of height $N$} of a tree $\t$ is a mapping $\spt$ from the
nodes of $T$ to $\set{1, \dots, N}$. Given a split $\spt$ and two nodes $x
\treelt[\t] y$, we define $\spt(x \colon y)$ to be the minimal value of
$\spt(z)$ for $x \treelt[\t] z \treelt[\t] y$, and $N+1$ otherwise. Two
elements $x \treelt[\t] y$ such that $\spt(x) = \spt(y) = k$ are
\intro{$k$-neighbours} if $\spt(x \colon y) \geq k$. Note that a \kl{split}
induces some kind of hierarchical structure on the tree based on the
\kl{$k$-neighbourhoods}. We illustrate the situation on a branch of a tree in
\cref{split-on-branch:fig}.

\begin{figure}
  \centering 
  \begin{tikzpicture}
  %
  % s : 3 2 1 2 2 1 1 3 2 1 3
  %     ---------------------
  %       -----------  ----
  %         -     --      - 
    \node (s) at (0,0) {$\spt \colon$};
    \foreach \i/\s in {1/1, 2/2, 3/3, 4/2, 5/2, 6/3, 7/3, 8/1, 9/2, 10/3, 11/1} {
      % x : \i * 0.5 
      % y : (3 -\s) * 0.5
      \pgfmathsetmacro{\x}{\i * 0.7}
      \pgfmathsetmacro{\y}{0}
      \node (x\i) at (\x,\y) {$\s$};
    };

    % draw edges of the branch
    \foreach \i in {1,...,10} {
      \pgfmathsetmacro{\xone}{\i}
      \pgfmathsetmacro{\xtwo}{int(\i + 1)}
      \draw[->] (x\xone) -- (x\xtwo);
    }

    % Draw 3-neighbourhoods
    % 1 / 8 / 11
    \draw[A4] (x1) to[bend left=70] (x8);
    \draw[A4] (x8) to[bend left=70] (x11);
    % Draw 2-neighbourhoods
    \draw[A5] (x2) to[bend left=50] (x4);
    \draw[A5] (x4) to[bend left=50] (x5);
    % Draw 1-neighbourhoods
    \draw[A3] (x6) to[bend left=30] (x7);
  \end{tikzpicture}

  \caption{A split $\spt$ on a branch of a tree. The arcs 
  in \textcolor{A4}{dark blue} connect $1$-neighbours, 
  the arcs in \textcolor{A5}{light blue} connect $2$-neighbours,
  and the arcs in \textcolor{A3}{light purple} connect $3$-neighbours. Full neighbourhoods
  are obtained by considering the connected components induced by the (reflxive closure) of
  arcs of each color.}
  \label{split-on-branch:fig}
\end{figure}


\AP A split $\spt$ is \intro{forward Ramseyan} if for every $k \in \set{1,
\dots, N}$ and every $x, y, x', y'$ in the same class of \kl{$k$-neighbourhood}
with $x \treelt[\t] y$ and $x' \treelt[\t] y'$, we have: 
\begin{equation}
  \label{fake-idempotent:eq} 
  \tlbl{\t}{x}{y} = \tlbl{\t}{x}{y} \cdot \tlbl{\t}{x'}{y'} \quad . 
\end{equation} 

\AP In particular, $\tlbl{\t}{x}{y}$ is always an \intro{idempotent}:
$\tlbl{\t}{x}{y} \cdot \tlbl{\t}{x}{y} = \tlbl{\t}{x}{y}$. However, note that
$\tlbl{\t}{x}{y}$ and $\tlbl{\t}{x'}{y'}$ may be different \kl{idempotents}.

\AP Let us illustrate how one can use \kl{forward Ramseyan splits} to compute
efficiently the value of $\tlbl{\t}{x}{y}$ for any pair of nodes $x \treelt[\t]
y$. We say that $x$ and $y$ are \intro{independent at level $k$} if $\spt(x:y)
= k$ and there exists three nodes $z_1, z_2, z_3$ such that $x \treelt[\t] z_1
\treelt[\t] z_2 \treeleq[\t] z_3 \treelt[\t] y$ and $\spt(z_1) = \spt(z_2) =
\spt(z_3) = k$, and $\spt(x:z_1) > k$, $\spt(z_1:z_2) > k$, and $\spt(z_3:y) >
k$. In the case of \kl{independence at level $k$}, we can use the property of
\kl{forward Ramseyan splits} to replace the middle part of the path from $x$ to
$y$ by a shorter path, as illustrated in \cref{fast-computation:fig}. This can
be used to provide a fast (and first-order definable) computation of the value
$\tlbl{\t}{x}{y}$ given a \kl{forward Ramseyan split} as proven in \cite[Lemma
3]{COLC07}.


\begin{figure}
    \centering
    \begin{tikzpicture}[xscale=0.75, yscale=0.7]
        \node (s) at (-0.5,-1) {$\spt \colon $};
        \node (x) at (0,0) {$x$};
        \node[A4] (z1) at (2,0) {$z_1$};
        \node[A4] (z2) at (4,0) {$z_2$};
        \node (z3) at (6,0) {$\cdots$};
        \node[A4] (z4) at (8,0) {$z_3$};
        \node (y) at (10,0) {$y$};

        \node[A4] at (2,-1) {$k$};
        \node[A4] at (4,-1) {$k$};
        \node[A4] at (8,-1) {$k$};
        \node (xz1) at (1,-1) {$> k$};
        \node (z1z2) at (3,-1) {$> k$};
        \node (z4y) at (9,-1) {$> k$};
        \node (z2z3) at (6,-1) {$\geq k$};

        \draw (0.2,-0.5) rectangle (9.8,-1.5);
        \draw (1.7,-0.5) -- (1.7, -1.5);
        \draw (2.3,-0.5) -- (2.3, -1.5);

        \draw (3.7,-0.5) -- (3.7, -1.5);
        \draw (4.3,-0.5) -- (4.3, -1.5);

        \draw (7.7,-0.5) -- (7.7, -1.5);
        \draw (8.3,-0.5) -- (8.3, -1.5);


        \draw[->] (x) -- 
        node[midway, above] {$\tlbl{\aTree}{x}{z_1}$}
        (z1);
        \draw[->] (z1) --
        node[midway, above] {$\tlbl{\t}{z_1}{z_2}$}
        (z2);
        \draw[->] (z2) -- (z3);
        \draw[->] (z3) -- (z4);
        \draw[->] (z4) -- 
        node[midway, above] {$\tlbl{\t}{z_3}{y}$}
        (y);

        \draw[->,A2] (z1) -- (2, 1.5) -- (8, 1.5) -- (z4);
        \node at (5, 2) {$\tlbl{\t}{z_1}{z_2} = \tlbl{\t}{z_1}{z_3}$};
    \end{tikzpicture}
    \caption{Fast computation of the value $\tlbl{\t}{x}{y}$ provided a \kl{forward Ramseyan split},
    using the fact that $x$ and $y$ are \kl{independent at level $k$}.}
    \label{fast-computation:fig}
\end{figure}

\AP The main theorem of \cite{COLC07} is that for every finite monoid $M$,
there exists a finite depth $N$ such that for every tree $T$ labelled using
$M$, there exists a forward Ramseyan split of height $N$ for $T$. We can
therefore assume that our trees are always equipped with a forward Ramseyan
split of height $N$. 

\AP Let us now state the main remark that will guide the rest of this paper:
given a branch of a tree equipped with a \kl{forward Ramseyan split} as the one
illustrated in \cref{fast-computation:fig}, the value of $\tlbl{\t}{x}{y}$ does
not change when inserting new nodes in the section between $z_1$ and $z_3$ as
long as these new nodes are assigned a value of the split bigger than or equal
to $k$, and the resulting tree is still \kl{forward Ramseyan}. A fortiori,
whether there is an edge between two leaves of $x$ and $y$ in the resulting
graph is not influenced by such insertions. This will guide our definition of a
suitable ordering on trees, where we allow many insertions (being coarser than
the \kl{composition ordering}) while still preserving the edges between
``most'' leaves in the interpreted graphs.

\subsection{Marked Nested Trees}
\label{sec:marked-nested-trees}

\newcommand{\marked}{M}
\newcommand{\separating}{S}
\newcommand{\dummy}{D}
\newcommand{\marking}{\rho}

\begin{definition}
  A \intro{marked nested tree} is a tuple $(\t, \spt, \marking)$ where
  $\t$ is a tree whose edges are labelled over a finite monoid $M$,
  $\spt$ is a forward Ramseyan split of height $N$ of $\t$,
  and $\marking$ is a function from the nodes of $\t$ to $\set{\marked, \separating, \dummy}$, 
  respectively called \intro(nodes){marked}, \intro(nodes){separating}, and \intro(nodes){dummy} nodes.

  A \kl{marked nested tree} $(\t, \spt, \marking)$ is \intro{well-marked} if
  the \kl{marked nodes} contain the root of $\t$, are closed under 
  the \kl{least common ancestor} operation, and furthermore satisfy that
  for every $x \treelt[\t] z_1 \treelt[\t] y$ such that 
  $\spt(x:z_1) > k$, $\spt(z_1) = k$, and  $\spt(z_1:y) \geq k$ for some $k \in \set{1, \dots, N}$,
  if $x$ and $y$ are \kl{marked nodes} and $z$ is not a \kl{marked node}, then there exists $z_2, z_3$
  such that 
  $x \treelt[\t] z_1 \treelt[\t] z_2 \treeleq[\t] z_3 \treeleq[\t] y$,
  $\spt(z_1) = \spt(z_2) = \spt(z_3) = k$, and 
  $\spt(x:z_1) > k$, $\spt(z_1:z_2) > k$, and $\spt(z_3:y) \geq k$, 
  with $\marking(z_1) = \marked$, and $\marking(z_2) \in \set{\marked, \separating}$.
\end{definition}

\NewDocumentCommand{\gemb}{}{\leq_{\mathrm{gap}}}

\begin{definition}
  \label{def:gap-embedding}
  Given two \kl{marked nested trees}
  $(\t_1, \spt_1, \marking_1)$ and
  $(\t_2, \spt_2, \marking_2)$,
  we say that $(\t_1, \spt_1, \marking_1)$ \intro{gap-embeds} into
  $(\t_2, \spt_2, \marking_2)$, written
  $(\t_1, \spt_1, \marking_1) \gemb (\t_2, \spt_2, \marking_2)$,
  if there exists a mapping $h \colon \t_1 \to \t_2$ that
  \begin{enumerate}
    \item is a tree embedding: respects the ancestor relation $\treeleq$, least
      common ancestors $\lca$, and maps leaves to leaves,
    \item respects the marking: for every node $x$ in $\t_1$,
      $\marking_1(x) = \marking_2(h(x))$,
    \item respects the split: for every node $x$ in $\t_1$,
      $\spt_1(x) = \spt_2(h(x))$,
      and $\spt_1(x:y) > k$ implies $\spt_2(h(x):h(y)) \geq k$,
    \item is strict on marked nodes: for every pair of nodes
      $x \treelt[\t_1] y$ in $\t_1$ such that $\marking_1(x) = \marked$
      and $\marking_1(y) \in \set{\marked, \separating}$,
      $\spt_1(x:y) = \spt_2(h(x):h(y))$.
    \item respects local products: if $y$ is the immediate left (resp. right)
      child of $x$ in $\t_1$, and $y'$ is the immediate left (resp. right) child of
      $h(x)$ in $\t_2$, then
      $\tlbl{\t_1}{x}{y} = \tlbl{\t_2}{h(x)}{y'}$.
    \item respects neighbourhood products: for every $k \in \set{1, \dots, N}$,
      and every node $x \in \t_1$, if $z$ is the least ancestor of $x$ such that
      $\spt_1(z) = k$ and $z'$ is the least ancestor of $h(x)$ such that
      $\spt_2(z') = k$, then 
      $\tlbl{\t_1}{z}{x} = \tlbl{\t_2}{z'}{h(x)}$. 
  \end{enumerate}
\end{definition}

\begin{lemma}
  \label{lem:gap-embedding-respects-products}
  Let $(\t_1, \spt_1, \marking_1)$ and
  $(\t_2, \spt_2, \marking_2)$ be two \kl{marked nested trees}
  such that
  $(\t_1, \spt_1, \marking_1) \gemb (\t_2, \spt_2, \marking_2)$ 
  via some mapping $h$.
  For every pair of \kl{marked nodes} $x \treelt[\t_1] y$ in $\t_1$,
  we have $\tlbl{\t_1}{x}{y} = \tlbl{\t_2}{h(x)}{h(y)}$.
\end{lemma}
\begin{proof}
  We proceed by induction on the value of $\spt_1(x:y)$.
  If $\spt(x:y) = N+1$, then $y$ is an immediate child of $x$,
  and $\spt(h(x):h(y)) = N+1$ by definition of \kl{gap-embedding}.
  Therefore, by point (5) of \cref{def:gap-embedding},
  we have $\tlbl{\t_1}{x}{y} = \tlbl{\t_2}{h(x)}{h(y)}$.

  Otherwise, let $k = \spt_1(x:y) \leq N$. We distinguish two cases.
  Either, there exists a unique child $z$ of $x$ such that
  $\spt_1(x:z) = k$ and $z$ is a \kl{marked node}.
  By induction hypothesis, we have
  $\tlbl{\t_1}{z}{y} = \tlbl{\t_2}{h(z)}{h(y)}$,
  and $\tlbl{\t_1}{x}{z} = \tlbl{\t_2}{h(x)}{h(z)}$. Hence, 
  $\tlbl{\t_1}{x}{y} = \tlbl{\t_1}{x}{z} \cdot \tlbl{\t_1}{z}{y}
  = \tlbl{\t_2}{h(x)}{h(z)} \cdot \tlbl{\t_2}{h(z)}{h(y)}
  = \tlbl{\t_2}{h(x)}{h(y)}$.

  Otherwise, since $(\t_1, \spt_1, \marking_1)$ is \kl{well-marked},
  there exists $z_1, z_2, z_3$ such that
  $x \treelt[\t_1] z_1 \treelt[\t_1] z_2 \treeleq[\t_1] z_3 \treelt[\t_1] y$,
  $\spt_1(z_1) = \spt_1(z_2) = \spt_1(z_3) = k$, and 
  $\spt_1(x:z_1) > k$, $\spt_1(z_1:z_2) > k$, and $\spt_1(z_3:y) > k$, 
  with $\marking_1(z_1) = \marked$, and $\marking_1(z_2) \in \set{\marked, \separating}$.

  \begin{center}
    \begin{tikzpicture}
        \node (x) at (0,0) {$x$};
        \node (z1) at (2,0) {$z_1$};
        \node (z2) at (4,0) {$z_2$};
        \node (z3) at (6,0) {$z_3$};
        \node (y) at (8,0) {$y$};

        \node at (1,-0.2) {$> k$};
        \node at (3,-0.2) {$> k$};
        \node at (5,-0.2) {$\geq k$};
        \node at (7,-0.2) {$> k$};

        \node[below=0.02cm of z1] {$k$};
        \node[below=0.02cm of z2] {$k$};
        \node[below=0.02cm of z3] {$k$};

        \draw[->] (x) --
        node[midway, above] {$\tlbl{\t_1}{x}{z_1}$}
        (z1);
        \draw[->] (z1) --
        node[midway, above] {$\tlbl{\t_1}{z_1}{z_2}$}
        (z2);
        \draw[->] (z2) --
        node[midway, above] {$\tlbl{\t_1}{z_2}{z_3}$}
        (z3);
        \draw[->] (z3) --
        node[midway, above] {$\tlbl{\t_1}{z_3}{y}$}
        (y);
    \end{tikzpicture}
  \end{center}

  Let us now consider $z_3'$ the least ancestor of $h(y)$ such that
  $\spt_2(z_3') = k$. This node exists because $h$ respects the split
  (point (3) of \cref{def:gap-embedding}).
  This yields the following configuration:

  \begin{center}
    \begin{tikzpicture}
        \node (hx) at (0,0) {$h(x)$};
        \node (hz1) at (2,0) {$h(z_1)$};
        \node (hz2) at (4,0) {$h(z_2)$};
        \node (hz3) at (6,0) {$z_3'$};
        \node (hy) at (8,0) {$h(y)$};

        \node at (1,-0.2) {$> k$};
        \node at (3,-0.2) {$> k$};
        \node at (5,-0.2) {$\geq k$};
        \node at (7,-0.2) {$> k$}; 
        \node[below=0.02cm of hz1] {$k$};
        \node[below=0.02cm of hz2] {$k$};
        \node[below=0.02cm of hz3] {$k$};

        \draw[->] (hx) --
        node[midway, above] {$\tlbl{\t_2}{h(x)}{h(z_1)}$}
        (hz1);
        \draw[->] (hz1) --
        node[midway, above] {$\tlbl{\t_2}{h(z_1)}{h(z_2)}$}
        (hz2);
        \draw[->] (hz2) --
        node[midway, above] {$\tlbl{\t_2}{h(z_2)}{z_3'}$}
        (hz3);
        \draw[->] (hz3) --
        node[midway, above] {$\tlbl{\t_2}{z_3'}{h(y)}$}
        (hy);
    \end{tikzpicture}
  \end{center}  

  Indeed, by point (3) of \cref{def:gap-embedding},
  we have $\spt_2(h(x):h(y)) \geq k$, and 
  $\spt_2(h(z_1)) = \spt_2(h(z_2)) = \spt_2(h(z_3)) = k$. Furthermore,
  since $x$, $y$, and $z_1$ are \kl{marked nodes}, and because
  $\marking(z_2) \in \set{\marked, \separating}$, we conclude from 
  point (4) of \cref{def:gap-embedding} that
  $\spt_1(x:z_1) = \spt_2(h(x):h(z_1)) > k$,
  $\spt_1(z_1:z_2) = \spt_2(h(z_1):h(z_2)) > k$.

  Now, by induction hypothesis, we have
  $\tlbl{\t_1}{z_1}{z_2} = \tlbl{\t_2}{h(z_1)}{h(z_2)}$.
  Furthermore, by point (6) of \cref{def:gap-embedding},
  we have $\tlbl{\t_1}{z_3}{y} = \tlbl{\t_2}{z_3'}{h(y)}$
  and $\tlbl{\t_1}{z_1}{z_2} = \tlbl{\t_2}{h(z_1)}{h(z_2)}$.
  Finally, because $x$ and $y$ are \kl{independent at level $k$},
  we have by \cref{fake-idempotent:eq} that
  \begin{align*}
    \tlbl{\t_1}{x}{y}
    & = \tlbl{\t_1}{x}{z_1} \cdot \tlbl{\t_1}{z_1}{z_2} \cdot \tlbl{\t_1}{z_3}{y} \\
    & = \tlbl{\t_2}{h(x)}{h(z_1)} \cdot \tlbl{\t_2}{h(z_1)}{h(z_2)} \cdot \tlbl{\t_2}{z_3'}{h(y)} \\
    & = \tlbl{\t_2}{h(x)}{h(y)} \quad . 
  \end{align*}
  We conclude the proof.
\end{proof}


\begin{corollary}
  \label{cor:gap-embedding-monotone}
  Let $(\t_1, \spt_1, \marking_1)$ and
  $(\t_2, \spt_2, \marking_2)$ be two \kl{well-marked nested trees}
  such that
  $(\t_1, \spt_1, \marking_1) \gemb (\t_2, \spt_2, \marking_2)$.
  Then, the graph interpreted from
  $(\t_1, \spt_1, \marking_1)$
  is an induced subgraph of the graph interpreted from
  $(\t_2, \spt_2, \marking_2)$
  when considering only \kl{marked leaves} as vertices.
\end{corollary}
\begin{proof}
  Follows directly from \cref{lem:gap-embedding-respects-products}
  and the definition of the interpretation using the set $P \subseteq M^3$,
  because the edges between two \kl{marked leaves} $x$ and $y$
  only depend on the values of
  $\tlbl{\t}{\treeRoot}{\lca(x,y)}$,
  $\tlbl{\t}{\lca(x,y)}{x}$, and
  $\tlbl{\t}{\lca(x,y)}{y}$, all of which are \kl{marked nodes}.
\end{proof}

\AP
A \kl{$L$-bounded marked nested tree} is a \kl{marked nested tree}
such that for every $k$, for every 
sequence $x_1 \treelt[\t] x_2 \treelt[\t] \dots \treelt[\t] x_n$ of \kl{marked nodes} with
$\spt(x_i : x_{i+1}) > k$ for every $i$, and $\spt(x_i) = k$ for every $i$,
we have $n \leq L$.

\begin{theorem}
  \label{thm:marked-nested-trees-wqo}
  For every finite $L$,
  the class of \kl(tree){$L$-bounded} \kl{well-marked nested trees} labelled over a
  \kl{well-quasi-ordered} set $(X, \leq_X)$
  is \kl{well-quasi-ordered} under the \kl{gap-embedding} ordering.
\end{theorem}

The proof of \cref{thm:marked-nested-trees-wqo}
follows directly from an earlier result of Dershowitz and Tzameret
\cite[Section 3.3 Path Comparable Trees]{DERSHOWITZ200380}, stating that the class of \kl{marked trees}
is \kl{well-quasi-ordered} under a weaker variant of the \kl{gap-embedding} ordering 
obtained by removing points (2) and (4) to (6) of \cref{def:gap-embedding}.


\begin{proof}
  The idea is to encode conditions (2) and  (4) to (6) of \cref{def:gap-embedding}
  by adding new labels on the nodes of the trees.
  Let us consider a \kl{$L$-bounded marked nested tree}
  $(\t, \spt, \marking)$
  labelled over a \kl{well-quasi-ordered} set $(X, \leq_X)$.

  We define a new labelling of the nodes of $\t$ as follows: we add to each node $x$ of $\t$
  the label $\tlbl{\t}{z}{x}$ for every $k \in \set{1, \dots, N}$,
  where $z$ is the least ancestor of $x$ such that $\spt(z) = k$,
  or a special symbol $\bot$ if no such ancestor exists.
  Furthermore, if $y$ is the immediate left (resp. right) child of $x$ in $\t$,
  we also add the label $\tlbl{\t}{x}{y}$ to $x$,
  or the special symbol $\bot$ if no such child exists.
  Finally, we also add the label $\marking(x)$ to $x$.
  Since $M$ is finite, the new labelling is still over a \kl{well-quasi-ordered} set.



  To encode condition (4) of \cref{def:gap-embedding}, we will
  crucially use the fact that the trees are \kl{$L$-bounded}.
  To that end, we change the split $\spt$ into a new split $\spt'$ defined as follows:
  for every \kl{marked node} $x$ in $\t$, if $\spt(x) = k$, we set $\spt'(x) = k \times (L+3) + i$,
  where $i$ is the index of $x$ in the sequence of \kl{marked nodes}
  with split value $k$ ordered by the ancestor relation $\treelt[\t]$.
  For every \kl{separating node} $x$ in $\t$, if $\spt(x) = k$, we set $\spt'(x) = k \times (L+3) + L + 1$.
  Finally, for every \kl{dummy node} $x$ in $\t$, if $\spt(x) = k$, we set $\spt'(x) = k \times (L+3)$.
  Since the tree is \kl{$L$-bounded}, the new split $\spt'$ has height at most $N \times (L+3)$.


  Now, assume that $(\t_1, \spt_1, \marking_1)$ and
  $(\t_2, \spt_2, \marking_2)$ are two \kl{$L$-bounded marked nested trees}
  such that
   $(\t_1', \spt_1')$ embeds into $(\t_2', \spt_2')$ by some mapping $h$,
   that respects conditions (1) and (3) of \cref{def:gap-embedding},
   and preserves the new labels.
   We claim that $h$ is a \kl{gap-embedding} from
   $(\t_1, \spt_1, \marking_1)$ to
   $(\t_2, \spt_2, \marking_2)$.

  It is clear that $h$ respects point (2) of \cref{def:gap-embedding}
  because the marking is part of the new labels. Furthermore, $h$ respects 
  points (5) and (6) of \cref{def:gap-embedding} because the relevant products
  are also part of the new labels.

  Finally, let us show that $h$ respects point (4) of \cref{def:gap-embedding}.
  To that end, let us consider two \kl{marked nodes} $x$ and $y$ in $\t_1$
  such that $x \treelt[\t_1] y$, $\spt(x) = \spt(y) = k$, and 
  $\spt(x:y) > k$. By definition of the new split $\spt_1'$,
  we have $\spt_1'(x) = k \times (L+3) + i$ and $\spt_1'(y) = k \times (L+3) + i+1$,
  for some $i < L$. Now, since $h$ respects the split,
  we have $\spt_2'(h(x)) = k \times (L+3) + i$ and $\spt_2'(h(y)) = k \times (L+3) + i+1$.
  Also, since $\spt_1(x:y) > k$, we have that $\spt_1'(x:y) > k \times (L+3)$,
  and therefore $\spt_2'(h(x):h(y)) \geq k \times (L+3)$.
  But this means that there are no nodes $z$ between $h(x)$ and $h(y)$
  such that $\spt_2(z) = k$, and therefore $\spt_2(h(x):h(y)) > k$.
  We conclude that $h$ respects point (4) of \cref{def:gap-embedding}.
  A similar reasoning applies when $\marking_1(y) = \separating$.

  In particular, 


\end{proof}