\clearpage
\section{From Interpretations to Nested Trees}
\label{sec:ramseyan}

As a first step towards proving \cref{main:theorem}, we are going to
transform a \kl{simple $\MSO$-interpretation} from trees to graphs into a more
combinatorial object that will make the interpretation \emph{quantifier-free}.
To that end, we will chain two standard transformations from automata theory:
the first one is to transform an \kl{$\MSO$-interpretation} into a so-called
\kl{monoid interpretation}, and the second one is to factorise the trees using
\kl{forward Ramseyan splits} (nested trees) \cite{COLC07}. Finally, we will
show that the natural ordering on \kl{nested trees} (the \kl{gap embedding}) is
\emph{almost} such that the resulting interpretation is order preserving from
\kl{nested trees} to graphs.

\AP  Let us fix for the rest of the section a finite alphabet $\Sigma$ and a
\kl{simple $\MSO$-interpretation} $\someInterp$ from $\Trees{\Sigma}{}$ to
finite undirected graphs defined by a formula $\varphi(x,y)$ that defines the
edges of the graphs.



\subsection{From Interpretation to Monoids}

\AP It follows from standard arguments that to a formula $\varphi(x,y)$ over
$\Trees{\Sigma}{}$ one can associate a finite monoid $M$ and a morphism $\mu
\colon \Sigma^* \to M$ such that for any tree $T$ and any pair of leaves $x
\treesibleq y$ in $T$, whether $T \models \varphi(x,y)$ is entirely determined
by the values of $\mu(\tword{z}{x})$, $\mu(\tword{z}{y})$ and
$\mu(\tword{\treeRoot}{z})$ where $z = \lca(x,y)$ is the \kl{least common
ancestor} of $x$ and $y$. We refer to the book on Tree Automata Techniques and
Applications for a comprehensive overview of the connection between Monadic
Second Order Logic on trees, and tree automata \cite{TATA08}. In the rest of
the section, we will fix such a monoid $M$ and morphism $\mu$. To simplify
notations, we will write $\intro*\tlbl{\aTree}{x}{y}$ for $\mu(\tword{x}{y})$, that
is, the product of the labels of the edges on the path from $x$ to $y$ in $\aTree$.

\AP As a consequence, one can collect in a subset $P \subseteq M^3$ all the
triples $(a,b,c)$ corresponding to a satisfying assignments
of the formula $\varphi(x,y)$. This is illustrated in
\cref{interpretation-to-monoid:fig}.
We define a \intro{monoid interpretation} from trees
labelled over $M$ to finite undirected graphs to be a tuple 
$(\Sigma, \mu, M, P)$ where $\Sigma$ is a finite alphabet,
$\mu \colon \Sigma^* \to M$ is a morphism to a finite monoid $M$,
and $P \subseteq M^3$ is a subset of triples of elements of $M$.
The interpretation works as follows: given a tree $T$ whose edges are labelled
by elements of $\Sigma$, the vertices of the interpreted graph are the leaves of
$T$, and there is an edge between two leaves $x \treesibleq y$ if and only if
the triple $(\tlbl{T}{\treeRoot}{\lca(x,y)},
\tlbl{T}{\lca(x,y)}{x}, \tlbl{T}{\lca(x,y)}{y})$ is in $P$.

Adding the function $\tlbl{T}{x}{y}$ to the signature of trees, and assuming 
that $\lca$ is also part of the signature, we can rewrite the formula
$\varphi(x,y)$ as a quantifier-free formula as follows:
\begin{equation}
    \label{monoid-interpretation:eq}
    (\tlbl{T}{\treeRoot}{\lca(x,y)},
     \tlbl{T}{\lca(x,y)}{x}, \tlbl{T}{\lca(x,y)}{y}) \in P
\end{equation}

\begin{figure}
    \centering
    \begin{tikzpicture}[
        leaf/.style={
            color=Prune
        },
        lca/.style={
            color=A1
        },
        edge/.style={
            color=A2
        },
        root/.style={
            color=Prune
        },
        gedge/.style={
            color=A4
        },
        ]
        \node[root] (root) at (0,1) {$\treeRoot$};
        \node[leaf] (x) at (-1,-1) {$x$};
        \node[leaf] (y) at (1,-1) {$y$};
        \node[lca] (lca) at (0,0) {$\lca(x,y)$};
        \coordinate (tl) at ($(x.south west)+(-0.2,0)$);
        \coordinate (tr) at ($(y.south east)+(0.2,0) $);
        \coordinate (t)  at ($(root.north)+(0, 1)$);
        \draw[edge,->] (root) to node[midway, left]        {$a$}  (lca);
        \draw[edge,->] (lca)  to node[midway, above  left] {$b$} (x);
        \draw[edge,->] (lca)  to node[midway, above right] {$c$} (y);
        \draw (tl) -- (tr) -- (t) -- cycle;

        \node (maps-to) at (2,0) {$\mapsto$};

        \begin{scope}[xshift=2cm]
            \node[leaf] (nx) at (2,-1) {$x$};
            \node[leaf] (ny) at (2,2) {$y$};
            \draw[gedge] (nx) to node[midway, above, rotate=90] {$(a,b,c) \in P$?} (ny);
        \end{scope}
    \end{tikzpicture}
    \caption{The interpretation of a tree using a monoid and an accepting part $P \subseteq M^3$.}
    \label{interpretation-to-monoid:fig}
\end{figure}

The main idea driving this paper is that understanding when a class of graphs
is \kl{$\forall$-well-quasi-ordered} can be reduced to finding a suitable
\kl{well-quasi-order} on trees such that the interpretation $\someInterp$ is
\emph{order-preserving} from trees to graphs, leveraging the following standard
\cref{fact:surjective-wqo}.

\begin{fact}
  \label{fact:surjective-wqo}
  Let $(X, \leq_X)$ and $(Y, \leq_Y)$ be two quasi-orders,
  and $f \colon X \to Y$ be a surjective order-preserving map.
  If $(X, \leq_X)$ is \kl{well-quasi-ordered},
  then $(Y, \leq_Y)$ is \kl{well-quasi-ordered}.
\end{fact}
\begin{proof}
  Let $(y_i)_{i \in \mathbb{N}}$ be an infinite sequence of elements of $Y$.
  Since $f$ is surjective, there exists an infinite sequence $(x_i)_{i \in \mathbb{N}}$
  of elements of $X$ such that for every $i$, $f(x_i) = y_i$.
  Since $(X, \leq_X)$ is \kl{well-quasi-ordered},
  there exists two indices $i < j$ such that $x_i \leq_X x_j$.
  Since $f$ is order-preserving, we have $y_i = f(x_i) \leq_Y f(x_j) = y_j$.
\end{proof}

\AP One candidate ordering on trees is the \intro{composition ordering} defined
as follows. Given two trees $\aTree_1$ and $\aTree_2$ labelled over a monoid
$M$, we say that $\aTree_1$ is less than $\aTree_2$ in the \kl{composition
ordering}, written $\aTree_1 \cmpleq \aTree_2$, if there exists a map $h \colon
\aTree_1 \to \aTree_2$ that maps leaves to leaves, respects the relation
$\lca$, $\treesibleq$, and the function $\tlbl{}{\cdot}{\cdot}$. This ordering
is naturally extended to node-labelled trees by asking that for every node $x$
in $\aTree_1$, the label of $x$ is less or equal than the label of $h(x)$ in
$\aTree_2$. It is straightforward to check that the interpretation
$\someInterp$ is order-preserving from trees equipped with the \kl{composition
ordering} to graphs equipped with the \kl{induced subgraph} ordering. Hence, to
prove that the image of $\someInterp$ is \kl{$\forall$-well-quasi-ordered}, it
suffices to prove that the class of trees is \kl{well-quasi-ordered} under the
\kl{composition ordering} when labelled using a \kl{well-quasi-ordered} set
$(X, \leq_X)$.


\AP  Historically, this  was (although not explicitly) the approach taken by
\cite{DING92,DRT10}. Unfortunately, the \kl{composition ordering} on trees is
often way stricter than the \kl{induced subgraph} ordering. It was already
observed that the \kl{composition ordering} on trees is \kl{well-quasi-ordered}
if and only if \emph{for every} $P \subseteq M^3$, the class of graphs obtained
from trees by considering only the triples in $P$ is
\kl{labelled-well-quasi-ordered} \cite[Theorem 24]{LOPEZ24}. To understand what
happens for a precise choice of $P$, we need to have a finer understanding of
the way trees can be decomposed.


\subsection{Forward Ramseyan Splits}

\def\t{\aTree} \AP The combinatorial ingredient allowing us to  further
decompose the trees will come from an adaptation of the classical Simon
Factorisation Theorem for semigroups \cite{SIMO90}, adapted to trees by
Colcombet \cite{COLC07}. The rest of this section is devoted to explaining how
this factorisation works, and how it can be used to define a more suitable
ordering on trees.

\AP A \intro{split of height $N$} of a tree $\t$ is a mapping $\spt$ from the
nodes of $T$ to $\set{1, \dots, N}$. Given a split $\spt$ and two nodes $x
\treelt[\t] y$, we define $\spt(x \colon y)$ to be the minimal value of
$\spt(z)$ for $x \treelt[\t] z \treelt[\t] y$, and $N+1$ otherwise. In
particular, this does not count the nodes $x$ and $y$ themselves.

\AP Two elements $x \treelt[\t] y$ such that $\spt(x) = \spt(y) = k$ are
\intro{$k$-neighbours} if $\spt(x \colon y) \geq k$. Note that a \kl{split}
induces some kind of hierarchical structure on the tree based on the
\kl{$k$-neighbourhoods}. We illustrate the situation on a branch of a tree in
\cref{split-on-branch:fig}. We will refer to the value of $\spt(x)$ as the
\reintro{split depth} of the node $x$.

\begin{figure}
  \centering 
  \begin{tikzpicture}
  %
  % s : 3 2 1 2 2 1 1 3 2 1 3
  %     ---------------------
  %       -----------  ----
  %         -     --      - 
    \node (s) at (0,0) {$\spt \colon$};
    \foreach \i/\s in {1/1, 2/2, 3/3, 4/2, 5/2, 6/3, 7/3, 8/1, 9/2, 10/3, 11/1} {
      % x : \i * 0.5 
      % y : (3 -\s) * 0.5
      \pgfmathsetmacro{\x}{\i * 0.7}
      \pgfmathsetmacro{\y}{0}
      \node (x\i) at (\x,\y) {$\s$};
    };

    % draw edges of the branch
    \foreach \i in {1,...,10} {
      \pgfmathsetmacro{\xone}{\i}
      \pgfmathsetmacro{\xtwo}{int(\i + 1)}
      \draw[->] (x\xone) -- (x\xtwo);
    }

    % Draw 3-neighbourhoods
    % 1 / 8 / 11
    \draw[A4] (x1) to[bend left=70] (x8);
    \draw[A4] (x8) to[bend left=70] (x11);
    % Draw 2-neighbourhoods
    \draw[A5] (x2) to[bend left=50] (x4);
    \draw[A5] (x4) to[bend left=50] (x5);
    % Draw 1-neighbourhoods
    \draw[A3] (x6) to[bend left=30] (x7);
  \end{tikzpicture}

  \caption{A split $\spt$ on a branch of a tree. The arcs 
  in \textcolor{A4}{dark blue} connect $1$-neighbours, 
  the arcs in \textcolor{A5}{light blue} connect $2$-neighbours,
  and the arcs in \textcolor{A3}{light purple} connect $3$-neighbours. Full neighbourhoods
  are obtained by considering the connected components induced by the (reflexive closure) of
  arcs of each color.}
  \label{split-on-branch:fig}
\end{figure}


\AP A \kl{split} $\spt$ is \intro{forward Ramseyan} if for every $k \in \set{1,
\dots, N}$ and every $x, y, x', y'$ in the same class of \kl{$k$-neighbourhood}
with $x \treelt[\t] y$ and $x' \treelt[\t] y'$, we have: 
\begin{equation}
  \label{fake-idempotent:eq} 
  \tlbl{\t}{x}{y} = \tlbl{\t}{x}{y} \cdot \tlbl{\t}{x'}{y'} \quad . 
\end{equation} 

\AP In particular, $\tlbl{\t}{x}{y}$ is always an \intro{idempotent}:
$\tlbl{\t}{x}{y} \cdot \tlbl{\t}{x}{y} = \tlbl{\t}{x}{y}$. However, note that
$\tlbl{\t}{x}{y}$ and $\tlbl{\t}{x'}{y'}$ may be different \kl{idempotents}.

\AP Let us illustrate how one can use \kl{forward Ramseyan splits} to compute
efficiently the value of $\tlbl{\t}{x}{y}$ for any pair of nodes $x \treelt[\t]
y$. We say that $x$ and $y$ are \intro{independent at level $k$} if $\spt(x:y)
= k$ and there exists three nodes $z_1, z_2, z_3$ such that $x \treelt[\t] z_1
\treelt[\t] z_2 \treeleq[\t] z_3 \treelt[\t] y$ and $\spt(z_1) = \spt(z_2) =
\spt(z_3) = k$, and $\spt(x:z_1) > k$, $\spt(z_1:z_2) > k$, and $\spt(z_3:y) >
k$. In the case of \kl{independence at level $k$}, we can use the property of
\kl{forward Ramseyan splits} to replace the middle part of the path from $x$ to
$y$ by a shorter path, as illustrated in \cref{fast-computation:fig}. This can
be used to provide a fast (and first-order definable) computation of the value
$\tlbl{\t}{x}{y}$ given a \kl{forward Ramseyan split} as proven in \cite[Lemma
3]{COLC07}.


\begin{figure}
    \centering
    \begin{tikzpicture}[xscale=0.75, yscale=0.7]
        \node (s) at (-0.5,-1) {$\spt \colon $};
        \node (x) at (0,0) {$x$};
        \node[A4] (z1) at (2,0) {$z_1$};
        \node[A4] (z2) at (4,0) {$z_2$};
        \node (z3) at (6,0) {$\cdots$};
        \node[A4] (z4) at (8,0) {$z_3$};
        \node (y) at (10,0) {$y$};

        \node[A4] at (2,-1) {$k$};
        \node[A4] at (4,-1) {$k$};
        \node[A4] at (8,-1) {$k$};
        \node (xz1) at (1,-1) {$> k$};
        \node (z1z2) at (3,-1) {$> k$};
        \node (z4y) at (9,-1) {$> k$};
        \node (z2z3) at (6,-1) {$\geq k$};

        \draw (0.2,-0.5) rectangle (9.8,-1.5);
        \draw (1.7,-0.5) -- (1.7, -1.5);
        \draw (2.3,-0.5) -- (2.3, -1.5);

        \draw (3.7,-0.5) -- (3.7, -1.5);
        \draw (4.3,-0.5) -- (4.3, -1.5);

        \draw (7.7,-0.5) -- (7.7, -1.5);
        \draw (8.3,-0.5) -- (8.3, -1.5);


        \draw[->] (x) -- 
        node[midway, above] {$\tlbl{\aTree}{x}{z_1}$}
        (z1);
        \draw[->] (z1) --
        node[midway, above] {$\tlbl{\t}{z_1}{z_2}$}
        (z2);
        \draw[->] (z2) -- (z3);
        \draw[->] (z3) -- (z4);
        \draw[->] (z4) -- 
        node[midway, above] {$\tlbl{\t}{z_3}{y}$}
        (y);

        \draw[->,A2] (z1) -- (2, 1.5) -- (8, 1.5) -- (z4);
        \node at (5, 2) {$\tlbl{\t}{z_1}{z_2} = \tlbl{\t}{z_1}{z_3}$};
    \end{tikzpicture}
    \caption{Fast computation of the value $\tlbl{\t}{x}{y}$ provided a \kl{forward Ramseyan split},
    using the fact that $x$ and $y$ are \kl{independent at level $k$}.}
    \label{fast-computation:fig}
\end{figure}

\AP The main theorem of \cite{COLC07} is that for every finite monoid $M$,
there exists a finite depth $N$ such that for every tree $\aTree$ edge-labelled
using $M$, there exists a forward Ramseyan split of height $N$ for $T$. We can
therefore assume that our trees are always equipped with a \kl{forward Ramseyan
split} of height $N$. 

\AP Let us now state the main remark that will guide the rest of this paper:
given a branch of a tree equipped with a \kl{forward Ramseyan split} as the one
illustrated in \cref{fast-computation:fig}, the value of $\tlbl{\t}{x}{y}$ does
not change when inserting new nodes in the section between $z_1$ and $z_3$ as
long as these new nodes are assigned a value of the split bigger than or equal
to $k$, and the resulting tree is still \kl{forward Ramseyan}. A fortiori,
whether there is an edge between two leaves of $x$ and $y$ in the resulting
graph is not influenced by such insertions. This will guide our definition of a
suitable ordering on trees in the upcoming \cref{sec:marked-nested-trees},
where we allow many insertions (being coarser than the \kl{composition
ordering}) while still preserving the edges between ``most'' leaves in the
interpreted graphs.

\subsection{Marked Nested Trees}
\label{sec:marked-nested-trees}

In this section, we will show how the notion of \kl{forward Ramseyan split} gets 
us closer to defining a \kl{well-quasi-order} on trees that is suitable for our purpose.
A first step is to notice that there already exists a notion of embedding 
that somehow respects \kl{forward Ramseyan splits}, called \kl{gap-embeddings}
and that endows trees with a \kl{well-quasi-order} \cite{DERSHOWITZ200380}.

\begin{definition}[{\cite[Definition 3.3]{DERSHOWITZ200380}}]
  A \intro{gap-embedding} between trees $\t_1, \t_2$
  endowed with \kl{forward Ramseyan splits} $\spt_1, \spt_2$
  is a mapping $h \colon \t_1 \to \t_2$ that
  \begin{enumerate}
    \item is a tree embedding: respects the ancestor relation $\treeleq$, least
      common ancestors $\lca$, and sibling ordering $\treesibleq$,
    \item satifies the root gap property:
      $\spt_2(r_2 : h(r_1)) \geq \spt_1(r_1)$ where $r_1, r_2$ are the roots of $\t_1, \t_2$,
    \item satisfies the edge gap property:
      for every edge $x \treelt[\t] y$ in $\t_1$,
      $\spt_2(h(x) : h(y)) \geq \spt_1(x)$.
  \end{enumerate}
  If the trees are node-labelled by a set $(X, \leq_X)$,
  we further require that for every node $x$ in $\t_1$,
  the label of $x$ is less or equal than the label of $h(x)$ in $\t_2$.
\end{definition}

\AP 
It follows from \cite[Theorem 3.1, Main Theorem]{DERSHOWITZ200380}
that the class of trees
equipped with \kl{forward Ramseyan splits} is \kl{well-quasi-ordered}
under the \kl{gap-embedding} ordering, even under the assumption
that nodes are labelled with a \kl{well-quasi-ordered} set $(X, \leq_X)$.
We will now devise a suitable modification of the \kl{gap-embedding} ordering
that will ensure that the interpretation $\someInterp$ is order-preserving
from trees to graphs.


\newcommand{\marked}{\mathbf{M}}
\newcommand{\separating}{\mathbf{S}}
\newcommand{\dummy}{\mathbf{D}}
\newcommand{\marking}{\rho}

\begin{definition}
  A \intro{marked nested tree} is a tuple $(\t, \spt, \marking)$ where
  $\t$ is a tree whose edges are labelled over a finite monoid $M$,
  $\spt$ is a forward Ramseyan split of height $N$ of $\t$,
  and $\marking$ is a function from the nodes of $\t$ to $\set{\marked, \separating, \dummy}$, 
  respectively called \intro(nodes){marked}, \intro(nodes){separating}, and \intro(nodes){dummy} nodes.

  A \kl{marked nested tree} $(\t, \spt, \marking)$ is \intro{well-marked} if
  the \kl{marked nodes} contain the root of $\t$, are closed under 
  the \kl{least common ancestor} operation, and furthermore satisfy that
  for every $x \treelt[\t] z_1 \treelt[\t] y$ such that 
  $\spt(x:z_1) > k$, $\spt(z_1) = k$, and  $\spt(z_1:y) \geq k$ for some $k \in \set{1, \dots, N}$,
  if $x$ and $y$ are \kl{marked nodes} and $z$ is not a \kl{marked node}, then there exists $z_2, z_3$
  such that 
  $x \treelt[\t] z_1 \treelt[\t] z_2 \treeleq[\t] z_3 \treeleq[\t] y$,
  $\spt(z_1) = \spt(z_2) = \spt(z_3) = k$, and 
  $\spt(x:z_1) > k$, $\spt(z_1:z_2) > k$, and $\spt(z_3:y) \geq k$, 
  with $\marking(z_1) = \marked$, and $\marking(z_2) \in \set{\marked, \separating}$.
\end{definition}

\NewDocumentCommand{\gemb}{}{\leq_{\mathrm{gap}}}

\begin{definition}
  \label{def:gap-embedding}
  Given two \kl{marked nested trees}
  $(\t_1, \spt_1, \marking_1)$ and
  $(\t_2, \spt_2, \marking_2)$,
  we say that $(\t_1, \spt_1, \marking_1)$ \intro(marked){gap-embeds} into
  $(\t_2, \spt_2, \marking_2)$, written
  $(\t_1, \spt_1, \marking_1) \gemb (\t_2, \spt_2, \marking_2)$,
  if there exists a mapping $h \colon \t_1 \to \t_2$ that
  \begin{enumerate}
    \item \label{item:gap-embedding:gap} is a \kl{gap embedding} from $(\t_1, \spt_1)$ to $(\t_2, \spt_2)$,

    \item \label{item:gap-embedding:root} sends the root of $\t_1$ to the root of $\t_2$,
  
    \item \label{item:gap-embedding:leaves} maps leaves to leaves,

    \item \label{item:gap-embedding:marking} respects the marking:
      $\marking_1 = \marking_2  \circ h$,

    \item \label{item:gap-embedding:local} respects local products: if $y$ is the immediate left (resp. right)
      child of $x$ in $\t_1$, and $y'$ is the immediate left (resp. right) child of
      $h(x)$ in $\t_2$, then
      $\tlbl{\t_1}{x}{y} = \tlbl{\t_2}{h(x)}{y'}$.

    \item \label{item:gap-embedding:neighbourhood} respects neighbourhood products: for every $k \in \set{1, \dots, N}$,
      and every node $x \in \t_1$, if $z$ is the least ancestor of $x$ such that
      $\spt_1(z) = k$ and $z'$ is the least ancestor of $h(x)$ such that
      $\spt_2(z') = k$, then 
      $\tlbl{\t_1}{z}{x} = \tlbl{\t_2}{z'}{h(x)}$.
    
    \item \label{item:gap-embedding:gluing} is gluing on non-\kl{dummy} nodes: 
    if $\marking_1(y) \in \set{\marked, \separating}$,
    and $x$ is the \kl(tree){parent} of $y$ in $\t_1$,
    then $h(x)$ is the \kl(tree){parent} of $h(y)$ in $\t_2$.

  \end{enumerate}
\end{definition}



\AP
A \kl{$L$-bounded marked nested tree} is a \kl{marked nested tree}
such that any path of non \kl{dummy} nodes has length at most $L$.
As we will see in the following \cref{thm:marked-nested-trees-wqo},
the class of \kl{$L$-bounded well-marked nested trees}
is \kl{well-quasi-ordered} under the \kl{marked gap-embedding} ordering
when labelled over a \kl{well-quasi-ordered} set. Note that the boundedness
assumption is necessary, as shown in \cref{rem:unbounded-mark-not-wqo}.

\begin{theorem}
  \label{thm:marked-nested-trees-wqo}
  For every finite $L$,
  the class of \kl(tree){$L$-bounded} \kl{well-marked nested trees} labelled over a
  \kl{well-quasi-ordered} set $(X, \leq_X)$
  is \kl{well-quasi-ordered} under the \kl{gap-embedding} ordering.
\end{theorem}
\begin{proof}
  One can clearly encode conditions (1) to (6) of \cref{def:gap-embedding}
  by adding suitable labels to the nodes of the trees. Formally,
  we define a new labelling of the nodes of a tree $\t$ as follows: we add to each node $x$ of $\t$
  the label $\tlbl{\t}{z}{x}$ for every $k \in \set{1, \dots, N}$,
  where $z$ is the least ancestor of $x$ such that $\spt(z) = k$,
  or a special symbol $\bot$ if no such ancestor exists.
  Furthermore, if $y$ is the immediate left (resp. right) child of $x$ in $\t$,
  we also add the label $\tlbl{\t}{x}{y}$ to $x$,
  or the special symbol $\bot$ if no such child exists.
  We also distinguish the root by adding a special label $\mathsf{root}$,
  and we add a special label $\mathsf{leaf}$ to every leaf.
  Finally, we also add the label $\marking(x)$ to $x$.
  Since $M$ is finite, the new labelling is still over a \kl{well-quasi-ordered} set.

  The only non-trivial part is to ensure that the mapping $h$ is gluing on non-\kl{dummy} nodes.
  To that end, we will modify the \kl{split} $\spt$ into a new split $\spt'$ as follows:
  for every non-\kl{dummy} node $x$, we have that $x$ is the $i$th element 
  in a maximal path of non-\kl{dummy} nodes $x_1 \treelt[\t] x_2 \treelt[\t] \cdots \treelt[\t] x_n$
  that contains $x$. Because the tree is \kl{$L$-bounded}, we have $n \leq L$.
  We set $\spt'(x) = N + 1 + i$. For a \kl{dummy} node $y$, we set $\spt'(y) = \spt(y)$.

  Note that any \kl{gap-embedding} $h$ from
  $(\t_1, \spt'_1)$ to $(\t_2, \spt'_2)$
  is necessarily gluing on non-\kl{dummy} nodes: 
  assume that $\marking_1(y) \in \set{\marked, \separating}$,
  and let $x$ be the parent of $y$ in $\t_1$.
  Since $\spt'_1(y) = N + 1 + i$, we know that 
  $\spt'_2(h(x):h(y)) \geq \spt'_1(y) = N + 1 + i$.
  But this means that between $h(x)$ and $h(y)$
  there are only nodes with \kl{split depth} greater or equal than $N + 1 + i$.
  If $i = 0$ (i.e., $y$ is the first node of a maximal path of non-\kl{dummy} nodes),
  then there is no such node, and $h(x)$ is the parent of $h(y)$.
  Otherwise, $\spt'_1(x) = N + i > N$, and therefore 
  the path from $h(x)$ to $h(y)$ contains only non \kl{dummy} nodes.
  Since the split values have been chosen to be strictly increasing inside
  maximal paths of non-\kl{dummy} nodes,
  there is no such node, and $h(x)$ is the parent of $h(y)$.
\end{proof}

\begin{lemma}
  \label{rem:unbounded-mark-not-wqo}
  The \kl{well-quasi-order} of \cref{thm:marked-nested-trees-wqo}
  does not hold if we remove the \kl{$L$-boundedness} assumption.
  Indeed, consider the infinite sequence of trees
  $(\t_n, \spt_n, \marking_n)$ where $\t_n$ is a branch of length $n+1$,
  $\spt_n$ assigns the value $1$ to every node of $\t_n$,
  and $\marking_n$ marks all the nodes of $\t_n$.
  It is straightforward to check that for every $n < m$,
  $(\t_n, \spt_n, \marking_n)$ does not \kl{gap-embed} into
  $(\t_m, \spt_m, \marking_m)$.
\end{lemma}
\begin{proof}
  We create an infinite antichain by 
  considering the infinite sequence of trees
  $(\t_n, \spt_n, \marking_n)$ where $\t_n$ is a branch of length $n+1$,
  $\spt_n$ assigns the value $1$ to every node of $\t_n$,
  and $\marking_n$ marks all the nodes of $\t_n$.
  It is straightforward to check that for every $n < m$,
  $(\t_n, \spt_n, \marking_n)$ does not \kl{gap-embed} into
  $(\t_m, \spt_m, \marking_m)$.
\end{proof}

\begin{lemma}
  \label{lem:gap-embedding-respects-products}
  Let $(\t_1, \spt_1, \marking_1)$ and
  $(\t_2, \spt_2, \marking_2)$ be two \kl{marked nested trees}
  such that
  $(\t_1, \spt_1, \marking_1) \gemb (\t_2, \spt_2, \marking_2)$ 
  via some mapping $h$.
  For every pair of \kl{marked nodes} $x \treelt[\t_1] y$ in $\t_1$,
  we have $\tlbl{\t_1}{x}{y} = \tlbl{\t_2}{h(x)}{h(y)}$.
\end{lemma}
\begin{proof}
  We proceed by induction on the value of $\spt_1(x:y)$.
  If $\spt(x:y) = N+1$, then $y$ is an immediate child of $x$,
  and $\spt(h(x):h(y)) = N+1$ by definition of \kl{gap-embedding}.
  Therefore, by \cref{item:gap-embedding:local},
  we have $\tlbl{\t_1}{x}{y} = \tlbl{\t_2}{h(x)}{h(y)}$.

  Otherwise, let $k = \spt_1(x:y) \leq N$. We distinguish two cases.
  Either, there exists a unique child $z$ of $x$ such that
  $\spt_1(x:z) = k$ and $z$ is a \kl{marked node}.
  By induction hypothesis, we have
  $\tlbl{\t_1}{z}{y} = \tlbl{\t_2}{h(z)}{h(y)}$,
  and $\tlbl{\t_1}{x}{z} = \tlbl{\t_2}{h(x)}{h(z)}$. Hence, 
  \begin{align*}
    \tlbl{\t_1}{x}{y} &= \tlbl{\t_1}{x}{z} \cdot \tlbl{\t_1}{z}{y} \\
                      &= \tlbl{\t_2}{h(x)}{h(z)} \cdot \tlbl{\t_2}{h(z)}{h(y)} \\
                      &= \tlbl{\t_2}{h(x)}{h(y)} \quad .
  \end{align*}

  Otherwise, since $(\t_1, \spt_1, \marking_1)$ is \kl{well-marked},
  there exists $z_1, z_2, z_3$ such that
  $x \treelt[\t_1] z_1 \treelt[\t_1] z_2 \treeleq[\t_1] z_3 \treelt[\t_1] y$,
  $\spt_1(z_1) = \spt_1(z_2) = \spt_1(z_3) = k$, and 
  $\spt_1(x:z_1) > k$, $\spt_1(z_1:z_2) > k$, and $\spt_1(z_3:y) > k$, 
  with $\marking_1(z_1) = \marked$, and $\marking_1(z_2) \in \set{\marked, \separating}$.

  \begin{center}
    \begin{tikzpicture}
        \node (x) at (0,0) {$x$};
        \node (z1) at (2,0) {$z_1$};
        \node (z2) at (4,0) {$z_2$};
        \node (z3) at (6,0) {$z_3$};
        \node (y) at (8,0) {$y$};

        \node at (1,-0.2) {$> k$};
        \node at (3,-0.2) {$> k$};
        \node at (5,-0.2) {$\geq k$};
        \node at (7,-0.2) {$> k$};

        \node[below=0.02cm of z1] {$k$};
        \node[below=0.02cm of z2] {$k$};
        \node[below=0.02cm of z3] {$k$};

        \draw[->] (x) --
        node[midway, above] {$\tlbl{\t_1}{x}{z_1}$}
        (z1);
        \draw[->] (z1) --
        node[midway, above] {$\tlbl{\t_1}{z_1}{z_2}$}
        (z2);
        \draw[->] (z2) --
        node[midway, above] {$\tlbl{\t_1}{z_2}{z_3}$}
        (z3);
        \draw[->] (z3) --
        node[midway, above] {$\tlbl{\t_1}{z_3}{y}$}
        (y);
    \end{tikzpicture}
  \end{center}

  By applying the \kl{gap-embedding} $h$,
  we the following configuration inside $\t_2$, where
  we used the notations
  $a = \tlbl{\t_2}{h(x)}{h(z_1)}$,
  $b = \tlbl{\t_2}{h(z_1)}{h(z_2)}$,
  $c = \tlbl{\t_2}{h(z_2)}{h(z_3)}$,
  and $d = \tlbl{\t_2}{h(z_3)}{h(y)}$.

  \begin{center}
    \begin{tikzpicture}[
      xscale=0.9,
      % style for `dummy` nodes
      dummy/.style={B1},
      marked/.style={A1},
      separating/.style={A2},
      value/.style={A1},
    ]
        \node[marked] (hx) at (0,0) {$h(x)$};
        \node[marked] (hz1) at (2,0) {$h(z_1)$};
        \node (hz2) at (4,0) {$h(z_2)$};
        \node (hz3) at (6,0) {$h(z_3)$};
        \node[marked] (hy) at (8,0) {$h(y)$};

        \node at (1,-0.2) {$> k$};
        \node at (3,-0.2) {$> k$};
        \node at (5,-0.2) {$\geq k$};
        \node at (7,-0.2) {$> k$}; 
        \node[below=0.02cm of hz1] {$k$};
        \node[below=0.02cm of hz2] {$k$};
        \node[below=0.02cm of hz3] {$k$};

        \draw[->] (hx) --
        node[midway, above, value] {$a$}
        (hz1);
        \draw[->] (hz1) --
        node[midway, above, value] {$b$}
        (hz2);
        \draw[->] (hz2) --
        node[midway, above, value] {$c$}
        (hz3);
        \draw[->] (hz3) --
        node[midway, above, value] {$d$}
        (hy);
    \end{tikzpicture}
  \end{center}

  Let us briefly argue the inequalities on the splits 
  on the above paths.
  To do that, we will use the fact that \cref{item:gap-embedding:gap}
  implies that if $\spt(u:v) > k$ and $\spt(v) = k$,
  then $\spt(h(u):h(v)) \geq k$.
  This is proven by immediate induction on the length
  of the path from $u$ to $v$, the base case being given by 
  the definition of \kl{gap-embedding}.
  Furthermore, if $v$ is a \kl{marked node} or a \kl{separating node},
  then we can strengthen the inequality to a strict one,
  i.e., $\spt(h(u):h(v)) > k$ even when $\spt(v) \neq k$.
  This is because of \cref{item:gap-embedding:gluing},
  which ensures that the immediate parent of $h(v)$
  is the image of the immediate parent of $v$, which has a split value
  strictly greater than $k$, and we conclude by the previous argument.\footnote{The case where there are no intermediate nodes between $u$ and $v$ happens if $v$ is the immediate child of $u$, 
  in which case the property is given directly by the 
  gluing property.}

  Using these facts we conclude that:
  \begin{align*}
    \spt_2(h(x):h(z_1))   &> k 
                          & \text{because $z_1$ is a \kl(nodes){marked},}
                          \\
    \spt_2(h(z_1):h(z_2)) &> k
                          & \text{because $z_2$ is a \kl(nodes){marked}
                                   or \kl(nodes){separating},}
                          \\
    \spt_2(h(z_2):h(z_3)) &\geq k
                          & \text{by the gap property,}
                          \\
    \spt_2(h(z_3):h(y))   &> k
                          & \text{because $y$ is a \kl(nodes){marked},}                               
  \end{align*}

  Now, we can compute the values of $a$, $b$, and $d$:
  \begin{align*}
    \tlbl{\t_1}{x}{z_1} & = a
    &\text{by induction hypothesis,} \\
    \tlbl{\t_1}{z_2}{z_3} & = b
    &\text{by \cref{item:gap-embedding:neighbourhood},} \\
    \tlbl{\t_1}{z_3}{y} & = d 
    &\text{by \cref{item:gap-embedding:neighbourhood}.}
  \end{align*}

  Finally, because $h(x)$ and $h(y)$ are \kl{independent at level $k$},
  and so are $x$ and $y$,
  we have by \cref{fake-idempotent:eq} that
  \begin{align*}
    \tlbl{\t_1}{x}{y}
          & = \tlbl{\t_1}{x}{z_1} \cdot
              \tlbl{\t_1}{z_1}{z_2} \cdot
              \tlbl{\t_1}{z_3}{y} \\
          & = a \cdot b \cdot d \\
    & = \tlbl{\t_2}{h(x)}{h(y)} \quad . \qedhere
  \end{align*}
\end{proof}


\begin{corollary}
  \label{cor:gap-embedding-monotone}
  Let $(\t_1, \spt_1, \marking_1)$ and
  $(\t_2, \spt_2, \marking_2)$ be two \kl{well-marked nested trees}
  such that
  $(\t_1, \spt_1, \marking_1) \gemb (\t_2, \spt_2, \marking_2)$.
  Then, the graph interpreted from
  $(\t_1, \spt_1, \marking_1)$
  is an induced subgraph of the graph interpreted from
  $(\t_2, \spt_2, \marking_2)$
  when considering only \kl{marked leaves} as vertices.
\end{corollary}
\begin{proof}
  Follows directly from \cref{lem:gap-embedding-respects-products}
  and the definition of the interpretation using the set $P \subseteq M^3$,
  because the edges between two \kl{marked leaves} $x$ and $y$
  only depend on the values of
  $\tlbl{\t}{\treeRoot}{\lca(x,y)}$,
  $\tlbl{\t}{\lca(x,y)}{x}$, and
  $\tlbl{\t}{\lca(x,y)}{y}$, all of which are \kl{marked nodes}.
\end{proof}
