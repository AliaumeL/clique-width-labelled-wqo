\section{Existentially Interpreting Finite Paths}
\label{sec:interpreting-paths}

TODO:
\begin{itemize}
    \item If not WQO, then arbitrarily large bad branches.
    \item Color the nodes of this bad branch using their type (relative to the branch)
    \item Without loss of generality, assume that the branch is super long, and that
        we do not have a lot of different types of nodes inside each bag.
        (uniform bound $K$)
    \item Take a large bad branch, and define 
        $\psi(x,y) \defined E(x,y) \neq E(l(x), r(y))$ where
        $l(x)$ and $r(y)$ respectively "putting to the left" or 
        "putting to the right" of the branch in a (non-existing) "good decomposition".
    \item Witness that $\psi$ defines \emph{almost} a directed path, except
        that it may have back edges due to the non-commutativity of the idempotent node.
    \item If there are no back edges, we just defined a long induced path.
    \item If there were some back edges, there were \emph{a lot of them}.
    \item Define $\phi(x,y)$ if there exists a directed edge $x \to y$
        \textbf{AND} a node $z$ such that $z \to z_1$, $z \to z_2$
        and a path of length at most $2K$ from $z_1$  to $z_2$
        that goes through $x$ and $y$ (in this order).
    \item Prove that $\phi$ defines a directed path.
    \item Take the symmetric version of $\phi$ and conclude.
\end{itemize}
 Theory

Consider a bad branch $B$. We define the following interpretation:
\begin{equation*}
    \varphi(x,y) \defined 
    E(x,y) \neq E(l(x), r(y))
\end{equation*}

\restate{transductions-paths:thm}
