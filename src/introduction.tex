\section{Introduction}
\label{sec:introduction}

\todo[inline]{Define WQO here}
\subparagraph{Context.} The theory of well-quasi-orderings (WQOs) is a powerful
combinatorial setting that has found applications in various areas of
mathematics and computer science. In graph theory, the celebrated result of
Robertson and Seymour~\cite{ROBSEY04} states that the class of all finite
graphs is well-quasi-ordered under the minor relation, a profound result with
deep algorithmic consequences. WQOs are also at the heart of
\emph{well-structured transition systems}, infinite state transitions systems
over which verification algorithms can be designed~\cite{ABDU96,ABDU98}. One of
the appeal of WQOs is that they are closed under various operations, such as
the sum and the product of WQOs. As an example, the closure of WQOs under the
finite words, also known as Higman's lemma \cite{HIG52}, has been used in the
verification of so-called \emph{lossy channel systems}~\cite{ABDU93}. 

This explains the \emph{bottom-up} research direction in the field of WQOs,
which is to understand which \emph{constructors} preserve the WQO property:
finite sums, finite products, finite words, finite trees, finite graphs
\emph{with the minor relation} etc. It is motivated by the idea that one can
build complex orderings to model a concrete system by combining simpler ones,
and has empirically shown to be a fruitful approach (see \cite{HSS13} for an
example of nesting Higman's orderings). One other research direction is to
devise \emph{decision procedures} that take as input a set and decide whether
it is WQO or not, and whether classical decision algorithm on well-structured
transition systems can be applied to the concrete model one has. This dual
\emph{top-down} approach also had its recent share of successes
\cite{FINGU19,LOPEZ24}.


\subparagraph{A Model-Theoretic Approach to WQOs.}
\AP 
While finite products and sums are natural algebraic constructions on sets, the
cornerstone results of Higman and Kruskal showing that well-quasi-ordered sets
are closed under \emph{finite words} \cite{HIG52} and \emph{finite trees}
\cite{KRU72} require a careful definition of the ordering on those structures,
together with a non-trivial and tailored proof that they are
well-quasi-ordered, often requiring the use of a so-called \emph{minimal bad
sequence argument} \cite{NASH65}. We argue that the better way to introduce
these orderings stems from a model-theoretic approach to WQOs, where one
considers subsets of \emph{finite structures} endowed with the usual notion of
\emph{embedding} (of models). Considering finite words as labelled total orders
one immediately sees that the seemingly \emph{ad-hoc} definition of the
\emph{subword ordering} powering Higman's lemma is in fact an instance of the
usual embedding relation from logic. A similar observation can be made for
Kruskal's tree theorem, where structures are labelled trees equipped with the
least common ancestor relation. These are folklore results, but we believe that
they are a strong motivation for a more systematic study of the embedding
relation on classes of finite structures.

\AP In this context, the simplest (non-trivial) structures one can consider are
\emph{finite undirected graphs}. The corresponding notion of embedding is the
\emph{induced subgraph relation} that is well-known from graph theory. Given a
class $\Cls$ of finite undirected graphs, one can as whether $\Cls$ is
well-quasi-ordered under the induced subgraph relation. One can also consider
$\Cls$ as a \emph{constructor} for new WQOs, mapping $(X, \leq)$ to
$\Label{X}{\Cls}$, the class of graphs in $\Cls$ labelled using elements of
$X$, this is how one obtains $X^*$ from $X$ in Higman's lemma. The natural
question is then, whether $\Label{X}{\Cls}$ is WQO when $X$ itself is WQO.
Freely labeling classes of graph is an operation that is prominent in the field
of \emph{structural graph theory}: this is for instance the difference between
being NIP and being \emph{monadically} NIP \todo{cite}. However, in this
context,  the set $X$ of labels is equipped with the equality relation and is
finite. Of course, one can also consider further restrictions on $X$, such as
requiring that $X$ is of size at most $k$ for some $k \in \Nat$. We  depicted
in
\cref{graph-properties:fig},
these various properties, and the two conjectures of Pouzet claiming that these
potentially distinct notions all collapse to either being \kl{WQO} or
\kl{wqo-WQO}. We state the two conjectures separately because only the first
one is explicitly found in \cite{POUZ72}.

\begin{conjecture}[Pouzet \cite{POUZ72}]
    \label{pouzet1:conj}
    Let $\Cls$ be a class of finite graphs.
    Then, $\Cls$ is 2-well-quasi-ordered
    if and only if
    $\Cls$ is labelled-well-quasi-ordered.
\end{conjecture}

\begin{conjecture}
    \label{pouzet2:conj}
    Let $\Cls$ be a class of finite graphs.
    Then, $\Cls$ is labelled-well-quasi-ordered
    if and only if
    $\Cls$ is wqo-well-quasi-ordered.
\end{conjecture}



\begin{figure}
    \centering
    \begin{tikzpicture}[
        conj/.style={rounded corners, thick, dashed, 
            A4
        },
        impl/.style={->, double},
        constr/.style={rounded corners,minimum width=1.7cm,minimum height=1cm,draw,rectangle,align=center}]
        \node[constr] (1w) at (0,0) {One \\ label};
        \node[constr] (2w) at (2,0) {Two \\ labels};
        \node[constr] (kw) at (4,0) {$k < \infty$ \\ labels};
        \node[constr] (lw) at (6,0) {Finite set \\ of labels};
        \node[constr] (ww) at (8,0) {WQO set    \\ of labels};

        \draw[conj]
        ($(lw.south west) + (-0.1, -1.1)$) rectangle ($(ww.north east) + (0.1, 0.1)$);

        \draw[conj]
        ($(2w.north west) + (-0.1, 1.1)$) rectangle ($(lw.south east) + (0.1, -0.1)$);
        \coordinate (P1m) at ($(2w.north)!0.5!(lw.north)$);
        \coordinate (P2m) at ($(lw.south)!0.5!(ww.south)$);
        \node (P1) at ($(P1m) + (0, 0.7)$) {\cref{pouzet1:conj}};
        \node (P2) at ($(P2m) + (0, -0.7)$) {\cref{pouzet2:conj}};

    \end{tikzpicture}
    \caption{The four well-quasi-ordering properties one can consider
             on classes of finite graphs.}
    \label{graph-properties:fig}
\end{figure}

In general, the class of all finite graphs (even undirected and unlabelled) is
not well-quasi-ordered under the induced subgraph relation (for instance, the
class of finite cycles forms an infinite antichain).

There have been many attempts to understand
which classes of (finite) graphs are WQOs when endowed with the induced
subgraph relation. One can think of the early work of Ding on classes of
bounded tree-depth \cite{DING92}, or more recent approaches
\cite{DRT10,DLP17,POZA22}.
Typical proofs that a set is well-quasi-ordered can be clustered into two
categories: \emph{structural proofs} that exhibit an encoding of the set into a
well-quasi-ordered set built from simpler elements, and \emph{minimal bad
sequence arguments} \cite{NASH65}. The latter are often tedious and error-prone
to carry out \cite[Discussion xxx]{LOPEZ23}. 

Attempts to prove these conjectures often relied on assuming some kind of
structural property of the class of graphs, such as enjoying suitable
\emph{tree decompositions}.

Now, to work on those trees, one needs to place a well-quasi-ordered relation
on them. This is usually done using the tree embedding relation, also known as
Kruskal's embedding relation, which is a well-quasi-ordering on trees
\cite{KRU72}. This works when the function that maps a tree to a graph is
simple enough, but more often than not, this is not the case. In \cite{DRT10},
the authors introduce a \emph{tailored} embedding relation on trees, and prove
that it is a well-quasi-ordering under certain hypotheses, leveraging a tedious
and error-prone proof using a \emph{minimal bad sequence argument}
\cite{NASH65}. \todo{talk about clique width now}

We know since \cite{FREU20} and \cite{LOPEZ23} that one can easily inductively
define well-quasi-orderings of increasing complexity on a given set. One
example of such a construction is the \emph{gap embedding relation} of
Derhowitz and Tzameret \cite{DERSHOWITZ200380}. This relation was already at
the core of \emph{priority channel systems}, although in a somewhat hidden form
\cite{HSS13}.  It was also noticed in \cite{LOPEZ24} that the \emph{ad-hoc}
construction of \cite{DRT10} was in fact a particular instance of the gap
embedding relation on trees.

A line of research initiated in \cite{LOPEZ24} explores the \emph{automata
theoretic} approach to the Pouzet conjectures. It was conjectured that the
notion of gap-embedding could be useful. We prove that this is the case.
\todo{Relied on monoids and automata theory on words}
\textbf{The gap embedding is essentially the only well-quasi-ordering on 
graph that can exist.}


Recently, a conjecture was formulated by Sylvain Schmitz in a personal
communication with the authors, which we will call the \emph{transduction
conjecture}. It attemps to bridge the gap between the study of
\kl{well-quasi-ordered} classes of graphs that are ``combinatorially
well-behaved'' and the field of ``structural graph theory'' that focuses on
class that are ``first-order well-behaved''.


\begin{conjecture}[Schmitz's Transduction Conjecture]
    \label{transduction:conj}
    Let $\Cls$ be a class of finite graphs.
    Then, $\Cls$ is labelled-well-quasi-ordered
    if and only if
    $\Cls$
    does not existentially transduce all finite paths.
\end{conjecture}


\paragraph*{Contributions.} At the core of our analysis is the fundamental
connection between a deep automata theoretic construction (the existence of
\emph{forward ramseyan factorisations} on trees \cite{COLC07}) and a deep
combinatorial construction (\emph{gap embedding} on trees
\cite{DERSHOWITZ200380}). These two constructions were meant to be used
together, as we will try to advocate in this paper.

We are also interested in the \emph{decidability} of the WQO property. This is
a question that has recently gained traction in the community, and is of
practical interest.

\begin{theorem}[restate=pouzet2:thm]
    \label{pouzet-2:thm}
    Let $\Cls$ be a class of finite graphs of bounded clique-width.
    Then, $\Cls$ is labelled-well-quasi-ordered
    if and only if 
    $\Cls$ is wqo-well-quasi-ordered.
\end{theorem}

An interpretation is monotone if it is monotone with respect to the
embedding relation.

\begin{theorem}[restate=transductions-paths:thm,label={transductions-paths:thm}]
    \label{transductions-paths:thm}
    Let $\Cls$ be a class of \kl{bounded clique-width}.
    Then $\Cls$ is \kl{labelled-well-quasi-ordered}
    if and only if
    it does not \kl{monotonely transduce}
    all \kl{finite paths}.
\end{theorem}


\begin{conjecture}[Szymon's Collapsing Conjecture]
    \label{nip-cw:conj}
    Let $\Cls$ be a class of graphs.
    If $\Cls$ is labelled-wqo,
    then $\Cls$ is of bounded clique-width.
\end{conjecture}

The motto would then be ``the only well-quasi-orderings on graphs are the gap
embeddings'' and would provide a fundamental barrier to the complexity of
induced subgraph well-quasi-orderings.

\paragraph*{Outline.}
