\section{Introduction}
\label{sec:introduction}

\AP A quasi-ordered set $(X, \leq)$ is \intro{well-quasi-ordered} (WQO) if
every non-empty subset of $X$ has a finite non-empty subset $Y \subfin X$ of
minimal elements, i.e. such that for every $x \in X$, there exists $y \in Y$
such that $y \leq x$. When the set $X$ is totally ordered, this is equivalent
to the usual notion of well-ordering, but in the presence of incomparable
elements, it is a stronger condition. The theory of well-quasi-orderings (WQOs)
is a powerful combinatorial setting that has found applications in various
areas of mathematics and computer science. In graph theory, the celebrated
result of Robertson and Seymour~\cite{ROBSEY04} states that the class of all
finite graphs is well-quasi-ordered under the minor relation, a profound result
with deep algorithmic consequences. WQOs are also at the heart of
\emph{well-structured transition systems}, infinite state transitions systems
over which verification algorithms can be designed~\cite{ABDU96,ABDU98}. One of
the appeal of WQOs is that they are closed under various operations, such as
the sum and the product of WQOs. As an example, the closure of WQOs under the
finite words, also known as Higman's lemma \cite{HIG52}, has been used in the
verification of so-called \emph{lossy channel systems}~\cite{ABDU93}. 


% Well-quasi-ordered classes of graphs
\AP Undirected finite graphs are naturally equipped with the \reintro{induced
subgraph relation}, where $G$ is an induced subgraph of $H$ if $G$ can be
obtained from $H$ by deleting vertices (and the edges adjacent to them). Unlike
the graph minor relation, the induced subgraph relation is not a
well-quasi-ordering on the class of all finite graphs, as witnessed by the
infinite family of cycles of increasing size, which form an infinite antichain.
However, some classes of finite graphs are well-quasi-ordered under the induced
subgraph relation: for instance, the class of all finite paths, or the class of
all finite cliques. Distinguishing which classes of finite graphs are
well-quasi-ordered under the induced subgraph relation is a long-standing open
problem in graph theory, dating back to Pouzet's seminal work \cite{POUZ72}.

\AP In general, there is little hope to characterise all such classes, since
any countable order with finite infixes can be embedded into the induced
subgraph relation on finite graphs (this is a consequence of \cite[Lemma
5.1]{Kuske06}). However, there are high hopes that a better notion of
well-quasi-ordering on classes of finite graphs is more amenable to structural
characterisations. This is why one usually considers \emph{labelled}
well-quasi-orderings on classes of finite graphs. Given a class $\Cls$ of
finite graphs, and a well-quasi-ordering $(X, \leq)$, one can consider the
class $\Label{X}{\Cls}$ of graphs in $\Cls$ whose vertices are labelled by
elements of $X$. The induced subgraph relation is then extended to labelled
graphs by requiring that the labels are preserved by the embedding, i.e. that
the label of a vertex in the smaller graph is less than or equal to the label
of its image in the larger graph. 

\begin{example}
  \label{ex:labelled-graphs}
  The class of all finite paths is not $2$-well-quasi-ordered,
  since the infinite sequence of paths with colored endpoints 
  forms an infinite antichain.
  The class of cliques is labelled-well-quasi-ordered,
  since any labelled clique can be identified with a multiset of labels,
  and the set of finite multisets over a WQO is itself WQO.
\end{example}


\AP
Bounded clique-width is a well-studied graph complexity measure
that generalises other well-known graph complexity measures such as
todo. There are several equivalent definitions of 
having \emph{bounded clique-width}, and we will use two 
of them in this paper. On the one hand, a class of graphs $\Cls$ has
\emph{bounded clique-width} if it is MSO-transducable from a class of finite
trees. On the other hand, a class of graphs $\Cls$ has 
\emph{bounded clique-width} if there exists a finite set of 
labels $\Sigma$ such that 
every graph in $\Cls$ can be constructed using the following operations:
\begin{itemize}
  \item creating a new vertex with label $a \in \Sigma$,
  \item taking the disjoint union of two labelled graphs,
  \item adding edges between all vertices of label $a$ and all vertices of label $b$,
  \item renaming all labels $a$ into $b$.
\end{itemize}

\begin{example}
  The class of all finite cliques, finite paths, finite cycles, 
  and finite trees all have bounded clique-width.
\end{example}

\begin{example}
  The class of all finite grids does not have bounded clique-width.
\end{example}



\paragraph*{Contributions}
\AP
We prove that one can decide whether a class of graphs of bounded clique-width
is labelled-well-quasi-ordered, when the class is given by
an $\MSO$ interpretation from finite trees to undirected graphs.
Our proof scheme 




\paragraph*{Related works}

There are three natural conjectures that have been formulated 
regarding the relationship between various notions of well-quasi-ordering on
classes of finite graphs.

\begin{conjecture}[Pouzet \cite{POUZ72}]
    \label{pouzet:conj}
    Let $\Cls$ be a class of finite graphs.
    Then, the following are equivalent:
    \begin{enumerate}
      \item $\Cls$ is pointed-well-quasi-ordered,
      \item $\Cls$ is 2-well-quasi-ordered,
      \item $\Cls$ is labelled-well-quasi-ordered,
      \item $\Cls$ is wqo-well-quasi-ordered.
      \item $\Cls$ one can add a total order on the vertices of 
        its graphs, and remain wqo-well-quasi-ordered.
    \end{enumerate}
\end{conjecture}

\begin{conjecture}[{\cite[Conjecture 4]{DRT10}}]
  \label{thomasse:conj}
  Let $\Cls$ be a hereditary class of finite graphs of bounded clique-width.
  Then, $\Cls$ is $2$-well-quasi-ordered if and only if
  there exists $\Cls \subseteq \Cls[D]$ 
  ``structurally simpler'' such that $\Cls[D]$ is $2$-well-quasi-ordered.
\end{conjecture}


\begin{conjecture}[\cite{ALM17}]
  \label{lozin:conj}
  Let $\Cls$ be a hereditary class of finite graphs that is not labelled-well-quasi-ordered.
  Then, there exists 
  an infinite bad sequence of graphs in $\Cls$
  that is \emph{regular}.
\end{conjecture}

\begin{conjecture}[Schmitz's Transduction Conjecture]
    \label{transduction:conj}
    Let $\Cls$ be a class of finite graphs.
    Then, $\Cls$ is labelled-well-quasi-ordered
    if and only if
    $\Cls$
    does not existentially transduce all finite paths.
\end{conjecture}


\begin{conjecture}[Szymon's Collapsing Conjecture]
    \label{nip-cw:conj}
    Let $\Cls$ be a class of graphs.
    If $\Cls$ is labelled-wqo,
    then $\Cls$ is of bounded clique-width.
\end{conjecture}

\paragraph*{Proof Techniques and Overview.} The overall proof technique can be
summarised in \cref{fig:proof-technique}. Given a class $\Cls$ of graphs
defined as the image of an $\MSO$ interpretation $I$ from finite trees to
graphs, we want to decide whether $\Cls$ is labelled-well-quasi-ordered. To
this end, we will use the following folklore result: if $(X, \leq)$ is a WQO,
and $f \colon X \to Y$ is a surjective order-preserving map, then $(Y, \leq)$
is also WQO. Because there exists a good candidate for $X$, namely, the set of
trees with Kruskal's tree embedding relation, we will try to see whether $I$ is
order-preserving. It turns out that for general interpretations, this is not
true (see \cref{example:msotree-not-monotone}). However, leveraging tools from
automata theory in the name of \emph{ramseyan factorisations} \cite{COLC07}, we
will be able to ``massage'' the input trees using a function $F$ into
\emph{nested trees}, with the property that for a good notion of ordering on
nested trees (nested Kruskal embedding essentially), the function $J$ that
factors $I$ through $F$ is order-preserving if and only if $\Cls$ is
labelled-well-quasi-ordered.

This proof technique will actually allow us to derive stronger results since it
shows that embeddings on graphs in $\Cls$ can be assumed to respect a certain
tree-like structure, and a fortiori, a linear order on the vertices of the
graphs. This will allow us to show that \cref{pouzet:conj} holds for classes of
graphs of bounded clique-width. Furthermore, listing the obstructions for $J$
to be order-preserving will allow us to show that counter-examples are regular
in nature, yielding a proof of \cref{lozin:conj} for classes of graphs of
bounded clique-width. Finally, the tree-like structure obtained is a form of
simple tree decomposition, answering positively to \cref{thomasse:conj} in
totality.


\paragraph*{Outline}



This explains the \emph{bottom-up} research direction in the field of WQOs,
which is to understand which \emph{constructors} preserve the WQO property:
finite sums, finite products, finite words, finite trees, finite graphs
\emph{with the minor relation} etc. It is motivated by the idea that one can
build complex orderings to model a concrete system by combining simpler ones,
and has empirically shown to be a fruitful approach (see \cite{HSS13} for an
example of nesting Higman's orderings). One other research direction is to
devise \emph{decision procedures} that take as input a set and decide whether
it is WQO or not, and whether classical decision algorithm on well-structured
transition systems can be applied to the concrete model one has. This dual
\emph{top-down} approach also had its recent share of successes
\cite{ALM17,FINGU19,LOPEZ24}.



\clearpage
