\section{Introduction}
\label{sec:introduction}

\subparagraph{Context.} The theory of well-quasi-orderings (WQOs) is a powerful
combinatorial setting that has found applications in various areas of
mathematics and computer science. In graph theory, the celebrated result of
Robertson and Seymour~\cite{ROBSEY04} states that the class of all finite
graphs is well-quasi-ordered under the minor relation, a profound result with
deep algorithmic consequences. WQOs are also at the heart of
\emph{well-structured transition systems}, infinite state transitions systems
over which verification algorithms can be designed~\cite{ABDU96,ABDU98}. One of
the appeal of WQOs is that they are closed under various operations, such as
the sum and the product of WQOs. As an example, the closure of WQOs under the
finite words, also known as Higman's lemma \cite{HIG52}, has been used in the
verification of so-called \emph{lossy channel systems}~\cite{ABDU93}. 

This explains one research direction in the field of WQOs, which is to
understand which \emph{constructors} preserve the WQO property: finite sums,
finite products, finite words, finite trees, finite graphs \emph{with the minor
relation} etc. It is motivated by the idea that one can build complex orderings
to model a concrete system by combining simpler ones, and has empirically shown
to be a fruitful approach (see \cite{HSS13} for an example of nesting Higman's
orderings). One other research direction is to devise \emph{decision
procedures} that take as input a set and decide whether it is WQO or not, and
whether classical decision algorithm on well-structured transition systems can
be applied to the concrete model one has. This dual approach that is more
\emph{top-down} also had its recent share of successes \cite{FINGU19,LOPEZ24}.

One particular constructor has remained quite elusive in the context of WQOs:
the (labelled) finite graphs equipped with the induced subgraph relation. Let
us first explain why it is believed to be of particular interest. The seminal
results of Higman and Kruskal show that finite words and finite trees are
well-quasi-ordered under the subword and subtree relations, respectively
\cite{HIG52,KRU72}. Considering finite words and finite trees as labelled
orderings (i.e., labelled \emph{directed graphs}), one immediately sees that
the seemingly \emph{ad-hoc} constructions of Higman and Kruskal are in fact
instances of the induced subgraph relation on labelled graphs. However, their
proof rely on the specific structure of trees and words, using so-called
\emph{minimal bad sequence arguments} \cite{NASH65}. In general, the class of
all finite graphs (even undirected and unlabelled) is not well-quasi-ordered
under the induced subgraph relation (for instance, the class of finite cycles
forms an infinite antichain).


There have been many attempts to understand which classes of (finite) graphs
are WQOs when endowed with the induced subgraph relation. One can think of the
work of Ding on classes of bounded tree-depth \cite{DING92}, or more recent
approaches \cite{DLP17,POZA22}. In general, one studies the simpler class of
\emph{undirected} finite graphs, where a lot of the complexity of the induced
subgraph relation already lies. In this context, there are \emph{four}
properties of interest: being \kl{well-quasi-orderded} (WQO), being
\kl{$k$-well-quasi-ordered} ($k$-WQO) for some finite $k$, being
\kl{labelled-well-quasi-ordered} (labelled-WQO), and being
\kl{wqo-well-quasi-ordered} (wqo-WQO). All of those properties stem from
different motivations. Being \kl{WQO} is a fundamental property but in concrete
applications, one often needs to add some colouring to the graphs (for instance
in the context of structural graph theory). But being \kl{labelled-WQO} is not
strong enough if one wants to consider a class of graphs as an \emph{operator}
to construct new WQOs.

Two conjectures of Pouzet claim that these potentially distinct notions all
collapse to either being \kl{WQO} or \kl{wqo-WQO}. We state them separately
because only the first one is explicitly found in \cite{POUZ72}.

\begin{conjecture}
    \label{pouzet1:conj}
    Let $\Cls$ be a class of finite graphs.
    Then, $\Cls$ is 2-well-quasi-ordered
    if and only if
    $\Cls$ is labelled-well-quasi-ordered.
\end{conjecture}

\begin{conjecture}
    \label{pouzet2:conj}
    Let $\Cls$ be a class of finite graphs.
    Then, $\Cls$ is labelled-well-quasi-ordered
    if and only if
    $\Cls$ is wqo-well-quasi-ordered.
\end{conjecture}

Typical proofs that a set is well-quasi-ordered can be clustered into two
categories: \emph{structural proofs} that exhibit an encoding of the set into a
well-quasi-ordered set built from simpler elements, and \emph{minimal bad
sequence arguments} \cite{NASH65}. The latter are often tedious and error-prone
to carry out \cite[Discussion xxx]{LOPEZ23}. 

Attempts to prove these conjectures often relied on assuming some kind of
structural property of the class of graphs, such as enjoying suitable
\emph{tree decompositions}.

Now, to work on those trees, one needs to place a well-quasi-ordered relation
on them. This is usually done using the tree embedding relation, also known as
Kruskal's embedding relation, which is a well-quasi-ordering on trees
\cite{KRU72}. This works when the function that maps a tree to a graph is
simple enough, but more often than not, this is not the case. In \cite{DRT10},
the authors introduce a \emph{tailored} embedding relation on trees, and prove
that it is a well-quasi-ordering under certain hypotheses, leveraging a tedious
and error-prone proof using a \emph{minimal bad sequence argument}
\cite{NASH65}. \todo{talk about clique width now}

We know since \cite{FREU20} and \cite{LOPEZ23} that one can easily inductively
define well-quasi-orderings of increasing complexity on a given set. One
example of such a construction is the \emph{gap embedding relation} of
Derhowitz and Tzameret \cite{DERSHOWITZ200380}. This relation was already at
the core of \emph{priority channel systems}, although in a somewhat hidden form
\cite{HSS13}.  It was also noticed in \cite{LOPEZ24} that the \emph{ad-hoc}
construction of \cite{DRT10} was in fact a particular instance of the gap
embedding relation on trees.

A line of research initiated in \cite{LOPEZ24} explores the \emph{automata
theoretic} approach to the Pouzet conjectures. It was conjectured that the
notion of gap-embedding could be useful. We prove that this is the case.
\todo{Relied on monoids and automata theory on words}
\textbf{The gap embedding is essentially the only well-quasi-ordering on 
graph that can exist.}


Recently, a conjecture was formulated by Sylvain Schmitz in a personal
communication with the authors, which we will call the \emph{transduction
conjecture}. It attemps to bridge the gap between the study of
\kl{well-quasi-ordered} classes of graphs that are ``combinatorially
well-behaved'' and the field of ``structural graph theory'' that focuses on
class that are ``first-order well-behaved''.


\begin{conjecture}[Schmitz's Transduction Conjecture]
    \label{transduction:conj}
    Let $\Cls$ be a class of finite graphs.
    Then, $\Cls$ is labelled-well-quasi-ordered
    if and only if
    $\Cls$
    does not existentially transduce all finite paths.
\end{conjecture}


\paragraph*{Contributions.} At the core of our analysis is the fundamental
connection between a deep automata theoretic construction (the existence of
\emph{forward ramseyan factorisations} on trees \cite{COLC07}) and a deep
combinatorial construction (\emph{gap embedding} on trees
\cite{DERSHOWITZ200380}). These two constructions were meant to be used
together, as we will try to advocate in this paper.

We are also interested in the \emph{decidability} of the WQO property. This is
a question that has recently gained traction in the community, and is of
practical interest.

\begin{theorem}[restate=effective-image:thm,label={effective-image:thm}]
    \label{effective-image:thm}
    Let $I$ be an $\MSO$ interpretation
    from finite trees to undirected graphs.
    There exists a computable $k \in \Nat$
    such that $\image{I}$
    is $k$-well-quasi-ordered
    if and only if 
    $\image{I}$ is wqo-well-quasi-ordered.
    These properties are furthermore decidable.
\end{theorem}

\begin{theorem}[restate=pouzet2:thm]
    \label{pouzet-2:thm}
    Let $\Cls$ be a class of finite graphs of bounded clique-width.
    Then, $\Cls$ is labelled-well-quasi-ordered
    if and only if 
    $\Cls$ is wqo-well-quasi-ordered.
\end{theorem}

\begin{theorem}[restate=transductions-paths:thm,label={transductions-paths:thm}]
    \label{transductions-paths:thm}
    Let $\Cls$ be a class of \kl{bounded clique-width}.
    Then $\Cls$ is \kl{labelled-well-quasi-ordered}
    if and only if
    it does not \kl{existentially transduce}
    all \kl{finite paths}.
\end{theorem}


\begin{conjecture}[Szymon's Collapsing Conjecture]
    \label{nip-cw:conj}
    Let $\Cls$ be a class of graphs.
    If $\Cls$ is labelled-wqo,
    then $\Cls$ is of bounded clique-width.
\end{conjecture}

The motto would then be ``the only well-quasi-orderings on graphs are the gap
embeddings'' and would provide a fundamental barrier to the complexity of
induced subgraph well-quasi-orderings.

\paragraph*{Outline.}
