\section{Introduction}
\label{sec:introduction}

\AP The theory of well-quasi-orderings (WQOs) provides a powerful combinatorial
setting that has found applications in various areas of mathematics and
computer science. In graph theory, the celebrated result of Robertson and
Seymour~\cite{ROBSEY04} states that the class of all finite graphs is
well-quasi-ordered under the minor relation, a profound result with deep
algorithmic consequences. WQOs are also at the heart of \emph{well-structured
transition systems}, infinite state transitions systems over which verification
algorithms can be designed~\cite{ABDU96,ABDU98}. One of the appeal of WQOs is
that they are closed under various operations, such as the sum and the product
of WQOs. As an example, the closure of WQOs under the finite words, also known
as Higman's lemma \cite{HIG52}, has been used in the verification of so-called
\emph{lossy channel systems}~\cite{ABDU93}.


% Well-quasi-ordered classes of graphs
\AP Undirected finite graphs are naturally equipped with the \reintro{induced
subgraph relation}, where $G$ is an induced subgraph of $H$ if $G$ can be
obtained from $H$ by deleting vertices. Unlike
the graph minor relation, the induced subgraph relation is not a
well-quasi-ordering on the class of all finite graphs, as witnessed by the
infinite family of cycles of increasing size, which form an infinite sequence of 
pairwise incomparable graphs.
However, some classes of finite graphs are well-quasi-ordered under the induced
subgraph relation: for instance, the class of all finite paths, or the class of
all finite cliques. Distinguishing which classes of (finite) graphs are
well-quasi-ordered under the induced subgraph relation is a long-standing open
problem in graph theory, dating back to Pouzet's seminal work \cite{POUZ72}.

\AP In general, there is little hope to characterise all such classes, since
any countable order with finite infixes can be embedded into the induced
subgraph relation on finite graphs (this is a consequence of \cite[Lemma
5.1]{Kuske06}).
\maelin{It does not seems that usual? Next sentence sounds weird. Maybe we should say something like "we can enrich the definition to say something" ?}
Fortunately, one usually considers \emph{labelled}
well-quasi-orderings on classes of finite graphs. Given a class $\Cls$ of
finite graphs, and a well-quasi-ordering $(X, \leq)$, one can consider the
class $\Label{X}{\Cls}$ of graphs in $\Cls$ whose vertices are labelled by
elements of $X$, that is, equipped with a function 
$\ell \colon V(G) \to X$. The induced subgraph relation is then extended to labelled
graphs by requiring that the labels are preserved by the embedding, i.e. that
the label of a vertex in the smaller graph is less than or equal to the label
of its image in the larger graph. A class $\Cls$ of finite graphs is said to be
\reintro{$(X,\leq)$-well-quasi-ordered} if the class $\Label{X}{\Cls}$ is
well-quasi-ordered under the labelled induced subgraph relation. We will say
that it is \reintro{$k$-well-quasi-ordered} if it is
$(X,\leq)$-well-quasi-ordered where $X$ is a $k$-element antichain, and that it
is \reintro{$\forall$-well-quasi-ordered} if it is
$(X,\leq)$-well-quasi-ordered for every well-quasi-ordering $(X, \leq)$.

\begin{example}
  \label{ex:wqo-classes}
  The class $\intro*\Cycles$ of all finite cycles is not \kl{well-quasi-ordered},
  since the infinite sequence of cycles forms an infinite \kl{antichain}.
  The class $\intro*\Paths$ of all finite paths is \kl{well-quasi-ordered}
  since $(\Paths, \isubleq)$ is isomorphic to $(\mathbb{N}, \leq)$.
  However, it is not \kl{$2$-well-quasi-ordered}, since the infinite
  sequence of paths with colored endpoints forms an infinite \kl{antichain}.
  The class $\intro*\Cliques$ of all finite cliques is \kl{$\forall$-well-quasi-ordered},
  since any $X$-labelled clique can be identified with a finite multiset of labels from $X$,
  and the set of finite multisets over a \kl{well-quasi-order} is itself \kl{well-quasi-ordered}
  by standard results on \kl{well-quasi-orders} \cite{SCSC12}.
\end{example}

\begin{figure}
  \centering
  \begin{tabular}{c|c|c|c}
    \toprule
    \textbf{Class} & \kl{WQO} & \kl{$2$-WQO} & \kl{$\forall$-WQO} \\
    \midrule
    $\Cycles$ & \no & \no & \no \\
    $\Paths$ & \yes & \no & \no \\
    $\Cliques$  & \yes & \yes & \yes \\
    \bottomrule
  \end{tabular}
  \caption{Well-quasi-ordering properties of classes in \cref{ex:wqo-classes}.}
  \label{fig:wqo-classes}
\end{figure}


\AP The study of labelled structures is a recurring theme in the theory of
well-quasi-orderings: classical results such as Higman's lemma \cite{HIG52} or
Kruskal's tree theorem \cite{KRUSK60} are actually stating that some classes of
finite relational structures (respectively finite linear orders and finite meet
trees) are well-quasi-ordered under the labelled embedding relation, for every
well-quasi-ordering of labels. It was conjectured by Pouzet \cite{POUZ72} that
being \kl{$\forall$-well-quasi-ordered} reduces to being
\kl{$2$-well-quasi-ordered} for hereditary classes of structures (see
\cite[Problems 9, (1)]{Pouzet24} for a modern formulation). Another interesting
conjecture of Pouzet \cite[Problems 12]{Pouzet24} states that when a class
$\Cls$ is \kl{$\forall$-well-quasi-ordered}, one can add a total order on the
vertices of its structures such that the new class is still
\kl{$\forall$-well-quasi-ordered}. This conjecture gained interest in the
recent development of algebraic methods for polynomials over relational
structures \cite[Section 3, Theorem 11 and 12]{GHOLAS24}, where ``monomials''
are defined as structures labelled by $(\mathbb{N}, \leq)$. We refer 
to this property as being \reintro{orderably $\forall$-well-quasi-ordered}.


\AP Even for classes of finite undirected graphs, these conjectures remain
open. Several works dating back to the 90s have idendified structural
properties of hereditary classes of finite graphs that guarantee that they are
\kl{$\forall$-well-quasi-ordered}, such as being of bounded tree-depth
\cite{DING92}. In 2010, Daligault, Rao and Thomassé \cite{DRT10} initiated the
study in the reverse direction, by considering a syntax for graph classes
(using NLC graph expressions), and identifying which ones among them are
\kl{$2$-well-quasi-ordered}. In this context, they verified Pouzet's
conjectures, but were unable to generalise their results to arbitrary graph
classes\mael{arbitrary subclasses ?} \cite[Conjecture 4]{DRT10}. All the classes considered in \cite{DRT10}
have so-called \emph{bounded clique-width},\footnote{This will be defined
formally in \cref{sec:prelims}} and it is believed that it is enough to
understand hereditary classes of bounded clique-width to solve these
conjectures in full generality, as stated in the following conjecture.

\begin{conjecture}[{\cite[Conjecture 5]{DRT10}}]
    \label{cwqo:conj}
    Let $\Cls$ be a hereditary class of finite graphs of 
    that is \kl{$2$-well-quasi-ordered}.
    Then, $\Cls$ is of bounded clique-width.
\end{conjecture}


On parallel with the conjectures of Pouzet, another line of research has
focused on finding structural obstructions to being
\kl{$\forall$-well-quasi-ordered}. The natural candidate for such an
obstruction is the class of all finite paths \cref{fig:wqo-classes}. For the
particular classes studied in \cite{DRT10}, it was shown that every class that
is not \kl{$2$-well-quasi-ordered} contains all finite paths \cite[Corollary
2]{DRT10}. There are examples of hereditary classes that are not
\kl{$2$-well-quasi-ordered} and do not contain all finite paths, but on every
known such example, one can ``extract'' finite paths using very simple logical
formulas. This lead to the following conjecture, where 
\kl{existentially transduces} is a notion from model theory
(see \cref{sec:prelims} for a formal definition).

\begin{conjecture}[{\cite[Conjecture 26]{LOPEZ24}}]
    \label{conj:path-transduction}
    Let $\Cls$ be a hereditary class of finite graphs.
    If $\Cls$ is not \kl{$2$-well-quasi-ordered},
    then it \kl{existentially transduces} the class of all finite paths.
\end{conjecture}

\AP Another kind of obstruction to being \kl{$\forall$-well-quasi-ordered}
comes from the existence of infite antichains with a periodic structure that
were called \reintro(weak){periodic antichains} in \cite[Section 7]{ALM17}.

\begin{conjecture}[{\cite[Conjecture 2]{ALM17}}]
  \label{lozin:conj}
  Let $\Cls$ be a hereditary class of finite graphs that is not
  \kl{$2$-well-quasi-ordered}.
  Then, there exists 
  an infinite \kl{(weak) periodic antichain} in $\Cls$.
\end{conjecture}

\maelin{Is it really new or does it build on existing tools ?}
\AP In 2024, Lopez \cite{LOPEZ24} introduced a new technique to study
well-quasi-orderings on classes of graphs of bounded \emph{linear}
clique-width, that leveraged combinatorial tools from automata theory (Simon's
factorisation forests \cite{SIMO90}). This allowed to prove a weakening of
Pouzet's conjectures \cite[Corollary 2]{LOPEZ24}, namely, that a hereditary
class of bounded linear clique-width is \kl{$\forall$-well-quasi-ordered} if
and only if it is \kl{$k$-well-quasi-ordered} for every $k \in \mathbb{N}$. It
is apparent from the proof that the techniques also yield a natural ordering on
the vertices whenever the class is \kl{$\forall$-well-quasi-ordered}, although
this is not formally stated in the paper. 
It was conjectured \cite[Section 5]{LOPEZ24} that these techniques could be refined to work on classes of
bounded clique-width (not necessarily linear), and actually obtain the
correspondence between \kl{$2$-well-quasi-ordering} and
\kl{$\forall$-well-quasi-ordering}.


\paragraph{Contributions} \AP Our main contribution is \cref{main:theorem},
that answers positively to the conjectures of \cite{LOPEZ24}.
To that end, we first consider generalisation of the framework 
of \cite{DRT10} from NLC graph expressions to
\kl{MSO interpretations} from trees (these will be formally defined 
in \cref{sec:prelims}). This also generalises 
the setting of \cite{LOPEZ24} that stutied \kl{MSO interpretations} from
\emph{words} (i.e., paths). Our main combinatorial result is the following.

\begin{theorem}
  \label{main:theorem}
  Let $\Cls$ be the image of a class of finite trees 
  under an $\MSO$ interpretation $\someInterp$. Then, the following are equivalent
  and decidable given $\someInterp$:
  \begin{enumerate}
    \item $\Cls$ is \kl{$2$-well-quasi-ordered},
    \item $\Cls$ is \kl{$\forall$-well-quasi-ordered},
    \item $\Cls$ is \kl{orderably $\forall$-well-quasi-ordered},
  \end{enumerate}
\end{theorem}

From \cref{main:theorem},
we derive the following corollary that solves
Pouzet's conjectures for hereditary classes of bounded clique-width.

\begin{corollary}
  \label{main:corollary}
  The same holds when $\Cls$ is a hereditary class of bounded clique-width graphs.
\end{corollary}

Because our proof techniques exhibit structural 
obstructions to being \kl{$2$-well-quasi-ordered}, we also 
obtain the following characterisations in
\cref{thm:characterisations}.
These answer positively to the conjectures of
\cite{ALM17,LOPEZ24}. In particular, for a hereditary class $\Cls$ to be
\kl{$2$-well-quasi-ordered}, it suffices that it avoids very simple
\kl{(strong)\mael{and weak ?} periodic antichains}.

\begin{theorem}
  \label{thm:characterisations}
  Let $\Cls$ be a hereditary class of finite graphs of bounded clique-width.
  Then, the following are equivalent:
  \begin{enumerate}
    \item $\Cls$ is not \kl{$2$-well-quasi-ordered},
    \item There exists a class $\Cls[D] \subseteq \Cls$
      of \kl{bounded linear clique-width} which is not 
      \kl{$2$-well-quasi-ordered},
    \item There exists a \kl{strong periodic antichain} in $\Cls$,
    \item There exists a \kl{weak periodic antichain} in $\Cls$,
    \item $\Cls$ \kl{existentially transduces} the class of all finite paths.
  \end{enumerate}
\end{theorem}

The results of \cref{main:corollary}
extend to non-hereditary classes of
bounded clique-width graphs, but for the sake of clarity and space, we only
state them on hereditary ones. \maelin{C'est peut-être un peu trop fort de dire ça, tu prévois d'essayer de le justifier quelque part ?}
On the other hand, \cref{thm:characterisations} does not immediately 
extend to non-hereditary classes, as witnessed by the following example.

\begin{example}
  \label{ex:non-hereditary}
  The class $\Cls$ of complete binary trees has bounded clique-width, is not \kl{$2$-well-quasi-ordered},
  but contains no \kl{periodic antichain}, 
  and every subclass $\Cls[D] \subseteq \Cls$ of bounded linear clique-width
  is \kl{$2$-well-quasi-ordered}.
  It happens that $\Cls$ \kl{existentially transduces} the class of all finite paths,
  but it does not \kl{existentially interprets} them.
\end{example}
\begin{proof}
  It is well-known that the class of complete binary trees has
  bounded clique-width. Furthermore, it is not \kl{$2$-well-quasi-ordered},
  since one can color the leaves and the root of the complete binary trees
  to form an infinite \kl{antichain}.
  On the other hand, any subclass $\Cls[D] \subseteq \Cls$ of bounded linear
  clique-width must be finite: indeed, complete binary trees have unbounded
  linear clique-width. Finally, because \kl{periodic antichains} have 
  bounded linear clique-width (see \cref{sec:interpreting-paths})
  there cannot be any \kl{periodic antichain} in $\Cls$.

  The \kl{existential transduction} of all finite paths from $\Cls$ is obtained
  by transducing all induced subgraphs, that in particular contains all finite
  paths. There cannot be an \kl{existential interpretation} of all finite paths from $\Cls$,
  because paths have only two automorphisms (reversing and identity), while
  complete binary trees have many more, and interpretations must preserve automorphisms. \mael{ref ?}
\end{proof}




\paragraph*{Outline}
\cref{sec:prelims} introduces the necessary background on well-quasi-orders,
graphs, MSO interpretations and first-order transductions.
\cref{sec:ramseyan} shows how to reduce the study of graph classes obtained by
MSO interpretations from trees to the study of trees labelled over finite monoids. This section also shows how to leverage 
\kl{forward Ramseyan splits} to obtain well-behaved tree representations of graphs,
that are well-quasi-ordered (as trees) with respect to the so-called \emph{gap embedding} relation.
\cref{sec:bad-patterns} identifies combinatorial obstructions to being 
2-well-quasi-ordered in images of interpretations from trees, and is the core combinatorial part of the paper.
\cref{sec:hereditary-classes} brings together the previous results to prove
our \cref{main:theorem} and \cref{main:corollary}. We also 
discuss the equivalence between the first two items of \cref{thm:characterisations}.
The remaining items of \cref{thm:characterisations} 
are proven equivalent in \cref{sec:interpreting-paths}.
Finally, in 
\cref{sec:conclusion} we discuss some perspectives and open problems.
