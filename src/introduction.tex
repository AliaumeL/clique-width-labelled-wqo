\section{Introduction}
\label{sec:introduction}

 Context:
 \begin{itemize}
 \item WQOs are cool (graph theory, verification of infinite systems (abdulla et al.),
                  lossy channel systems (etessami et al.), etc.)
 \item Composition of wqos (sum, products, finite words, finite trees, etc.)
 \item Graphs: not clear for induced substructure, lot of literature on the subject, seems
   somehow orthogonal to the usual notions of "well-structured" classes of graphs (sparsity)
 \item Two conjectures by Pouzet:
       2-wqo => labelled-wqo
       labelled-wqo => wqo-wqo
 \item Tree decompositions of graphs and *gap-embedding relation*
   \item gap embedding relation  -> Dershowitz and Manna
   \item priority chanel systems -> nested higman ~ gap embedding relation on "linear trees"
   \item gap embedding -> nested higman in general [Freund 2020]
 \item Connection between gap embedding and tree decompositions has already been 
   established [LOPEZ24], exploring again a paper of \cite{DRT10}
\end{itemize}
 
 Contributions:
 \begin{itemize}
     \item We connect two deep results in the literature: the *gap embedding*
 and the existence of *forward ramseyan factorisations* in trees.
 These two technical results were meant to be used together, as we will try to
 demonstrate in this paper.
 This allows us to answer positively to the weak First Pouzet conjecture on
 classes of graphs of bounded clique-width, and to the strong Second Pouzet conjecture
 on classes of graphs of bounded clique-width.

\item This analysis also connects to the sturcural graph theory, because we 
 are able to show the following conjecture due to Sylvain Schmitz
 in the context of bounded clique-width calsses of graphs:

 \item  A class of graphs is labelled-wqo if and only if it does not FOexistentially-transduce
 the class of all paths.

\item We also prove  that
 Well-quasi-ordered classes of graphs  are NIP
 and conjecture that 
 NIP classes that are labelled-wqo are of bounded clique-width.
 This would imply the *strong Pouzet conjecture* for every class of graphs.
\end{itemize}


\subparagraph{Context.} The theory of well-quasi-orderings (WQOs) is a powerful
combinatorial setting that has found applications in various areas of
mathematics and computer science. In graph theory, the celebrated result of
Robertson and Seymour~\cite{ROBSEY04} states that the class of all finite
graphs is well-quasi-ordered under the minor relation. WQOs are also at the
heart of \emph{well-structured transition systems}, infinite state transitions
systems over which verification algorithms can be
designed~\cite{ABDU96,ABDU98}.

One of the appeal of WQOs is that they are closed under various operations,
such as the sum and the product of WQOs. As an example, the closure of WQOs
under the finite words, also known as Higman's lemma \cite{HIG52}, has been
used in the verification of so-called \emph{lossy channel
systems}~\cite{ABDU93}. 

In the context of graphs, there is a theoretical interest in understanding
which classes of finite graphs are WQOs when endowed with the induced subgraph
relation. Ding \cite{DING92}, the study of WQOs on classes of


\begin{conjecture}
    \label{pouzet1:conj}
    Let $\Cls$ be a class of finite graphs.
    Then, $\Cls$ is 2-well-quasi-ordered
    if and only if
    $\Cls$ is labelled-well-quasi-ordered.
\end{conjecture}

\begin{conjecture}
    \label{pouzet2:conj}
    Let $\Cls$ be a class of finite graphs.
    Then, $\Cls$ is labelled-well-quasi-ordered
    if and only if
    $\Cls$ is wqo-well-quasi-ordered.
\end{conjecture}

Classes of graphs that are labelled-wqo are of interest because they remains
stable under monadic colorings, an operation that is omnipresent in the
structrual graph theory community. Classes of graphs that are wqo-wqo are of
interest because they can be understood as \emph{operators} on
well-quasi-orders, much like the finite word and finite tree constructions. In
the case of (ordered) graphs, these two constructions are special cases of the
induced substructure relation on the class of linear orders and tree orders.

Attempts to prove these conjectures often relied on assuming some kind of
structural property of the class of graphs, such as enjoying suitable
\emph{tree decompositions}. Now, to work on those trees, one needs to place a
well-quasi-ordered relation on them. This is usually done using the tree
embedding relation, also known as Kruskal's embedding relation, which is a
well-quasi-ordering on trees \cite{KRU72}. This works when the function that
maps a tree to a graph is simple enough, but more often than not, this is not
the case. In \cite{DRT10}, the authors introduce a \emph{tailored} embedding
relation on trees, and prove that it is a well-quasi-ordering under certain
hypotheses, leveraging a tedious and error-prone proof using a \emph{minimal
bad sequence argument} \cite{NASH65}. 
\todo{talk about clique width now}

We know since \cite{FREU20} and \cite{LOPEZ23} that one can easily inductively
define well-quasi-orderings of increasing complexity on a given set. One
example of such a construction is the \emph{gap embedding relation} of
Derhowitz and Tzameret \cite{DERSHOWITZ200380}. This relation was already at
the core of \emph{priority channel systems}, although in a somewhat hidden form
\cite{HSS13}.  It was also noticed in \cite{LOPEZ24} that the \emph{ad-hoc}
construction of \cite{DRT10} was in fact a particular instance of the gap
embedding relation on trees.

\paragraph*{Contributions.} We continue the line of research initiated in
\cite{LOPEZ24} and explore the \emph{automata theoretic} approach to the Pouzet
conjectures. At the core of our analysis is the fundamental connection between
a deep automata theoretic construction (the existence of \emph{forward ramseyan
factorisations} on trees \cite{COLC07}) and a deep combinatorial construction
(\emph{gap embedding} on trees \cite{DERSHOWITZ200380}). These two
constructions were meant to be used together, as we will try to advocate in
this paper.

We are also interested in the \emph{decidability} of the WQO property. This is
a question that has recently gained traction in the community, and is of
practical interest.

\begin{theorem}[restate=effective-image:thm,label={effective-image:thm}]
    \label{effective-image:thm}
    Let $I$ be an $\MSO$ interpretation
    from finite trees to undirected graphs.
    There exists a computable $k \in \Nat$
    such that $\image{I}$
    is $k$-well-quasi-ordered
    if and only if 
    $\image{I}$ is wqo-well-quasi-ordered.
    These properties are furthermore decidable.
\end{theorem}

\begin{theorem}[restate=pouzet2:thm]
    \label{pouzet-2:thm}
    Let $\Cls$ be a class of finite graphs of bounded clique-width.
    Then, $\Cls$ is labelled-well-quasi-ordered
    if and only if 
    $\Cls$ is wqo-well-quasi-ordered.
\end{theorem}

\begin{theorem}[restate=transductions-paths:thm,label={transductions-paths:thm}]
    \label{transductions-paths:thm}
    Let $\Cls$ be a class of \kl{bounded clique-width}.
    Then $\Cls$ is \kl{labelled-well-quasi-ordered}
    if and only if
    it does not \kl{existentially transduce}
    all \kl{finite paths}.
\end{theorem}

\begin{remark}
    \label{transductions-paths:rem}
    If $\Cls$ is labelled-wqo, then it cannot transduce
    the class of all finite paths.
\end{remark}

\begin{remark}
    If $\Cls$ is labelled-wqo, then it cannot (FO)-transduce 
    the class of all graphs (also known as monadically NIP).
\end{remark}

\begin{conjecture}
    \label{nip-cw:conj}
    Let $\Cls$ be a class of graphs.
    If $\Cls$ is NIP and labelled-wqo,
    then $\Cls$ is of bounded clique-width.
\end{conjecture}

\paragraph*{Outline.}
