\section{Preliminaries}
\label{sec:preliminaries}

\paragraph*{Graphs.} A \intro{graph} is a pair $G = (V, E)$ where $V$ is a set
of \intro{vertices} and $E \subseteq V \times V$ is a set of \intro{edges}. A
graph is \intro{undirected} if $(u, v) \in E$ implies $(v, u) \in E$ for all
$u, v \in V$. In this paper we will focus on finite undirected graphs. Given a
set $X$, an \intro{$X$-labelled graph} is a graph equipped with a function
$\ell \colon V \to X$ that assigns a label to each vertex. Given a class $\Cls$
of finite undirected graphs, write $\Label{X}{\Cls}$ for the class of freely
$X$-labelled graphs in $\Cls$, i.e., the class of all \kl{$X$-labelled graphs}
$G = (V, E, \ell)$ such that $G \in \Cls$.


\paragraph*{Well-quasi-orderings.} A quasi-ordered set $(X, \leq)$ is a set $X$
equipped with a reflexive and transitive relation $\leq$. A sequence
$\seqof{x_i}$ of elements in $X$ is \intro(sequence){good} if there exist $i <
j$ such that $x_i \leq x_j$. A quasi-ordered set is \intro{well-quasi-ordered}
(\reintro{wqo}) if every infinite sequence is \kl(sequence){good}.

\AP
Let $(X, \leq)$ be a quasi-order, and let $\Cls$ be a class of \kl{$X$-labelled
graphs}. Then, we define the \intro{$X$-induced quasi-order} on $\Cls$ as
follows: for $G, H \in \Cls$, we write $G \leq H$ if there exists a
\kl{monomorphism} $f \colon G \to H$ such that for all $v \in V$, we have
$\ell_G(v) \leq \ell_H(f(v))$. 

\AP A class $\Cls$ of finite undirected graphs is
\intro{labelled-well-quasi-ordered} (\reintro{labelled-wqo}) if for every
finite set $(X,=)$, the class $\Label{X}{\Cls}$ is \kl{well-quasi-ordered}
using the \kl{$X$-induced quasi-order}. A class of graph is
\intro{wqo-well-quasi-ordered} (\reintro{wqo-wqo}) if for every
\kl{well-quasi-order} $(X, \leq)$, the class $\Label{X}{\Cls}$ is
\kl{well-quasi-ordered} using the \kl{$X$-induced quasi-order}.

\paragraph*{Transductions.} An \intro{$\MSO$-interpretation} between two
classes $X$ and $Y$ of finite relational structures is given by formulii
$\phi_{\Delta}, \phi_{\delta}, \phi_{R}$ in \kl{monadic second-order logic}.
The interpretation defines a relation between structures of $X$ and $Y$ as
follows: ... We define similarly the notion of \intro{existential
transduction}.

\paragraph*{Bounded clique-width.} A class of graph has \intro{bounded
clique-width} if it is included in the image of some \kl{$\MSO$-interpretation}
from finite trees to finite undirected graphs \cite{COUR91}.

\paragraph*{Automata Theoretical Results.} 
Monoid morphism, idempotents,
factorisation. Simon's factorisation theorem.





