\section{Preliminaries}
\label{sec:prelims}

We assume the reader to be familiar with basic notions of logic on finite
relational structures, such as monadic second-order logic ($\MSO$), first-order
logic and quantifier-free formulas.

\paragraph*{Graphs} \AP
A \intro{graph} is a pair $G = (V, E)$ where $V$ is a set
of \intro{vertices} and $E \subseteq V \times V$ is a set of \intro{edges}. A
graph is \intro(graph){undirected} if $(u, v) \in E$ implies $(v, u) \in E$ for all
$u, v \in V$. In this paper we will focus on finite undirected graphs. Given a
set $X$, an \intro{$X$-labelled graph} is a graph equipped with a function
$\ell \colon V \to X$ that assigns a label to each vertex. Given a class $\Cls$
of finite undirected graphs, write $\Label{X}{\Cls}$ for the class of freely
$X$-labelled graphs in $\Cls$, i.e., the class of all \kl{$X$-labelled graphs}
$G = (V, E, \ell)$ such that $G \in \Cls$.

\AP
Whenever $(X, \leq)$ is a quasi-order, we say that a graph $G$ \intro(graph){embeds}
into a graph $H$ if there exists a \intro{monomorphism} from $G$ to $H$, i.e.,
an injective function $f \colon V_G \to V_H$ such that for all $u, v \in V_G$,
$(u, v) \in E_G$ if and only if $(f(u), f(v)) \in E_H$, and such that for all
$v \in V_G$, we have $\ell_G(v) \leq \ell_H(f(v))$. We write $G \intro*\isubleq H$ to
denote that $G$ \kl(graph){embeds} into $H$, which we also call the \reintro{induced
subgraph relation}. When not specified, we consider that labels are
quasi-ordered by equality.

\AP A class of graphs $\Cls$ is a \intro{hereditary class} if it is closed under
taking \kl{induced subgraphs}, i.e., if for all $G, H$ such that 
$G \isubleq H$ and $H \in \Cls$, we have $G \in \Cls$.
The \intro{hereditary closure} of a class of graphs $\Cls$ is the smallest
\kl{hereditary class} containing $\Cls$.

\AP
An \intro{ordered graph} is a finite undirected graph $G = (V, E)$ equipped
with a linear order $\leq_G$ on its vertices. The notion of \kl{induced
subgraph relation} extends to ordered graphs by requiring that the embedding $f
\colon V_G \to V_H$ preserves the order, i.e., for all $u, v \in V_G$, $u
\leq_G v$ if and only if $f(u) \leq_H f(v)$.

\paragraph*{Well-quasi-orderings} A quasi-ordered set $(X, \leq)$ is a set $X$
with a reflexive and transitive relation $\leq$. A sequence
$\seqof{x_i}$ of elements in $X$ is \intro(sequence){good} if there exist $i <
j$ such that $x_i \leq x_j$. A quasi-ordered set is \intro{well-quasi-ordered}
(\reintro{wqo}) if every infinite sequence is \kl(sequence){good}.
A \intro{bad sequence} is an infinite sequence that is not \kl(sequence){good},
and a sequence of pairwise incomparable elements is called an
\intro{antichain}.

\AP As mentioned in the introduction, several notions of well-quasi-ordering
can be defined for a given class of finite undirected graphs $\Cls$. Let us 
briefly state them:
$\Cls$ is \reintro{well-quasi-ordered}
    if the class $(\Cls, \isubleq)$ is \kl{well-quasi-ordered},
$\Cls$ is \intro{$k$-well-quasi-ordered} for some $k \in \mathbb{N}$,
    if for every set $(X,=)$ of size $k$, the class
    $(\Label{X}{\Cls}, \isubleq)$ is \kl{well-quasi-ordered},
and $\Cls$ is \intro{$\forall$-well-quasi-ordered} if for every 
    \kl{well-quasi-order} $(X, \leq)$, the class
    $(\Label{X}{\Cls}, \isubleq)$ is \kl{well-quasi-ordered}.

Finally, a class of graph is \intro{orderably well-quasi-ordered} (resp.
\reintro{orderably $k$-well-quasi-ordered}, resp. \reintro{orderably
$\forall$-well-quasi-ordered}) if there is a way to equip each graph $G$ in
$\Cls$ with a linear order $\leq_G$ on its vertices such that the resulting
class of \kl{ordered graphs} is \kl{well-quasi-ordered} (resp.
\kl{$k$-well-quasi-ordered}, resp. \kl{$\forall$-well-quasi-ordered}).





\paragraph*{Trees} \AP In this paper, \intro{trees} are finite, binary, and
ordered (have \intro(child){left} and \reintro{right children}). They are
understood as finite relational structures over the signature $\sigma =
(\treeleq, \treesibleq)$ where $\treeleq$ is the \intro{ancestor relation}, and
$\treesibleq$ is the \intro{sibling order} on the nodes of the tree, extended
to all pairs of nodes as a topological order. In a tree $\aTree$, the
\intro(tree){root} is denoted by $\intro*\treeRoot(\aTree)$, with $\aTree$
being left implicit when clear from context. The set of \intro(tree){leaves} is
denoted by $\Leaves{\aTree}$. Given two nodes $x$ and $y$ in a tree $T$, their
\intro{least common ancestor} is denoted by $\intro*\lca(x,y)$. We will often
use the \intro{parent relation}, which state that $y$ is the unique immediate
\kl(tree){ancestor} of $x$.

\AP
As in the case of graphs, we will be interested in
vertex-labelled trees where labels are taken from a \kl{well-quasi-order} $(X,
\leq)$. We will also be interested in edge-labelled trees, typically using a
finite set of labels $\Sigma$. These trees are denoted using
$\intro*\Trees{\Sigma}{X}$ to denote the class of trees whose edges are labelled by
elements of $\Sigma$ and whose nodes are labelled by elements of $X$. These are
omitted when unused (i.e., $\Trees{}{}$ denotes the class of unlabelled trees,
and $\Trees{\Sigma}{}$ denotes the class of trees whose edges are labelled by
elements of $\Sigma$).
Given two nodes $x,y$ such
that $y$ is an \kl(tree){ancestor} of $x$, 
the notation $\intro*\tword{x}{y}$ denotes the \emph{word}
in $\Sigma^*$ obtained by reading the labels of the edges on the path from $x$
to $y$ in $\aTree$, we can leave $\aTree$ implicit if clear from context.

\paragraph*{Bounded Clique-Width}
\AP An \intro{$\MSO$-interpretation} $\someInterp$ from $\Trees{\Sigma}{}$ to
finite undirected graphs is given by a tuple of $\MSO$ formulas that specify
how to respectively: select a subset of nodes of the leaves of tree to be the
vertices of the graph $\phi_{\mathsf{univ}}(x)$, define the edges of the graph
$\phi_{\mathsf{edge}}(x,y)$, and define the domain of the interpretation
$\phi_{\mathsf{dom}}$. An interpretation $\someInterp$ is called
\intro(interpretation){simple} if $\phi_{\mathsf{univ}}(x)$ and
$\phi_{\mathsf{dom}}$ are tautologies.

\begin{example}
  \label{ex:mso-interp-cycles}
  An \kl{$\MSO$-interpretation} $\someInterp$ from $\Trees{}{}$ to finite undirected graphs
that produces all cliques of odd size can be defined as follows: 
    $\phi_{\mathsf{dom}}$ is the formula that is true on trees with an odd number of leaves,
    $\phi_{\mathsf{univ}}(x)$ is the formula that is true if and only if $x$ is a leaf,
    $\phi_{\mathsf{edge}}(x,y)$ is the formula that is true if and only if $x \neq y$.
\end{example}


\AP A class of graph has \intro{bounded clique-width} if it is included in the
image of some \kl{$\MSO$-interpretation} from finite trees to finite undirected
graphs \cite{COUR91}. It has \intro{bounded linear clique-width} if it is
included in the image of some \kl{$\MSO$-interpretation} from \intro{linear
trees} (i.e., trees where every node has at most one non-leaf child) to finite
undirected graphs. 

\paragraph{Transductions from graphs to graphs} \AP Another notion of logical
transformation that will be relevant in this paper is that of \kl{existential
transduction} and \kl{existential interpretation} that were mentioned in
\cref{thm:characterisations,conj:path-transduction}, and are defined in a
similar way as \kl{$\MSO$-interpretations}.

\AP An \intro{existential interpretation} from (labelled) graphs to graphs is
given by an existential first order formula $\phi_{\text{univ}}(x)$ that
selects the vertices of the interpreted graph, an existential first order
formula $\phi_{\text{edge}}(x,y)$ that selects the edges of the interpreted
graph, and an existential first order formula $\phi_{\text{dom}}$ that selects
the graphs of $\Cls$ on which the interpretation is defined. The semantics is
that for every graph $G \in \Cls$ satisfying $\phi_{\text{dom}}$, the
interpreted graph $\mathcal{I}(G)$ has for vertex set the vertices of $G$
satisfying $\phi_{\text{univ}}$, and has an edge between two vertices $u,v$ if
and only if $G$ satisfies $\phi_{\text{edge}}(u,v)$ (which is assumed to be
symmetric and irreflexive on the selected vertices). A class $\Cls$
\intro{existentially interprets} a class $\mathcal{D}$ if there exists an
existential interpretation $\someInterp$ from $\Cls$ to graphs such that
$\mathcal{D} \subseteq \someInterp(\Cls)$. We say that $\Cls$
\intro{existentially transduces} $\mathcal{D}$ if there exists a finite
labelling set $\Sigma$ such that $\Label{\Sigma}{\Cls}$ \kl{existentially
interprets} $\mathcal{D}$.



