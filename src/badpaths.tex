\section{Bad Patterns in Images of Interpretations}
\label{sec:bad-patterns}

In this section, we will isolate combinatorial obstructions to being
\kl{2-well-quasi-ordered} in classes of graphs that are images of \kl{simple
MSO interpretations} from trees. We will take the same notations as in
\cref{sec:ramseyan}. 

\begin{definition}
    \label{ramseyan-branch:def}
    Let $\aTree$ be a tree and $\spt$ be a \kl{forward Ramseyan split} of height $N$.
    A \intro{bough of level $k$} in $\aTree$ is an infix of a branch of $\aTree$
    such that its maximal and minimal elements have level $k$,
    and such that 
    all elements of the bough have level greater or equal to $k$.

    The \intro{dimension of a bough} is the number of elements of level $k$
    in the \kl{bough}.
\end{definition}

\AP Let $B$ be a \kl{bough of level $k$} in $\aTree$. Given two nodes $b_1$ and
$b_2$ in $B$ such that $\spt(b_1) = \spt(b_2) = k$, we define the $\aTree_{b_1:b_2}$
as the set of nodes $x$ in $\aTree$ such that $b_1 \treeleq x$ and $\neg (b_2
\treeleq x)$. For every leaf $x \in \aTree_{b_0: b_n}$ where $n$ is the
\kl{dimension of the bough} $B$, we define $\Bt(x)$ to be the least ancestor of
$x$ that belongs to $B$. Similarly, we define $\Bl(x)$ to be the least element
of $B$ that is greater or equal to $\Bt(x)$, and $\Br(x)$ to be the greatest
element of $B$ that is less or equal to $\Bt(x)$. To every leaf $x$ of $B(\aTree)$,
we can therefore associate the following values in $M$:
$\BtL(x) \defined
\tlbl{\t}{\Bt(x)}{x}$, $\BlL(x) \defined \tlbl{\aTree}{\Bl(x)}{\Bt(x)}$,
$\BrL(x) \defined \tlbl{\t}{\Bt(x)}{\Br(x)}$,
and $\BrootL(x) \defined \tlbl{\aTree}{\treeRoot}{\Bl(x)}$. We call this tuple 
the \intro{bough type} of the leaf $x$ with respect to the \kl{bough} $B$.
We refer to
\cref{type-of-a-leaf-in-branch:fig} for an illustration of the type of a leaf
with respect to a given \kl{bough} $B$. We also refer to
\cref{partitionning-a-graph:fig} for an illustration of the resulting partition
of the tree $\aTree$ with respect to a given \kl{bough} $B$.

\begin{figure}
    \centering
    \begin{tikzpicture}[
        branch/.style={
            color=Prune,
            inner sep=0pt,
            minimum size=4pt,
            fill,
            circle
        },
        inner/.style={
            color=A1,
            inner sep=0pt,
            minimum size=4pt,
            draw,
            circle
        },
        staredge/.style={
            color=A1,
            ->,
            dashed
        },
        root/.style={
            color=Prune,
        },
        leaf/.style={
            color=Prune,
        },
        monoid/.style={
            color=A2
        },
        ]
        % first draw the branch 
        \node[root]   (root) at (-2,0) {$\treeRoot$};
        \node[branch] (b0) at (0,0) {};
        \node[branch] (b1) at (4,0) {};
        \node[inner]  (t)  at (2,0) {};
        \node[leaf]   (x)  at (3.5,-2) {$x$};

        \node[above=0.1cm of b0] {$\Bl(x)$};
        \node[above=0.1cm of t]  {$\Bt(x)$};
        \node[above=0.1cm of b1] {$\Br(x)$};

        \draw[staredge] (root) -- (b0);
        \draw[staredge] (b0) -- node[monoid, midway, below] {$\BlL(x)$} (t);
        \draw[staredge] (t)  -- node[monoid, midway, below] {$\BrL(x)$} (b1);
        \draw[staredge] (b1) -- (5,0);

        \draw[staredge] (t)  -- node[monoid, midway, below left] {$\BtL(x)$} (x);
    \end{tikzpicture}
    \caption{The type of a leaf with respect to a given \kl{bough} $B$.}
    \label{type-of-a-leaf-in-branch:fig}
\end{figure}

\begin{figure}
    \centering
    \resizebox{0.9\linewidth}{!}{
    \begin{tikzpicture}[
        localType/.style={
            color=C3,
            thick
        },
        branchProj/.style = {
            color=Prune,
            inner sep=0pt,
            minimum size=4pt,
            fill,
            circle
        },
        leaf/.style={
            color=Prune,
        },
        ]
        \draw (0,0) rectangle (8,2);
        \node (root) at (-1.2,2) {$\treeRoot$};
        \foreach \x in {0,1,2,3,4} {
            \coordinate (n\x) at ({ 2 * \x},0);
            \coordinate (pb\x) at ({ 2 * \x},2);
            \node[branchProj] (b\x) at (pb\x) {};
            \node (lb\x) at ({ 2 * \x},2.4) {$b_{\x}$};
            \draw (n\x) -- (b\x);
        }
        \draw[dashed,<-] (b0) -- (root);
        \draw[dashed] (b4) -- (9,2.5);
        \draw[dashed] (b4) -- (9,1.5);

        \foreach[count=\x] \y in {0,1,2,3} {
            \node (E\x) at ({ 2 * \x - 1},2.6) {$e_{\y}$};
            \draw[->,thick] (b\y) to[bend left=40] (b\x);
            \node (T\x) at ({ 2 * \x - 1},-0.6) {$T_{b_{\y}:b_{\x}}$};
        }
        
        \node (x) at (3,0.2)  {$x$};
        \node (y) at (7,0.2)  {$y$};
        \node[branchProj] (tx) at (3,2) {};
        \node[branchProj] (ty) at (7,2) {};

        \draw[localType, <-] (x)  -- 
        node[midway, left] {$\BtL(x)$} (tx);
        (tx);
        \draw[localType, ->] (tx) -- 
        node[midway, below] {$\BrL(x)$}
        (b2);
        \draw[localType, <-] (tx) -- 
        node[midway, below] {$\BlL(x)$}
        (b1);

        \draw[localType, <-] (y)  -- 
        node[midway, left] {$\BtL(y)$} 
        (ty);
        \draw[localType, ->] (ty) -- 
        node[midway, below] {$\BrL(y)$}
        (b4);
        \draw[localType, <-] (ty) --
        node[midway, below] {$\BlL(y)$}
        (b3);
    \end{tikzpicture}
    }
    \caption{Partitionning a branch of a tree using a \kl{bough}. To compute
    the presence of an edge between $x$ and $y$ in the resulting graph, 
    it is sufficient to know the values of
    $\tlbl{\t}{b_2}{b_3}$, and the respective \kl{bough types} of $x$ and $y$.}
    \label{partitionning-a-graph:fig}
\end{figure}

\AP In the rest of the paper, we will ofter rely on some tree surgery operation
consisting in replacing a \kl{bough} in a tree $\t$ by some other \kl{bough}.
The \intro{kind of a bough} is given by its level $k$, the value of the first
idempotent edge $\tlbl{\t}{b_0}{b_1}$, and the value of the last idempotent
edge $\tlbl{\t}{b_{n-1}}{b_n}$. The following \cref{fact:bough-replacement}
states that one can safely perform such a replacement without modifying the
part of the graph represented by the tree $\t$ outside of the \kl{bough}.

\begin{fact}
  \label{fact:bough-replacement}
  Let $\t$ be a tree with a \kl{bough} $B$ of level $k$, 
  and let $H$ be another \kl{bough} of level $k$
  with the same \kl{kind} as $B$. Then, 
  one can replace $B$ by $H$ in $\t$ to obtain a new tree $\t'$
  that is also equipped with a \kl{forward Ramseyan split} $\spt'$
  obtained by copying $\spt$ outside of $H$, and using the levels of $H$ inside $H$.

  Furthermore, the subraph of $\someInterp(\t)$ induced by the leaves 
  outside of $B$ is isomorphic (using the identity map) 
  to the subgraph of $\someInterp(\t')$
  induced by the leaves outside of $H$.
\end{fact}

\AP Now that we have an understanding of what happens when replacing a
\kl{bough} in a fixed tree, let us study what happens when we do the opposite
operation, i.e., when we chance the ``context'' in which a \kl{bough} is
placed. 

% TODO:
% - a context is a tree with a bough removed, given by a triple (Troot, Tleft, Tright) where
%   where Troot is the tree above the bough with a distinguished leaf where to plug the bough,
%   and Tleft and Tright are the trees below the bough.
% 
% - the context type of a leaf in a context C is given by the values of 
%   paths from the leaf to the root of Tleft resp Tright


Let us define a \intro{context} $C[\square]$ to be a pair of a tree $A$
with a distinguished hole $\square$ that can be filled by another tree, and a
tree $B$ that can be plugged into this hole. Given a tree $T$ with a hole, one
can place it inside a context $C[\square]$ to obtain a new tree $C[T]$ that is
defined by plugging $T$ into the hole of $A$, and plugging $B$ into the hole of
$T$.


\begin{definition}
    \label{good-bough:def}
    Let $B$ be a \kl{bough} of level $k$ in $T$.
    We say that $B$ is a \intro{good bough} if, for every context 
    there exists a \kl{bough} $H$ of level $k$ in some other tree $T'$,
    and a function $h \colon B(T) \to H(T')$ (from leaves to leaves) such that
    \begin{enumerate}
        \item The subgraph generated by $m B(T)$ 
            is an \kl{induced subgraph} of $H(T')$
            via $h$.
        \item The image of the first block of $B$ by $h$
            is contained in the first block of $H$.
        \item The image of the last block of $B$ by $h$
            is contained in the last block of $H$.
        \item The map $h$ respects the \kl{bough-types} of the leaves.
        \item The map $h$ respects the \kl{idempotent} directly on the left
            and \emph{directly on the right of the leaves}.
        \item There exists a block of $H$ that does not intersect
            the image of $B$ by $h$.
    \end{enumerate}
\end{definition}

\begin{lemma}
    \label{bad-bough-wqo:lem}
    Assume that there are no bounds on the $k$-length of \kl{bad boughs}, then 
    the class of images is not $(3 \times \card{M}^3 + 1)$-well-quasi-ordered.
\end{lemma}
\begin{proof}
    Let $\seqof{B_i}$ be an infinite sequence of bad branches of strictly increasing $k$-length.
    This means that for every $i$ there exists an $m_i \in M$ such that
    no other \kl{bough} $H$ can contain $m B_i(T)$. 
    By extracting a subsequence, we can assume that $m$ is constant.
    Extracting more, we can assume that the $k$-length of $B_i$ is greater than three times the
    number of nodes in $B_{i-1}$.

    Look at the generated graphs, where nodes are labelled by the \kl{bough
    types} of the leaves plus distinguishing colors for the first and last
    elements of the branches if it is defined, and a special color otherwise.
    That is, we have $3 \times \card{M}^3 + 1$ different labels.

    Assume by contradiction that there exists an embedding from $T_i$ to $T_j$
    for some $i < j$. Then, this is also an embedding between $B_i(T_i)$ and
    $B_i(T_j)$, and shows that $B_i(T_i)$ is a \kl{good branch} which is a
    contradiction.
\end{proof}

\AP What \cref{bad-bough-wqo:lem} shows is that the presence of arbitrarily
long ``unbreakable" \kl{boughs} prenvents the class of images from being
\kl{labelled-well-quasi-ordered}. The next step will be to show that provided a
bound on the length of the \kl{bad boughs}, the class of images is
\kl{labelled-well-quasi-ordered}.  To that end, we will 
prove that \kl{good boughs} are in fact easily recognizable
using $\MSO$, hence obtain a decidability result.

\subsection{Good Expansion of A Tree}


In this subsection, we will prove the converse of \cref{bad-bough-wqo:lem},
i.e., we are going to build a suitable embedding provided that the length of
the \kl{bad boughs} is bounded.
To that end, let us first define the \kl{good expansion} of a tree.

\AP Assume that there is a bound $K$ on the length of \kl{bad boughs}, then one
can embed any Ramseyan tree $(T, \spt)$ into a Ramseyan tree $(T',\spt')$ such
that every \kl{$k$-neighbourhood} in $T'$ is \emph{split}. We will therefore
work on trees $(T, \spt, \Delta)$ where $\Delta$ is a subtree of $T$ consisting
of \emph{important nodes}.
We can refine our notion of \kl{gap embedding} to respect important nodes.

\begin{definition}
    The \intro{generalized gap embedding relation}
    between trees.
    It is a map $h \colon T \to T'$ such that
    $h$ is a \kl{gap embedding} between $(T, \spt)$
    and $(T', \spt')$,
    and furthermore if $x \treelt y$ are such that
    $\spt(x) = \spt(y)$ and $\spt(x:y) > k$,
    then $h(x) \treelt h(y)$ are such that
    $\spt'(h(x)) = \spt'(h(y))$ and $\spt'(h(x):h(y)) > k$.
    Furthermore, every node $x$ is labelled 
    with the values of $\tlbl{\t}{z}{x}$ for the first nodes
    $z$ of depth $k$ among ancestors of $x$.
\end{definition}

\begin{lemma}
    If $T$ embeds into $T'$ via the \kl{generalized gap embedding relation}
    then their \kl{marked subgraphs} embed into each other.
\end{lemma}
\begin{proof}
    We prove by induction on $\spt(x:y)$ that
    $\tlbl{\t}{z}{x} = \tlbl{\t'}{h(z)}{h(x)}$ for marked nodes.
    If $\spt(x:y) = \infty$, then the result is immediate.
    If $\spt(x:y) = k$, then
    either $x$ and $y$ are separated by some $z$ that is unmarked
    and the result is immediate, or $x$ and $y$ are only separated by

    \todo[inline]{proof}
\end{proof}


In general, it is not true that the \kl{generalized gap embedding relation}
is a \kl{well-quasi-ordering}. However, we have a bound on the maximal
distance between two marked nodes, and this allows us to prove the following
theorem.

\begin{theorem}
    Let $d$ be a number. The class of all $(T, \spt, \Delta)$
    such that the distance between two marked nodes at level $k$
    is either at most $d$, or they are separated by at least $2$ unmarked nodes
    at level $k$ is \kl{labelled-well-quasi-ordered}.
\end{theorem}

Using this theorem, we conclude this section.

\begin{proofof}{effective-image:thm}[main]
    Test
\end{proofof}


\section{Neighbour Respecting Gap-Embedding}
\label{sec:neighbour-respecting-gap-embedding}

The goal of this subsection is to prove that the gap-embedding relation with a
bounded dependency relation is well-quasi-ordered. Unfortunately, encoding the
dependency inside the usual notion of gap-embedding (the one without
dependencies) is non-trivial. We will instead leverage the recent advances in
inductively defined well-quasi-orderings \cite{FREU20,LOPEZ23}, and the
characterization of the gap-embedding relation as an inductive Kruskal
construction \cite{FREU20} to conclude.

\begin{equation*}
    T_{k+1}^N (X) \defined
    \mu Y. T_{k}^N( \cdots T_k^N(Y) \cdots ) + X
\end{equation*}

And then we use \cite{LOPEZ23} to conclude that it is a WQO + induction 
to prove that it respects the "first-$N$-neighbours".
\begin{theorem}
    \label{good-generalized-gap:thm}
    The gap ordering associated to the ramseyan split of height $XXX$ is a well-quasi-ordering.
\end{theorem}
\begin{proof}
    Trivial. \todo{Not at all!!}
\end{proof}

TODOs:
\begin{itemize}
    \item Define the simon's factorisation theorem for trees.
    \item Check that we can assume that "all products are nice" and not just the ones
        appearing in the decomposition.
    \item Define "branch decompositions" and the "type of nodes" in a branch decomposition.
    \item Define a "bad branch"
    \item Prove that arbitrarily large bad branches violate WQO.
    \item Prove that "good branches" can be embedded into three copies of themselves
    \item Define the "good expansion" of a tree $T$ by expanding all good branches.
    \item Update the tree-decomposition of this good expansion to add \emph{non-sibling}
        idempotent nodes.
    \item Prove that the good expansions of a tree $T$ are well-quasi-ordered using a 
        suitable gap embedding.
    \item Conclude that the class of images is WQO.
\end{itemize}

\section{From images to general classes}

\subsection{Bounded clique width}

\subsection{Decidability using colored boughs}

\begin{lemma}
    \label{good-bough-recognizable:lem}
    Let $T$ be a tree and let $B$ be a \kl{good bough}.
    Then one can assume that $H$ is three copies of $B$.
\end{lemma}
\begin{proof}
    Let $H$ together with $h$ be a witness that $B$ is a \kl{good bough}.
    Let us define the following embedding $k \colon B \to BBB$.
    For every leaf $x$ in $B$, if $h(x)$ is \emph{before} the untouched graph
    of $H$, then $k(x) = (x,1)$ the first copy of $B$.
    Otherwise, $k(x) = (x,3)$ the third copy of $B$.

    Let us check that $k$ respects the required properties.
    First, $k$ respects the \kl{bough-types} of the leaves by construction.
    Second, $k$ respects the \kl{idempotent} directly on the left and right
    by construction too.
    Third $k$ respects the last block and first-block conditions, because $h$
    did.

    Finally, the only thing left to check is that $(x,y)$ is an edge in the
    graph $B$ if and only if $(k(x), k(y))$ is one in $BBB$.
    If $x$ and $y$ are sent on the same copy, there is nothing to be done.
    If $x$ and $y$ are separated, then
    they were already separated in $H$.
    As a consequence, the presence of an edge between $x$ and $y$ in $B$ is
    equivalent to the presence of an edge between $h(x)$ and $h(y)$,
    but these introduce an idempotent node between them,
    and therefore the same thing holds between $k(x)$ and $k(y)$: indeed,
    the inserted idempotent node \emph{must have the same value}
    because $h$ preserved this!
\end{proof}

As a consequence, checking whether a \kl{bough} is \kl{good} is $\MSO$
definable, one simply has to guess a partition of $B$ into two different
copies.

\begin{lemma}
    \label{bad-boughs-decidable:lem}
    One can decide whether there exists unbounded
    \kl{bad boughs}.
\end{lemma}
\begin{proof}
    We consider the automaton that reads from a decorated tree
    the bough $B$, the partition, the type of the leaves. 
    It first checks that the decoration is correct, and then
    guesses whether the decoration would still be good when
    mapping every leaf of the bough to either a "left" or a "right"
    duplicate of this branch. This is an $\MSO$-definable property.
    It suffices to check that the corresponding automaton
    has an unbounded language, which is decidable.
\end{proof}



\subsection{The special case of hereditary classes}


\begin{corollary}
  The conjecture of regular antichains of 
  \cite{ALM17} holds for classes of bounded clique-width!
\end{corollary}



\begin{theorem}[restate=effective-image:thm,label={effective-image:thm}]
    \label{effective-image:thm}
    Let $I$ be an $\MSO$ interpretation
    from finite trees to undirected graphs.
    There exists a computable $k \in \Nat$
    such that $\image{I}$
    is $k$-well-quasi-ordered
    if and only if 
    $\image{I}$ is wqo-well-quasi-ordered.
    These properties are furthermore decidable.
\end{theorem}

From this theorem, we immediately provide a positive answer to
\cref{pouzet2:conj} for classes of graphs of \kl{bounded clique-width}.

\begin{corollary}
    \label{effective-image:cor}
    A class of graphs $\Cls$ of \kl{bounded clique width} 
    is \kl{labelled-well-quasi-ordered}
    if and only if it is \kl{wqo-well-quasi-ordered}.
\end{corollary}
\begin{proof}
    Consider a class $\Cls$ of graphs having bounded clique width.
    There is an interpretation from trees to a superset of this class $\Cls$.
    Because $\Cls$ is \kl{labelled-well-quasi-ordered}, the class of images
    can be described using finitely many forbidden patterns,
    and we can assume that we do not take subsets.
    As a consequence, we can assume that there exists a $\varphi$
    such that $\Cls \subseteq \varphi(\Trees{\Sigma}) \subseteq \dwset{\Cls}$.
    Now, because $\Cls$ is \kl{labelled-well-quasi-ordered}, we can assume that
    $\dwset{\Cls}$ is \kl{labelled-well-quasi-ordered} as well.
    This proves that $\varphi(\Trees{\Sigma})$ is \kl{labelled-well-quasi-ordered},
    hence \kl{wqo-well-quasi-ordered},
    and as a consequence so is $\Cls$.
\end{proof}

Before proving \cref{effective-image:thm}, let us focus our attention on
\intro{simple interpretations}, i.e., those that are defined solely by a
formula $\varphi_E(x,y)$.
This is not a restriction, as the following lemma
shows.

\begin{lemma}
    \label{simple-interpretation:lem}
    Let $I$ be an $\MSO$ interpretation from finite trees to undirected graphs.
    There exists a (computable) simple interpretation $I'$ such that
    for all $k \geq 1$,
    $\image{I}$ is \kl{$k$-well-quasi-ordered} if and only if
    $\image{I'}$ is \kl{$k$-well-quasi-ordered}.
\end{lemma}
\begin{proof}
\end{proof}

\AP Therefore, the main combinatorial obstacle to understanding whether or not
a class is \kl{labelled-well-quasi-ordered} is to understand the behaviour of
pairs of nodes $x$ and $y$ that are \kl{dependent}. This is the motivation
for the following construction that studies the behaviour of
consecutive \kl{$k$-neighbourhoods} in a tree.
\subsection{Gap Embedding(s)}

In order to compare two trees equipped with \kl{forward Ramseyan splits}, we
will design an ordering that respects the tree structure as well as the splits.
This is usually called the \intro{gap embedding} relation, originally defined
by Dershowitz and Tzameret \cite{DERSHOWITZ200380}. It was noticed by Freund
\cite{FREU20} that this ordering can be understood as a nested version of
Kruskal's Tree Theorem, and this will be our point of view in the following,
since we will need to slightly adapt the definition to our setting.

\AP Let us consider some $k \in \set{1, \dots, N}$. We will write
$\Trees{\Sigma,k}{X}(Y)$ for the class of trees with labels in $X \times \set{k,
\ldots, N}$ on the internal nodes, labels in $\Sigma$ on the edges, with a
distinguished root, and such that leaves are labelled in $Y$.
It is immediate that $\Trees{\Sigma,k-1}{X}(Y)$ 
is the least fixed point of the following operator:
\begin{equation*}
  F \colon Z \mapsto Y + \Trees{\Sigma,k}{X}(Z) \quad .
\end{equation*}

It was shown in \cite{LOPEZ23} that any such inductive definition gives 
rise to a \kl{well-quasi-order}, obtained by 





Hence, we are in the opposite situation as with the \kl{composition ordering}:
we have a \kl{well-quasi-order} on trees, but we do not know whether the
interpretation $\someInterp$ is order-preserving from trees (with a choice of
split) to graphs. However, when inserting new nodes, we are often able to state
that the value of $\tlbl{\t}{x}{y}$ is preserved provided that $x$ and $y$ are
sufficiently far apart in the \kl{forward Ramseyan split} of the tree, as
stated in the following lemma.

Problems may arise when $x$ and $y$ are too close in the \kl{forward Ramseyan
split}, for instance because they are \kl{one-separated at level $k$}, or
because one inserts new nodes in the first and last part of the path from $x$
to $y$. A typical example of such a situation is illustrated in the case of a
monoid $M$ with two elements $\set{a, b}$ such that $a^2 = b$ and $b$ is
absorbing. Then, replacing an edge $x \treelt y$ labelled with $a$ by two or more
edges will change the value of $\tlbl{\t}{x}{y}$ from $a$ to $b$.
