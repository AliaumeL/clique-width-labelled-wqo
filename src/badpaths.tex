\clearpage
\section{Bad Patterns in Images of Interpretations}
\label{sec:bad-patterns}

\maelin{Commentaire général sur la section, j'ai vraiment du mal à bien saisir ce qu'est la notion de bough. J'ai laissé des commentaires là où ça me posait des problème, mais de manière générale il faudrait réussir à mieux introduire l'objet, à expliquer ce qu'il est.}

In this section, we will isolate combinatorial obstructions to being
\kl{2-well-quasi-ordered} in classes of graphs that are images of \kl{simple
MSO interpretations} from trees. We will take the same notations as in
\cref{sec:ramseyan}. 


% use a large two column figure here to illustrate the various notions
% we are using sigcomp latex class in two columns where the command
% \begin{figure*} ... \end{figure*} creates a figure spanning both columns
\begin{figure*}[htbp]
    \centering
    \begin{tikzpicture}[
        localType/.style={
            color=C3,
            thick
        },
        branchProj/.style = {
            color=Prune,
            inner sep=0pt,
            minimum size=4pt,
            fill,
            circle
        },
        leaf/.style={
            color=Prune,
        },
        ]
        \draw (0,0) rectangle (8,2);
        \node (root) at (-1.2,2) {$\treeRoot$};
        \foreach \x in {0,1,2,3,4} {
            \coordinate (n\x) at ({ 2 * \x},0);
            \coordinate (pb\x) at ({ 2 * \x},2);
            \node[branchProj] (b\x) at (pb\x) {};
            \node (lb\x) at ({ 2 * \x},2.4) {$b_{\x}$};
            \draw (n\x) -- (b\x);
        }
        \draw[dashed,<-] (b0) -- (root);
        \draw[dashed] (b4) -- (9,2.5);
        \draw[dashed] (b4) -- (9,1.5);

        \foreach[count=\x] \y in {0,1,2,3} {
            \node (E\x) at ({ 2 * \x - 1},2.6) {$e_{\y}$};
            \draw[->,thick] (b\y) to[bend left=40] (b\x);
            \node (T\x) at ({ 2 * \x - 1},-0.6) {$\BBlock[\t]{b_{\y}}{b_{\x}}$};
        }
        
        \node (x) at (3,0.2)  {$x$};
        \node (y) at (7,0.2)  {$y$};
        \node[branchProj] (tx) at (3,2) {};
        \node[branchProj] (ty) at (7,2) {};

        \draw[localType, <-] (x)  -- 
        node[midway, left] {$\BtL(x)$} (tx);
        (tx);
        \draw[localType, ->] (tx) -- 
        node[midway, below] {$\BrL(x)$}
        (b2);
        \draw[localType, <-] (tx) -- 
        node[midway, below] {$\BlL(x)$}
        (b1);

        \draw[localType, <-] (y)  -- 
        node[midway, left] {$\BtL(y)$} 
        (ty);
        \draw[localType, ->] (ty) -- 
        node[midway, below] {$\BrL(y)$}
        (b4);
        \draw[localType, <-] (ty) --
        node[midway, below] {$\BlL(y)$}
        (b3);
    \end{tikzpicture}
    \caption{Partitionning a branch of a tree using a \kl{bough}. To compute
    the presence of an edge between $x$ and $y$ in the resulting graph, 
    it is sufficient to know the values of
    $\tlbl{\t}{b_2}{b_3}$, and the respective \kl{bough types} of $x$ and $y$.}
    \label{partitionning-a-graph:fig}
\end{figure*}

\begin{definition}
    \label{ramseyan-branch:def}
    Let $\aTree$ be a tree and $\spt$ be a \kl{forward Ramseyan split} of height $N$.
    A \intro{bough of level $k$} in $\aTree$ is defined from a part of a $k$-neighbourhood as follows:
    Consider a sequence $b_0, b_1, \ldots, b_n$ of nodes of level $k$ on a branch of $\aTree$
    forming a \kl{$k$-neighbourhood}.
    Elements of the \reintro{bough} $B$ are all nodes $x$ in $\aTree$ such that 
    $b_0 \treeleq x$, and $\neg (b_n \treeleq x)$, plus $b_n$ itself.
    
    We say that the \intro(bough){dimension} of the \kl{bough} $B$ is $n+1$.
\end{definition}

\AP Let $B$ be a \kl{bough of level $k$} in $\aTree$.
Given two nodes $b_1 \treelt b_2$ in $B$ such that $\spt(b_1) = \spt(b_2) = k$, we define the $\BBlock[\aTree]{b_1}{b_2}$
as the set of nodes $x$ in $\aTree$ such that $b_1 \treeleq x$ and $\neg (b_2
\treeleq x)$. We also refer to
\cref{partitionning-a-graph:fig} for an illustration of the resulting partition
of the tree $\aTree$ with respect to a given \kl{bough} $B$.

\AP
For every leaf $x \in \BBlock[\aTree]{b_0}{b_n}$ where $n$ is the
\kl{dimension of the bough} $B$, we define $\intro*\Bt(x)$ to be the least ancestor of
$x$ that belongs to $B$. Similarly, we define $\intro*\Bl(x)$ to be the least element
of $B$ that is greater or equal to $\intro*\Bt(x)$, and $\intro*\Br(x)$ to be the greatest
element of $B$ that is less or equal to $\intro*\Bt(x)$. To every leaf $x$ of $B(\aTree)$,
we can therefore associate the following values in $M$:
\begin{align*}
  \intro*\BtL(x) &\defined \tlbl{\t}{\Bt(x)}{x}, \\
  \intro*\BlL(x) &\defined \tlbl{\aTree}{\Bl(x)}{\Bt(x)}, \\
  \intro*\BrL(x) &\defined \tlbl{\t}{\Bt(x)}{\Br(x)}, \\
  \intro*\BrootL(x) &\defined \tlbl{\aTree}{\treeRoot}{\Bl(x)}.
\end{align*}
We call this tuple 
the \intro{bough type} of the leaf $x$ with respect to the \kl{bough} $B$.
We again refer to 
\cref{partitionning-a-graph:fig}
for an illustration of the type of a leaf
with respect to a given \kl{bough} $B$. 


\AP In the rest of the paper, we will ofter rely on some tree surgery operation
consisting in replacing a \kl{bough} in a tree by some other \kl{bough}. Given
a \kl{bough} $B$ in a tree $\t$ defined using a \kl{$k$-neighbourhood}
$b_0$, $b_1$, \ldots, $b_n$,
we write $\t = C[B]$ to denote that $\t$ can
be obtained by plugging the \kl{bough} $B$ into a \intro(bough){context} $C[\square]$.
Formally, a \reintro(bough){context} $C[\square]$ is given by a tree $\t_{\mathsf{root}}$
with a distinguished leaf $\square$, together with two trees
$\t_{\mathsf{left}}$ and $\t_{\mathsf{right}}$. These three elements are
reprenented as dashed lines in \cref{partitionning-a-graph:fig}.
The tree $C[B]$ is then obtained as follows: one first attaches 
$\t_{\mathsf{left}}$ as the left subtree of $b_n$,
where $b_n$ is the last element of the \kl{$k$-neighbourhood} defining $B$,
then one attaches $\t_{\mathsf{right}}$ as the right subtree of $b_n$, and finally,
one replaces the distinguished leaf $\square$ in $\t_{\mathsf{root}}$ by the
node $b_0$, the first element of the \kl{$k$-neighbourhood} defining $B$.

\AP Two boughs $B$ and $B'$ are \intro{compatible} if they start with the same
idempotent value, and if for every tree $\aTree = C[B]$, the tree $C[B']$ is
also well-defined (i.e., it remains \kl{forward Ramseyan}). The following
\cref{fact:bough-replacement} states that one can safely perform such a
replacement without modifying the part of the graph represented by the tree
$\t$ outside of the \kl{bough}.

\maelin{Je suis confus sur ce qu'est un bough, c'est une chemin ? un sous-arbre ? la façon dont tu le défini pour moi c'est simplement un chemin, sauf que j'ai l'impression que ça devrait être un arbre (genre un chemin avec des sous-arbre pendant).}

\begin{fact}
  \label{fact:bough-replacement}
  Let $\t = C[B]$ be a tree with a \kl{bough} $B$ of level $k$, 
  and let $H$ be a \kl{bough} that is \kl{compatible}
  with $B$. Let $\t' = C[H]$ be the tree obtained by replacing $B$ by $H$ in $\t$.
  Then, the subraph of $\someInterp(\t)$ induced by the leaves 
  outside of $B$ is isomorphic (using the identity map) 
  to the subgraph of $\someInterp(\t')$
  induced by the leaves outside of $H$.
\end{fact}



\begin{definition}
    \label{good-bough:def}
    Let $B$ be a \kl{bough} of level $k$ in a tree $\t$.
    We say that $B$ is a \intro{good bough} if, 
    there exists a \kl{compatible} \kl{bough} $H$ of level $k$,
    and a map $h \colon \someInterp(C[B]) \to \someInterp(C[H])$
    such that:
    \begin{enumerate}
        \item $h$ is an embedding of graphs,
        \item $h$ is the identity map on leaves belonging $C[\square]$,
        \item there exists a \kl{block} in $H$ that is left untouched 
          by $h$.
    \end{enumerate}
    A \intro{bad bough} is a \kl{bough} that is not a \kl{good bough}.
\end{definition}

We claim that the existence of \kl{bad boughs} of arbitrarily large
\kl{dimension} is the only obstruction to being \kl{2-well-quasi-ordered}.

\begin{lemma}
  \label{lem:good-boughs-wqo}
  The image of $\someInterp$ is \kl{$2$-well-quasi-ordered} 
  if and only if
  the image of $\someInterp$ is \kl{$\forall$-well-quasi-ordered} 
  if and only if 
  there exists a bound on the \kl{dimension} of \kl{bad boughs}.
\end{lemma}

The proof of \cref{lem:good-boughs-wqo} is split into two parts, one will
leverage the bound on the \kl{dimension} of \kl{bad boughs} to construct a
suitable tree representation of the graphs proving that they are
\kl{$\forall$-well-quasi-ordering} (\cref{sec:gap-embedding}). On the other
hand, we will show that the existence of \kl{bad boughs} of arbitrarily large
\kl{dimension} allows to construct an infinite two-labelled antichain in the
image of $\someInterp$ (\cref{sec:obstructions}).


\subsection{Using the Gap Embedding}
\label{sec:gap-embedding}

\AP If there is an upper bound $N_0$ on the \kl{dimension} of \kl{bad boughs}, we can
use \cref{fact:bough-replacement} to upgrade our trees in the following way:
given a tree $\t$, we will insert dummy nodes in every \kl{bough} of level $k$
that has \kl{dimension} greater than $N_0$. The resulting tree $\t'$ will be a
\kl{marked nested tree} as defined in \cref{sec:ramseyan}, with three
kinds of nodes: \kl{marked nodes}, that correspond to nodes of $\t$,
\kl{dummy nodes} that were inserted during this procedure,
and \kl{separating nodes}, that are the last nodes of the \kl{$k$-neighbourhoods}
defining the \kl{boughs} we manipulated.
The following 
\cref{lem:important-nodes} shows that this construction produces
\kl{well-marked nested trees} that are \kl{$N_0$-bounded}, allowing us to apply
\cref{thm:marked-nested-trees-wqo}.

\begin{lemma}
    \label{lem:important-nodes}
    Assume that there exists a bound $N_0$ on the \kl{dimension} of \kl{bad
    boughs}. Then, for every tree $\t$ equipped with a \kl{forward Ramseyan split} $\spt$, 
    one can effectively construct a \kl{well-marked nested tree} $(\t', \spt', \marking)$ 
    that is \kl{$N_0$-bounded} and such that
    the restriction of $\someInterp(\t')$ to \kl{marked leaves}
    is isomorphic to $\someInterp(\t)$.
\end{lemma}
\begin{proof}
  We proceed top-down in the tree $\t$. At the beginning, we define 
  every node of $\t$ as \kl(nodes){marked}.

  Whenever we encounter a \kl{bough} $B$ of level $k$ with dimension greater
  than $N_0$, we first consider its prefix of \kl{dimension} $N_0 + 1$. 
  This prefix is a \kl{good bough} by assumption on the bound $N_0$,
  hence there exists a \kl{compatible} \kl{bough} $H$ of level $k$
  and one can replace $B$ by $H$ in $\t$. This operation does not 
  modify the graph outside of the \kl{bough} by \cref{fact:bough-replacement}.
  Furthermore, by definition of \kl{good boughs}, the graph induced 
  by $B$ is an induced subgraph of the graph induced by $H$.
  We update the marking accordingly so that leaves of $B$ that were marked 
  remain marked in $H$, untouched leaves of $H$ become \kl{dummy},
  and we mark the last element of the $k$-neighbourhood defining $H$
  as a \kl{separating node}.
  Then, we complete
  the marking by ensuring that the least common ancestor
  of two \kl{marked nodes} is also \kl{marked}. By definition, the graph
  induced by the \kl{marked leaves} remains unchanged by this operation.
  
  To verify that $(\t', \spt', \marking)$ is \kl{well-marked}, note that by construction,
  whenever we have \kl{marked nodes} $x$ and $y$ with a non-\kl{marked} node in between
  at level $k$, this configuration arises from a \kl{bough} replacement where
  the required witnesses $z_1, z_2, z_3$ exist by the definition of \kl{good boughs}.
  Moreover, the tree is \kl{$N_0$-bounded} because every maximal path of non-\kl{dummy}
  nodes has length at most $N_0$ by a simple induction on the length.
\end{proof}

\AP By \cref{lem:important-nodes}, we can transform any tree into a \kl{$N_0$-bounded}
\kl{well-marked nested tree}. By \cref{thm:marked-nested-trees-wqo}, the class of
such trees is \kl{well-quasi-ordered} under the \kl{gap-embedding} ordering. Furthermore,
by \cref{cor:gap-embedding-monotone}, the \kl{gap-embedding} ordering between
\kl{marked nested trees} preserves the interpretation: if $(\t_1, \spt_1, \marking_1)
\gemb (\t_2, \spt_2, \marking_2)$, then the graph interpreted from
$(\t_1, \spt_1, \marking_1)$ (restricted to \kl{marked leaves}) is an induced
subgraph of the graph interpreted from $(\t_2, \spt_2, \marking_2)$ (restricted to
\kl{marked leaves}). Since the interpretation restricted to \kl{marked leaves}
coincides with the original interpretation on $\t_1$ and $\t_2$ by \cref{lem:important-nodes},
this establishes the desired order-preserving property.

\subsection{Obstructions}
\label{sec:obstructions}

\AP In this section, we will assume that there 
exist \kl{bad boughs} of arbitrarily large \kl{dimension}, and we will
leverage this to construct an infinite \kl{antichain} in the image of
$\someInterp$.

Our first remark is that there are only finitely many equivalence 
classes for \kl{boughs} of a given level $k$ with respect to \kl{compatibility}.
This is the content of the following \cref{lem:finitely-many-boughs}.


\begin{lemma}
  \label{lem:finitely-many-boughs}
  Let $k$ be a level. There exists a finite set of \kl{boughs} of level $k$
  such that every \kl{bough} of level $k$ is \kl{compatible} with one of them.
\end{lemma}
\begin{proof}
  Sketch: Compatibility between two \kl{boughs} $B$ and $B'$ of level $k$ 
  is determined by the shape of the beginning and ends of the splits of $B$ and $B'$,
  and there are finitely many such shapes.
\end{proof}

\begin{lemma}
  \label{lem:finitely-many-contexts}
  There exists a finite set $\mathcal{F}$ of contexts $C[\square]$ such that
  for every bad bough $B$ of level $k$, there exists a context
  $C[\square] \in \mathcal{F}$ such that $B$ is a \kl{bad bough} in $C[B]$.
\end{lemma}
\begin{proof}
  Sketch: the context is not very important in good boughs
  because outside it is the identity map. Furthermore, the behaviour
  of nodes in the context with respect to those of the bough 
  is determined by a finite number of monoid elements.
  Hence, if a bough is good in some context, it is good in any context with 
  the same behaviour and there are finitely many such contexts.
\end{proof}


\begin{lemma}
  \label{lem:bad-bough-antichain}
  If there exist \kl{bad boughs} of arbitrarily large \kl{dimension},
  then there exists an infinite two-colored \kl{antichain} in the image of $\someInterp$.
\end{lemma}
\begin{proof}
  Assume towards a contradiction that the class of graphs 
  obtained as images of $\someInterp$ is \kl{$2$-well-quasi-ordered}.
  Let $\seqof{B_i}[i \geq 1]$ be an infinite sequence of \kl{bad boughs}. Without loss 
  of generality, one can assume that all \kl{boughs} $B_i$ have the same level $k$,
  and are pairwise \kl{compatible}, by \cref{lem:finitely-many-boughs}.

  Now, using \cref{lem:finitely-many-contexts}, one can extract one context 
  $C[\square]$ such that infinitely many \kl{boughs} $B_i$ are \kl{bad boughs}
  in $C[B_i]$. Finally, one can extract further to assume that 
  the \kl{dimension} of the \kl{boughs} $B_{i+1}$ 
  is greater than twice the number of leaves in $B_i$.

  Let us now color every graph $\someInterp(C[B_i])$ with two labels: $\top$
  to leaves belonging to $B_i$, and color $\bot$ to leaves belonging to
  $C[\square]$. Because we assume that the image of $\someInterp$ is
  \kl{$2$-well-quasi-ordered}, we can extract an infinite increasing
  subsequence, and assume that for all $i < j$, there exists an embedding $f_{i,j}
  \colon \someInterp(C[B_i]) \to \someInterp(C[B_j])$ that preserves labels.
  Let us remark that for all $i < j$, the map 
  $f_{i,j}$ acts as a permutation when restricted to nodes labelled $\bot$,
  that is of size the number of leaves in $C[\square]$.
  By a Ramsey argument, we can therefore assume that this permutation is the identity
  for all $i < j$.
  Finally, consider any two indices $i < j$. 

  Because the \kl{dimension} of $B_j$ is greater than twice the number of leaves
  in $B_i$, there must be some \kl{block} in $B_j$ that is left untouched by $f_{i,j}$.
  Furthermore, we know that $f_{i,j}$ is the identity on leaves
  belonging to $C[\square]$. Hence, the map $f_{i,j}$
  witnesses that $B_i$ is a \kl{good bough} in $C[B_i]$, which is a contradiction.
\end{proof}

\clearpage
\section{Bounded Clique Width}
\label{sec:hereditary-classes}


In this section, we will aim at leveraging the previous results 
to tackle classes of \kl{bounded clique-width}. Our first goal is to 
prove our \cref{main:theorem} regarding the 
images of finite trees under \kl{MSO interpretations}. There are two difficulties here:
first, we only dealt with \kl{simple MSO interpretations} so far, and second,
we have not yet given any decision procedure to check whether the image of
a given \kl{simple MSO interpretation} is \kl{2-well-quasi-ordered}. Let us first
tackle the second difficulty.

\begin{definition}[Perfect Bough]
  \label{def:perfect-bough}
  Let $\aTree = C[B]$ be a tree with a \kl{bough} $B$ of level $k$.
  We say that $B$ is a \intro{perfect bough} there exists a compatible
  \kl{bough} $H$ of level $k$ and 
  a map $h \colon \someInterp(C[B]) \to \someInterp(C[H])$
  \begin{itemize}
      \item $h$ is an embedding of graphs,
      \item $h$ is the identity map on leaves belonging $C[\square]$
      \item the \kl{bough type} of every leaf $x$ in $H$ is the same as 
        the \kl{bough type} of $h(x)$ in $B$.
      \item there exists a \kl{block} in $H$ that is left untouched 
          by $h$.
  \end{itemize}
  An \intro{imperfect bough} is a \kl{bough} that is not a \kl{perfect bough}.
\end{definition}

\begin{lemma}
  \label{lem:perfect-boughs-wqo}
  The image of $\someInterp$ is \kl{$2$-well-quasi-ordered} 
  if and only if
  the image of $\someInterp$ is \kl{$\forall$-well-quasi-ordered} 
  if and only if 
  there exists a bound on the \kl{dimension} of \kl{imperfect boughs}.
\end{lemma}
\begin{proof}
  First, if we have a bound on the \kl{dimension} of \kl{imperfect boughs},
  then we also have a bound on the \kl{dimension} of \kl{bad boughs},
  because every \kl{bad bough} is an \kl{imperfect bough}.
  Hence, by \cref{lem:good-boughs-wqo}, the image of $\someInterp$ is
  \kl{$\forall$-well-quasi-ordered}.

  Conversely, assume that there exist \kl{imperfect boughs} of arbitrarily
  large \kl{dimension}. Then, by an argument similar to the one of 
  \cref{lem:bad-bough-antichain}, we can construct an infinite
  labelled \kl{antichain} in the image of $\someInterp$. The only change is that 
  one also adds the \kl{bough type} of the leaves as additional labels.
\end{proof}

The important advantage of \cref{lem:perfect-boughs-wqo} over
\cref{lem:good-boughs-wqo} is that \kl{imperfect boughs} are easily
recognizable.

\begin{lemma}
  \label{lem:recognizing-perfect-boughs}
  There exists an $\MSO$ formula that recognizes \kl{imperfect boughs}.
\end{lemma}
\begin{proof}[Proof Sketch]
  The proof follows the same pattern as 
  \cite[Lemma 19]{LOPEZ24}. The main idea is that if a \kl{bough} $B$ is
  \kl{perfect}, then the only thing that matters for a given leaf $x$ in $B$
  is its \kl{bough type} and whether it is mapped on the left or right of the
  ``untouched block'' in the \kl{bough} $H$. Hence, one can guess for every leaf 
  in $B$ where it is mapped (left or right of the untouched block),
  and verify that the mapping preserves edges using only \kl{MSO} formulas.
  If such a mapping exists, then one can reconstruct a bough $H$
  by triplicating $B$, that is, $H = B B B$, and mapping leaves 
  to their corresponding copy in $H$. The middle copy will be untouched, 
  witnessing that $B$ is a \kl{perfect bough}.
  This reasoning was made formal for \kl{linear clique-width} in
  \cite[Lemma 16]{LOPEZ24}, and the exact same proof works here.
\end{proof}

\begin{corollary}
  One can decide whether there are \kl{imperfect boughs} of arbitrarily large
  \kl{dimension}.
\end{corollary}
\begin{proof}
  One can build a cost-mso formula on trees that outputs $0$ if the tree 
  is not an \kl{imperfect bough}, and outputs its \kl{dimension} otherwise.
  By \cite{COLOD10}, one can decide whether this cost-mso formula is
  bounded on the class of finite trees, hence the result.
\end{proof}


We are now ready to prove our \cref{main:theorem}.
\begin{proof}
  Let $\someInterp$ be an \kl{MSO interpretation}.
  We can get rid of the domain formula and selection formula 
  by standard arguments without changing the \kl{2-well-quasi-ordering}
  status of the image of $\someInterp$ (see \cite{LOPEZ24}).
  Hence, we can assume that $\someInterp$ is a \kl{simple MSO interpretation}.
  Furthermore, the transformation from an \kl{simple MSO interpretation}
  to a \kl{monoid interpretation} from trees
  described in \cref{sec:ramseyan} is effective.
  Finally, by \cref{lem:perfect-boughs-wqo}
  the image of $\someInterp$ is \kl{$2$-well-quasi-ordered}
  if and only if it is \kl{$\forall$-well-quasi-ordered}, and 
  by \cref{lem:recognizing-perfect-boughs},
  one can decide whether the image of $\someInterp$ is \kl{$2$-well-quasi-ordered}. 

  It remains to discuss the fact that one can add a total ordering on the graphs 
  without changing the \kl{2-well-quasi-ordering} status of the image of $\someInterp$.
  This is because we prove that the image of $\someInterp$ is \kl{$2$-well-quasi-ordered}
  using the \kl{marked gap-embedding} ordering on \kl{marked nested trees}
  representing the graphs, which already includes a total ordering on the leaves
  (the topological ordering).
\end{proof}

Let now turn our attention to hereditary classes to obtain our 
\cref{main:corollary}.
\begin{proof}
  By a standard argument, if the class $\someInterp$ is hereditary
  and \kl{$2$-well-quasi-ordered}, then it is defined by finitely many
  forbidden induced subgraphs (see \cite{LOPEZ24} for details).
  Hence, if $\Cls$ has \kl{bounded clique-width} and is hereditary,
  one can assume that $\Cls$ is the image of an \kl{MSO interpretation}
  $\someInterp$ from finite trees, and the result follows from 
  \cref{main:theorem}.
\end{proof}


Let us now turn our attetion to \cref{thm:characterisations},
and in particular the link between \kl{2-well-quasi-ordering} and
\kl{bounded linear clique-width}. The following lemma shows that
the existence of \kl{imperfect boughs} of arbitrarily large \kl{dimension}
can be witnessed using only \kl{blocks} from a finite set.

\begin{lemma}
  \label{lem:hereditary-perfect-boughs}
  Assume that there are \kl{imperfect boughs} of arbitrarily large \kl{dimension}.
  Then, there exists a finite set of blocks $\mathbb{F}$ such that 
  one can build an \kl{imperfect bough} of arbitrarily large \kl{dimension}
  using only \kl{blocks} from $\mathbb{F}$.
\end{lemma}

As an immediate consequence of \cref{lem:perfect-boughs-wqo} and
\cref{lem:hereditary-perfect-boughs}, we see that 

\begin{corollary}
  \label{cor:hereditary-perfect-boughs-wqo}
  Assume that the class $\someInterp$ is hereditary.
  Then, the image of $\someInterp$ is \kl{$2$-well-quasi-ordered}
  if and only if 
  every class of \kl{bounded linear clique-width} in $\someInterp$
  is \kl{$2$-well-quasi-ordered}.
\end{corollary}

