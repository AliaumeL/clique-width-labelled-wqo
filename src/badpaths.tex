\clearpage
\section{Bad Patterns in Images of Interpretations}
\label{sec:bad-patterns}

\maelin{Commentaire général sur la section, j'ai vraiment du mal à bien saisir ce qu'est la notion de bough. J'ai laissé des commentaires là où ça me posait des problème, mais de manière générale il faudrait réussir à mieux introduire l'objet, à expliquer ce qu'il est.}

In this section, we will isolate combinatorial obstructions to being
\kl{2-well-quasi-ordered} in classes of graphs that are images of \kl{simple
MSO interpretations} from trees. We will take the same notations as in
\cref{sec:ramseyan}. 

\begin{definition}
    \label{ramseyan-branch:def}
    Let $\aTree$ be a tree and $\spt$ be a \kl{forward Ramseyan split} of height $N$.
    A \intro{bough of level $k$} in $\aTree$ is an infix of a branch of $\aTree$
    such that its maximal and minimal elements have level $k$,
    and such that 
    all elements of the bough have level greater or equal to $k$.

    The \intro{dimension of a bough} is the number of elements of level $k$
    in the \kl{bough}.
\end{definition}

\AP Let $B$ be a \kl{bough of level $k$} in $\aTree$. Given two nodes $b_1 \treelt b_2$ in $B$ such that $\spt(b_1) = \spt(b_2) = k$, we define the $\aTree_{b_1:b_2}$
as the set of nodes $x$ in $\aTree$ such that $b_1 \treeleq x$ and $\neg (b_2
\treeleq x)$. For every leaf $x \in \aTree_{b_0: b_n}$ where $n$ is the
\kl{dimension of the bough} $B$, we define $\Bt(x)$ to be the least ancestor of
$x$ that belongs to $B$. Similarly, we define $\Bl(x)$ to be the least element
of $B$ that is greater or equal to $\Bt(x)$, and $\Br(x)$ to be the greatest
element of $B$ that is less or equal to $\Bt(x)$. To every leaf $x$ of $B(\aTree)$,
we can therefore associate the following values in $M$:
$\BtL(x) \defined
\tlbl{\t}{\Bt(x)}{x}$, $\BlL(x) \defined \tlbl{\aTree}{\Bl(x)}{\Bt(x)}$,
$\BrL(x) \defined \tlbl{\t}{\Bt(x)}{\Br(x)}$,
and $\BrootL(x) \defined \tlbl{\aTree}{\treeRoot}{\Bl(x)}$. We call this tuple 
the \intro{bough type} of the leaf $x$ with respect to the \kl{bough} $B$.
We refer to
\cref{type-of-a-leaf-in-branch:fig} for an illustration of the type of a leaf
with respect to a given \kl{bough} $B$. We also refer to
\cref{partitionning-a-graph:fig} for an illustration of the resulting partition
of the tree $\aTree$ with respect to a given \kl{bough} $B$.

\begin{figure}
    \centering
    \begin{tikzpicture}[
        branch/.style={
            color=Prune,
            inner sep=0pt,
            minimum size=4pt,
            fill,
            circle
        },
        inner/.style={
            color=A1,
            inner sep=0pt,
            minimum size=4pt,
            draw,
            circle
        },
        staredge/.style={
            color=A1,
            ->,
            dashed
        },
        root/.style={
            color=Prune,
        },
        leaf/.style={
            color=Prune,
        },
        monoid/.style={
            color=A2
        },
        ]
        % first draw the branch 
        \node[root]   (root) at (-2,0) {$\treeRoot$};
        \node[branch] (b0) at (0,0) {};
        \node[branch] (b1) at (4,0) {};
        \node[inner]  (t)  at (2,0) {};
        \node[leaf]   (x)  at (3.5,-2) {$x$};

        \node[above=0.1cm of b0] {$\Bl(x)$};
        \node[above=0.1cm of t]  {$\Bt(x)$};
        \node[above=0.1cm of b1] {$\Br(x)$};

        \draw[staredge] (root) -- (b0);
        \draw[staredge] (b0) -- node[monoid, midway, below] {$\BlL(x)$} (t);
        \draw[staredge] (t)  -- node[monoid, midway, below] {$\BrL(x)$} (b1);
        \draw[staredge] (b1) -- (5,0);

        \draw[staredge] (t)  -- node[monoid, midway, below left] {$\BtL(x)$} (x);
    \end{tikzpicture}
    \caption{The type of a leaf with respect to a given \kl{bough} $B$,
    here $\Bl(x)$ and $\Br(x)$ are elements of level $k$ in the \kl{bough},
    and $\Bt(x)$ is the least ancestor of $x$ in the \kl{bough}.}
    \label{type-of-a-leaf-in-branch:fig}
\end{figure}

\begin{figure}
    \centering
    \resizebox{0.9\linewidth}{!}{
    \begin{tikzpicture}[
        localType/.style={
            color=C3,
            thick
        },
        branchProj/.style = {
            color=Prune,
            inner sep=0pt,
            minimum size=4pt,
            fill,
            circle
        },
        leaf/.style={
            color=Prune,
        },
        ]
        \draw (0,0) rectangle (8,2);
        \node (root) at (-1.2,2) {$\treeRoot$};
        \foreach \x in {0,1,2,3,4} {
            \coordinate (n\x) at ({ 2 * \x},0);
            \coordinate (pb\x) at ({ 2 * \x},2);
            \node[branchProj] (b\x) at (pb\x) {};
            \node (lb\x) at ({ 2 * \x},2.4) {$b_{\x}$};
            \draw (n\x) -- (b\x);
        }
        \draw[dashed,<-] (b0) -- (root);
        \draw[dashed] (b4) -- (9,2.5);
        \draw[dashed] (b4) -- (9,1.5);

        \foreach[count=\x] \y in {0,1,2,3} {
            \node (E\x) at ({ 2 * \x - 1},2.6) {$e_{\y}$};
            \draw[->,thick] (b\y) to[bend left=40] (b\x);
            \node (T\x) at ({ 2 * \x - 1},-0.6) {$\t_{b_{\y}:b_{\x}}$};
        }
        
        \node (x) at (3,0.2)  {$x$};
        \node (y) at (7,0.2)  {$y$};
        \node[branchProj] (tx) at (3,2) {};
        \node[branchProj] (ty) at (7,2) {};

        \draw[localType, <-] (x)  -- 
        node[midway, left] {$\BtL(x)$} (tx);
        (tx);
        \draw[localType, ->] (tx) -- 
        node[midway, below] {$\BrL(x)$}
        (b2);
        \draw[localType, <-] (tx) -- 
        node[midway, below] {$\BlL(x)$}
        (b1);

        \draw[localType, <-] (y)  -- 
        node[midway, left] {$\BtL(y)$} 
        (ty);
        \draw[localType, ->] (ty) -- 
        node[midway, below] {$\BrL(y)$}
        (b4);
        \draw[localType, <-] (ty) --
        node[midway, below] {$\BlL(y)$}
        (b3);
    \end{tikzpicture}
    }
    \caption{Partitionning a branch of a tree using a \kl{bough}. To compute
    the presence of an edge between $x$ and $y$ in the resulting graph, 
    it is sufficient to know the values of
    $\tlbl{\t}{b_2}{b_3}$, and the respective \kl{bough types} of $x$ and $y$.}
    \label{partitionning-a-graph:fig}
\end{figure}

\AP In the rest of the paper, we will ofter rely on some tree surgery operation
consisting in replacing a \kl{bough} in a tree by some other \kl{bough}. Given
a \kl{bough} $B$ in a tree $\t$, we write $\t = C[B]$ to denote that $\t$ can
be obtained by plugging the \kl{bough} $B$ into a \intro{context} $C[\square]$.
Formally, a \kl{context} $C[\square]$ is given by a tree $\t_{\mathsf{root}}$
with a distinguished leaf $\square$, together with two trees
$\t_{\mathsf{left}}$ and $\t_{\mathsf{right}}$. These three elements are
reprenented as dashed lines in \cref{partitionning-a-graph:fig}.
\maelin{It is unclear what this leaf $\square$. Maybe you should detail morehow you attach $B$ to the context.}

\AP Two boughs $B$ and $B'$ are \intro{compatible} if they start with the same
idempotent value, and if for every tree $\aTree = C[B]$, the tree $C[B']$ is
also well-defined (i.e., it remains \kl{forward Ramseyan}). The following
\cref{fact:bough-replacement} states that one can safely perform such a
replacement without modifying the part of the graph represented by the tree
$\t$ outside of the \kl{bough}.

\maelin{Je suis confus sur ce qu'est un bough, c'est une chemin ? un sous-arbre ? la façon dont tu le défini pour moi c'est simplement un chemin, sauf que j'ai l'impression que ça devrait être un arbre (genre un chemin avec des sous-arbre pendant).}

\begin{fact}
  \label{fact:bough-replacement}
  Let $\t = C[B]$ be a tree with a \kl{bough} $B$ of level $k$, 
  and let $H$ be a \kl{bough} that is \kl{compatible}
  with $B$. Let $\t' = C[H]$ be the tree obtained by replacing $B$ by $H$ in $\t$.
  Then, the subraph of $\someInterp(\t)$ induced by the leaves 
  outside of $B$ is isomorphic (using the identity map) 
  to the subgraph of $\someInterp(\t')$
  induced by the leaves outside of $H$.
\end{fact}



\begin{definition}
    \label{good-bough:def}
    Let $B$ be a \kl{bough} of level $k$ in a tree $\t$.
    We say that $B$ is a \intro{good bough} if, 
    there exists a \kl{compatible} \kl{bough} $H$ of level $k$,
    and a map $h \colon \someInterp(C[B]) \to \someInterp(C[H])$
    such that:
    \begin{enumerate}
        \item $h$ is an embedding of graphs,
        \item $h$ is the identity map on leaves belonging $C[\square]$,
        \item there exists a \kl{block} in $H$ that is left untouched 
          by $h$.
    \end{enumerate}
    A \intro{bad bough} is a \kl{bough} that is not a \kl{good bough}.
\end{definition}

We claim that the existence of \kl{bad boughs} of arbitrarily large
\kl{dimension} is the only obstruction to being \kl{2-well-quasi-ordered}.

\begin{lemma}
  \label{lem:good-boughs-wqo}
  The image of $\someInterp$ is \kl{$2$-well-quasi-ordered} 
  if and only if
  the image of $\someInterp$ is \kl{$\forall$-well-quasi-ordered} 
  if and only if 
  there exists a bound on the \kl{dimension} of \kl{bad boughs}.
\end{lemma}

The proof of \cref{lem:good-boughs-wqo} is split into two parts, one will
leverage the bound on the \kl{dimension} of \kl{bad boughs} to construct a
suitable tree representation of the graphs proving that they are
\kl{$\forall$-well-quasi-ordering} (\cref{sec:gap-embedding}). On the other
hand, we will show that the existence of \kl{bad boughs} of arbitrarily large
\kl{dimension} allows to construct an infinite two-labelled antichain in the
image of $\someInterp$ (\cref{sec:obstructions}).


\subsection{Gap Embedding}
\label{sec:gap-embedding}

\AP If there is a bound\mael{upper/lower ?} $N_0$ on the \kl{dimension} of \kl{bad boughs}, we can
use \cref{fact:bough-replacement} to upgrade our trees in the following way:
given a tree $\t$, we will insert dummy nodes in every \kl{bough} of level $k$
that has dimension greater than $N_0$. The resulting tree $\t'$ will have two
kind of nodes: \kl{important nodes}, that correspond to nodes of $\t$, and
\kl{dummy nodes} that were inserted during this procedure. The following 
\cref{lem:important-nodes} precises the properties of this construction.

\begin{lemma}
    \label{lem:important-nodes}
    Assume that there exists a bound $N_0$ on the \kl{dimension} of \kl{bad
    boughs}. Then, for every tree $\t$, one can effectively
    construct a tree $\t'$ equipped with a set of \kl{important nodes} such that:
    \begin{itemize}
      \item the restriction of $\t'$ to its \kl{important nodes} is $\t$,
      \maelin{Maybe def restriction of a tree (it is intuitive but may be unclear)}
      \item the restriction of $\someInterp(\t')$ to \kl{important leaves}
        is exactly $\someInterp(\t)$, \maelin{Here restriction is induced subgraph}
      \item every \kl{bough} containing only \kl{important nodes}
        has dimension less than $N_0$.
    \end{itemize}
\end{lemma}
\begin{proof}
  We proceed top-down in the tree $\t$. At the beginning, we mark 
  every node of $\t$ as \kl{important}.

  Whenever we encounter a \kl{bough} $B$ of level $k$ with dimension greater
  than $N_0$, we replace it by a \kl{good bough}\mael{equivalent ?} $H$ of level $k$ with
  dimension exactly $N_0$, which exists by assumption.
  \maelin{Je pense que ce n'est pas clair les arguments ici, pourquoi il en existe un plus petit qui est compatible ?}
   This operation does not
  modify the graph outside of the \kl{bough} by \cref{fact:bough-replacement}.
  We then mark the leaves of $H$ that correspond to leaves of $B$ as
  \kl{important}, and the other ones of $H$ as \kl{dummy}. Then, we complete
  the marking of \kl{important} nodes by asking that the least common ancestor
  of two \kl{important} nodes is also \kl{important}. By definition, the graph
  induced by the \kl{important leaves} remains unchanged by this operation.
\end{proof}

Leveraging \cref{lem:important-nodes}, we can now assume that every nodes of
interest in our trees are somehow \emph{well-separated} from each other,
allowing us to leverage the remarks of \cref{sec:ramseyan} and
\cref{fast-computation:fig}.

\AP We will now define a \kl{well-quasi-ordering} on the trees equipped with
\kl{forward Ramseyan splits} and \kl{important nodes}. The natural candidate is
called the \intro{gap embedding} relation, originally defined by Dershowitz and
Tzameret \cite{DERSHOWITZ200380}. It was noticed by Freund \cite{FREU20} that
this ordering can be understood as a nested version of Kruskal's Tree Theorem,
and this will be our point of view in the following, since we will need to
slightly adapt the classical definition to our setting.

\todo[inline]{waaaaaaa}

% TODO start with the positive part !
%
% Lemma: one can extend trees with dummy nodes such that all k-neighbourhoods of
% ``importand nodes'' are of bounded size.
%
% Lemma: one can define a quasi-ordering on trees with important nodes that 
% is wqo, and only touches dummy nodes.
%
% Conclude using Lemma 12
%

% Now: the bad part.
% Assume that there are arbitrarily long bad boughs.
% 
% Lemma 1: we can assume that the context is fixed, because 
% there are essentially finitely many contexts.
%
% Lemma 2: we can then color the graphs obtained by fixing the context in
% one color. By 2-WQO, we would obtain a pair of graphs where one embeds
% into the other, but this gives an embedding of bad boughs, which is a contradiction.
%

\subsection{Obstructions}
\label{sec:obstructions}

\AP In this section, we will assume that there 
exist \kl{bad boughs} of arbitrarily large \kl{dimension}, and we will
leverage this to construct an infinite \kl{antichain} in the image of
$\someInterp$.

Our first remark is that there are only finitely many equivalence 
classes for \kl{boughs} of a given level $k$ with respect to \kl{compatibility}.
This is the content of the following \cref{lem:finitely-many-boughs}.


\begin{lemma}
  \label{lem:finitely-many-boughs}
  Let $k$ be a level. There exists a finite set of \kl{boughs} of level $k$
  such that every \kl{bough} of level $k$ is \kl{compatible} with one of them.
\end{lemma}
\begin{proof}
  Sketch: Compatibility between two \kl{boughs} $B$ and $B'$ of level $k$ 
  is determined by the shape of the beginning and ends of the splits of $B$ and $B'$,
  and there are finitely many such shapes.
\end{proof}

\begin{lemma}
  \label{lem:finitely-many-contexts}
  There exists a finite set $\mathcal{F}$ of contexts $C[\square]$ such that
  for every bad bough $B$ of level $k$, there exists a context
  $C[\square] \in \mathcal{F}$ such that $B$ is a \kl{bad bough} in $C[B]$.
\end{lemma}
\begin{proof}
  Sketch: the context is not very important in good boughs
  because outside it is the identity map. Furthermore, the behaviour
  of nodes in the context with respect to those of the bough 
  is determined by a finite number of monoid elements.
  Hence, if a bough is good in some context, it is good in any context with 
  the same behaviour and there are finitely many such contexts.
\end{proof}


\begin{lemma}
  \label{lem:bad-bough-antichain}
  If there exist \kl{bad boughs} of arbitrarily large \kl{dimension},
  then there exists an infinite two-colored \kl{antichain} in the image of $\someInterp$.
\end{lemma}
\begin{proof}
  Assume towards a contradiction that the class of graphs 
  obtained as images of $\someInterp$ is \kl{$2$-well-quasi-ordered}.
  Let $\seqof{B_i}[i \geq 1]$ be an infinite sequence of \kl{bad boughs}. Without loss 
  of generality, one can assume that all \kl{boughs} $B_i$ have the same level $k$,
  and are pairwise \kl{compatible}, by \cref{lem:finitely-many-boughs}.

  Now, using \cref{lem:finitely-many-contexts}, one can extract one context 
  $C[\square]$ such that infinitely many \kl{boughs} $B_i$ are \kl{bad boughs}
  in $C[B_i]$. Finally, one can extract further to assume that 
  the \kl{dimension} of the \kl{boughs} $B_{i+1}$ 
  is greater than twice the number of leaves in $B_i$.

  Let us now color every graph $\someInterp(C[B_i])$ with two labels: $\top$
  to leaves belonging to $B_i$, and color $\bot$ to leaves belonging to
  $C[\square]$. Because we assume that the image of $\someInterp$ is
  \kl{$2$-well-quasi-ordered}, we can extract an infinite increasing
  subsequence, and assume that for all $i < j$, there exists an embedding $f_{i,j}
  \colon \someInterp(C[B_i]) \to \someInterp(C[B_j])$ that preserves labels.
  Let us remark that for all $i < j$, the map 
  $f_{i,j}$ acts as a permutation when restricted to nodes labelled $\bot$,
  that is of size the number of leaves in $C[\square]$.
  By a Ramsey argument, we can therefore assume that this permutation is the identity
  for all $i < j$.
  Finally, consider any two indices $i < j$. 

  Because the \kl{dimension} of $B_j$ is greater than twice the number of leaves
  in $B_i$, there must be some \kl{block} in $B_j$ that is left untouched by $f_{i,j}$.
  Furthermore, we know that $f_{i,j}$ is the identity on leaves
  belonging to $C[\square]$. Hence, the map $f_{i,j}$
  witnesses that $B_i$ is a \kl{good bough} in $C[B_i]$, which is a contradiction.
\end{proof}


\section{Hereditary Classes}
\label{sec:hereditary-classes}

We use a notion of \kl{perfect boughs}

\begin{definition}[Perfect Bough]
  \label{def:perfect-bough}
  Let $\aTree = C[B]$ be a tree with a \kl{bough} $B$ of level $k$.
  We say that $B$ is a \intro{perfect bough} there exists a compatible
  \kl{bough} $H$ of level $k$ and 
  a map $h \colon \someInterp(C[B]) \to \someInterp(C[H])$
  \begin{itemize}
      \item $h$ is an embedding of graphs,
      \item $h$ is the identity map on leaves belonging $C[\square]$
      \item the \kl{bough type} of every leaf $x$ in $H$ is the same as 
        the \kl{bough type} of $h(x)$ in $B$.
      \item there exists a \kl{block} in $H$ that is left untouched 
          by $h$.
  \end{itemize}
  An \intro{imperfect bough} is a \kl{bough} that is not a \kl{perfect bough}.
\end{definition}

\begin{lemma}
  \label{lem:perfect-boughs-wqo}
  The image of $\someInterp$ is \kl{$2$-well-quasi-ordered} 
  if and only if
  the image of $\someInterp$ is \kl{$\forall$-well-quasi-ordered} 
  if and only if 
  there exists a bound on the \kl{dimension} of \kl{imperfect boughs}.
\end{lemma}
\begin{proof}
  First, if we have a bound on the \kl{dimension} of \kl{imperfect boughs},
  then we also have a bound on the \kl{dimension} of \kl{bad boughs},
  hence by \cref{lem:good-boughs-wqo}, the image of $\someInterp$ is
  \kl{$\forall$-well-quasi-ordered}.
  \maelin{I don't see the implication, can't you have unbounded dimension on bad bough why having bound for the imperfects}

  Conversely, assume that there exist \kl{imperfect boughs} of arbitrarily
  large \kl{dimension}. Then, by an argument similar to the one of 
  \cref{lem:bad-bough-antichain}, we can construct an infinite
  labelled \kl{antichain} in the image of $\someInterp$. The only change is that 
  one also adds the \kl{bough type} of the leaves as additional labels.
\end{proof}

The important advantage of \cref{lem:perfect-boughs-wqo} over
\cref{lem:good-boughs-wqo} is that \kl{imperfect boughs} are easily
recognizable.

\begin{lemma}
  \label{lem:recognizing-perfect-boughs}
  There exists an $\MSO$ formula that recognizes \kl{imperfect boughs}.
\end{lemma}

\begin{corollary}
  One can decide whether there are \kl{imperfect boughs} of arbitrarily large
  \kl{dimension}.
\end{corollary}

One other advantage of \cref{lem:perfect-boughs-wqo} is that
in the case of hereditary classes, we can further simplify 
the statement. 

\begin{lemma}
  \label{lem:hereditary-perfect-boughs}
  Assume that there are \kl{imperfect boughs} of arbitrarily large \kl{dimension}.
  Then, there exists a finite set of blocks $\mathbb{F}$ such that 
  one can build an \kl{imperfect bough} of arbitrarily large \kl{dimension}
  using only \kl{blocks} from $\mathbb{F}$.
\end{lemma}

As an immediate consequence of \cref{lem:perfect-boughs-wqo} and
\cref{lem:hereditary-perfect-boughs}, we see that 

\begin{corollary}
  \label{cor:hereditary-perfect-boughs-wqo}
  Assume that the class $\someInterp$ is hereditary.
  Then, the image of $\someInterp$ is \kl{$2$-well-quasi-ordered}
  if and only if 
  every class of \kl{bounded linear clique-width} in $\someInterp$
  is \kl{$2$-well-quasi-ordered}.
\end{corollary}
