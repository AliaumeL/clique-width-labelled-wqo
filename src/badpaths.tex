\section{Bad Patterns in Images of Interpretations}
\label{sec:bad-patterns}

In this section, we will isolate combinatorial obstructions to being
\kl{$2$-well-quasi-ordered} in classes of graphs that are images of \kl{monoid interpretations} from trees.
 We will take the same notations as in
\cref{sec:ramseyan} 
regarding \kl{forward Ramseyan splits} and \kl{monoid interpretations}.
The key idea of this section is to start from a tree $\t$ equipped with a
\kl{forward Ramseyan split} $\spt$, and try to understand under which conditions
one can turn this tree into a \kl{$L$-bounded marked nested tree} representing the same graph, 
allowing us to leverage \cref{thm:marked-nested-trees-wqo} 
to conclude that the class of graphs is \kl{$\forall$-well-quasi-ordered}.


\begin{figure*}[ht]
    \centering
    \begin{tikzpicture}[
        localType/.style={
            color=C3,
            thick
        },
        branchProj/.style = {
            color=Prune,
            inner sep=0pt,
            minimum size=4pt,
            fill,
            circle
        },
        leaf/.style={
            color=Prune,
        },
        ]
        \draw (0,0) rectangle (8,2);
        \node (root) at (-3.2,2) {$\treeRoot$};
        \foreach \x in {0,1,2,3,4} {
            \coordinate (n\x) at ({ 2 * \x},0);
            \coordinate (pb\x) at ({ 2 * \x},2);
            \node[branchProj] (b\x) at (pb\x) {};
            \node (lb\x) at ({ 2 * \x},2.4) {$b_{\x}$};
            \draw (n\x) -- (b\x);
        }
        \draw[dashed,<-] (b0) -- (root);
        \draw[dashed] (b4) -- (9,2.5);
        \draw[dashed] (b4) -- (9,1.5);

        \foreach[count=\x] \y in {0,1,2,3} {
            \node (E\x) at ({ 2 * \x - 1},2.6) {$e_{\y}$};
            \draw[->,thick] (b\y) to[bend left=40] (b\x);
            \node (T\x) at ({ 2 * \x - 1},-0.6) {$\BBlock[\t]{b_{\y}}{b_{\x}}$};
        }
        
        \node (x) at (3,0.2)  {$x$};
        \node (y) at (7,0.2)  {$y$};
        \node[branchProj] (tx) at (3,2) {};
        \node[branchProj] (ty) at (7,2) {};

        \draw[localType, <-] (x)  -- 
        node[midway, left] {$\BtL(x)$} (tx);
        (tx);
        \draw[localType, ->] (tx) -- 
        node[midway, below] {$\BrL(x)$}
        (b2);
        \draw[localType, <-] (tx) -- 
        node[midway, below] {$\BlL(x)$}
        (b1);

        \draw[localType, <-] (y)  -- 
        node[midway, left] {$\BtL(y)$} 
        (ty);
        \draw[localType, ->] (ty) -- 
        node[midway, below] {$\BrL(y)$}
        (b4);
        \draw[localType, <-] (ty) --
        node[midway, below] {$\BlL(y)$}
        (b3);

        \draw[dashed] (9,2.5) -- (10,2.1) -- (10,2.9) -- cycle;
        \draw[dashed] (9,1.5) -- (10,1.1) -- (10,1.9) -- cycle;

        % now draw similar dashed triangles on the path from the root to b0
        \draw[dashed] (-2,2) -- (-2,1.5);
        \draw[dashed] (-2,1.5) -- (-1.7,1) -- (-2.3,1) -- cycle;
        \draw[dashed] (-1,2) -- (-1,1.5);
        \draw[dashed] (-1,1.5) -- (-0.7,1) -- (-1.3,1) -- cycle;

    \end{tikzpicture}
    \caption{Partitionning a branch of a tree $\aTree$ using a \kl{bough} $B$.
    Here, the nodes $b_i$ for $0 \leq i \leq 4$ are \kl{$k$-neighbours}
    forming the \kl(bough){backbone} $B$.
    The letter $e_i$ denotes \kl{idempotent} monoid element 
    $\tlbl{\aTree}{b_i}{b_{i+1}}$. In dashed, we represented the
    \kl(bough){context} $C[\square]$ such that $\t = C[B]$.
    To compute
    the presence of an edge between $x$ and $y$ in the resulting graph, 
    it is sufficient to know the values of
    $\tlbl{\t}{b_2}{b_3}$, and the respective \kl{bough types} of $x$ and $y$
    in their respective \kl{bough blocks} $\BBlock{b_1}{b_2}$ and $\BBlock{b_3}{b_4}$.}
    \label{partitionning-a-graph:fig}
\end{figure*}

\begin{definition}
    \label{ramseyan-branch:def}
    Let $\aTree$ be a tree and $\spt$ be a \kl{forward Ramseyan split} of height $N$.
    A \intro{bough of level $k$} in $\aTree$ is defined from a part of a $k$-neighbourhood as follows:
    Consider a sequence $b_0, b_1, \ldots, b_n$ of nodes of level $k$ on a branch of $\aTree$
    forming a \kl{$k$-neighbourhood}.
    Elements of the \reintro{bough} $B$ are all nodes $x$ in $\aTree$ such that 
    $b_0 \treeleq x$, and $\neg (b_n \treeleq x)$, plus $b_n$ itself.
    
    We say that the \intro(bough){dimension} of the \kl{bough} $B$ is $n$,
    and that $b_0, b_1, \ldots, b_n$ form the \intro(bough){backbone} of $B$.
\end{definition}

\AP Let $B$ be a \kl{bough of level $k$} in $\aTree$
and let $b_0, b_1, \ldots, b_n$ be its \kl(bough){backbone}.
Given indices $i < j$, 
we define the $\BBlock[\aTree]{b_i}{b_{j}}$
as the set of nodes $x$ in $\aTree$ such that
$b_i \treeleq x$ and $\neg (b_{j} \treeleq x)$.
When $j = i + 1$,  this set of nodes is called a \intro(bough){block}. We also refer to
\cref{partitionning-a-graph:fig} for an
illustration of the resulting partition
of the tree $\aTree$ with respect to a given \kl{bough} $B$.

\AP
For every leaf $x \in \BBlock[\aTree]{b_0}{b_n}$ where $n$ is the
\kl{dimension of the bough} $B$, we define $\intro*\Bt(x)$ to be the closest \kl(tree){ancestor} of
$x$ such that $b_0 \treeleq \reintro*\Bt(x) \treelt b_n$, that is, the closest \kl(tree){ancestor}
of $x$ on the branch going through the nodes $b_0, b_1, \ldots, b_n$. 
Similarly, we define $\intro*\Bl(x)$ and $\intro*\Br(x)$
to be such that $\Bl(x)$ and $\Br(x)$ belong to $\set{b_0, b_1, \ldots, b_n}$,
and are the closest ones such that $\Bl(x) \treeleq \Bt(x) \treeleq \Br(x)$.
To every leaf $x$ of $B(\aTree)$,
Using these referenece points, we define the following monoid elements 
associated to a leaf $x$ in a \kl(bough){block} of $B$:
\begin{align*}
  \intro*\BtL(x) &\defined \tlbl{\t}{\Bt(x)}{x}, \\
  \intro*\BlL(x) &\defined \tlbl{\aTree}{\Bl(x)}{\Bt(x)}, \\
  \intro*\BrL(x) &\defined \tlbl{\t}{\Bt(x)}{\Br(x)}, \\
  \intro*\BrootL(x) &\defined \tlbl{\aTree}{\treeRoot}{\Bl(x)}.
\end{align*}
We call this tuple 
the \intro{bough type} of the leaf $x$ with respect to the \kl{bough} $B$.
We again refer to 
\cref{partitionning-a-graph:fig}
for an illustration of the type of a leaf
with respect to a given \kl{bough} $B$. There are only finitely 
many possible \kl{bough types} because $M$ is finite.


\AP In the rest of the paper, we will ofter rely on some tree surgery operation
consisting in replacing a \kl{bough} in a tree by some other \kl{bough}. Given
a \kl{bough} $B$ in a tree $\t$ defined using a \kl{$k$-neighbourhood}
$b_0$, $b_1$, \ldots, $b_n$,
we write $\t = C[B]$ to denote that $\t$ can
be obtained by plugging the \kl{bough} $B$ into a \intro(bough){context} $C[\square]$.
Formally, a \reintro(bough){context} $C[\square]$ is given by a tree $\t_{\mathsf{root}}$
with a distinguished leaf $\square$, together with two trees
$\t_{\mathsf{left}}$ and $\t_{\mathsf{right}}$. These three elements are
reprenented as dashed lines in \cref{partitionning-a-graph:fig}.
The tree $C[B]$ is then obtained as follows: one first attaches 
$\t_{\mathsf{left}}$ as the left subtree of $b_n$,
where $b_n$ is the last element of the \kl{$k$-neighbourhood} defining $B$,
then one attaches $\t_{\mathsf{right}}$ as the right subtree of $b_n$, and finally,
one replaces the distinguished leaf $\square$ in $\t_{\mathsf{root}}$ by the
node $b_0$, the first element of the \kl{$k$-neighbourhood} defining $B$.

\AP Two boughs $B$ and $B'$ are \intro(bough){compatible} if they start with the same
idempotent value, and if for every tree (with a \kl{split}) 
$\aTree = C[B]$ that is \kl{forward Ramseyan}, the tree $C[B']$ is
remains \kl{forward Ramseyan}. The following
\cref{fact:bough-replacement} states that one can safely perform such a
replacement without modifying the part of the graph represented by the tree
$\t$ outside of the \kl{bough}. This is because to determine the presence of 
edges between leaves outside of the \kl{bough}, one only needs to know the 
value of the idempotent element labelling the \kl{bough}, which does not 
change when replacing $B$ by a \kl{compatible bough} $H$.

\begin{fact}
  \label{fact:bough-replacement}
  Let $\t = C[B]$ be a tree with a \kl{bough} $B$ of level $k$, 
  and let $H$ be a \kl{bough} that is \kl(bough){compatible}
  with $B$. Let $\t' = C[H]$ be the tree obtained by replacing $B$ by $H$ in $\t$.
  Then, the subraph of $\someInterp(\t)$ induced by the leaves in 
  the \kl(bough){context} $C[\square]$ equals the 
  subgraph of $\someInterp(\t')$ induced by the leaves in
  the \kl(bough){context} $C[\square]$.
\end{fact}



\begin{definition}
    \label{good-bough:def}
    Let $B$ be a \kl{bough} of level $k$ in a tree $\t$.
    We say that $B$ is a \intro{good bough} if, 
    there exists a \kl{compatible bough} $H$ of level $k$,
    and a map $h \colon \someInterp(C[B]) \to \someInterp(C[H])$
    such that:
    \begin{enumerate}
        \item $h$ is an embedding of graphs,
        \item $h$ is the identity map on leaves belonging $C[\square]$,
        \item there exists a pair of consecutive 
            \kl(bough){blocks} in $H$ such that all their leaves
            are left untouched by $h$.
    \end{enumerate}
    A \intro{bad bough} is a \kl{bough} that is not a \kl{good bough}.
\end{definition}

We claim that the existence of \kl{bad boughs} of arbitrarily large
\kl(bough){dimension} is the only obstruction to being \kl{$2$-well-quasi-ordered}.

\begin{lemma}
  \label{lem:good-boughs-wqo}
  The image of $\someInterp$ is \kl{$2$-well-quasi-ordered} 
  if and only if
  the image of $\someInterp$ is \kl{$\forall$-well-quasi-ordered} 
  if and only if 
  there exists a bound on the \kl(bough){dimension} of \kl{bad boughs}.
\end{lemma}

The proof of \cref{lem:good-boughs-wqo} is split into two parts, one will
leverage the bound on the \kl(bough){dimension} of \kl{bad boughs} to construct a
suitable tree representation of the graphs proving that they are
\kl{$\forall$-well-quasi-ordering} (\cref{sec:gap-embedding}). On the other
hand, we will show that the existence of \kl{bad boughs} of arbitrarily large
\kl(bough){dimension} allows to construct an infinite two-labelled antichain in the
image of $\someInterp$ (\cref{sec:obstructions}).


\subsection{Using the Gap Embedding}
\label{sec:gap-embedding}

\AP If there is an upper bound $N_0$ on the \kl{dimension} of \kl{bad boughs}, we can
use \cref{fact:bough-replacement} to upgrade our trees in the following way:
given a tree $\t$, we will insert dummy nodes in every \kl{bough} of level $k$
that has \kl{dimension} greater than $N_0$. The resulting tree $\t'$ will be a
\kl{marked nested tree} as defined in \cref{sec:ramseyan}, with three
kinds of nodes: \kl{marked nodes}, that correspond to nodes of $\t$,
\kl{dummy nodes} that were inserted during this procedure,
and \kl{separating nodes}, that are the last nodes of the \kl{$k$-neighbourhoods}
defining the \kl{boughs} we manipulated.
The following 
\cref{lem:important-nodes} shows that this construction produces
\kl{well-marked nested trees} that are \kl{$N_0$-bounded}, allowing us to apply
\cref{thm:marked-nested-trees-wqo}.

\begin{lemma}
    \label{lem:important-nodes}
    Assume that there exists a bound $N_0$ on the \kl{dimension} of \kl{bad
    boughs}. Then, for every tree $\t$ equipped with a \kl{forward Ramseyan split} $\spt$, 
    one can effectively construct a \kl{well-marked nested tree} $(\t', \spt', \marking)$ 
  that is \kl{$L$-bounded} for some $L \in \Nat$, and such that
    the restriction of $\someInterp(\t')$ to \kl{marked leaves}
    is isomorphic to $\someInterp(\t)$.
\end{lemma}
\begin{proof}
  We proceed iteratively starting from the tree $\t$
  where every node is set as a \kl{marked node}.
  A rewrite step is as follows:
  given a \kl{bough backbone} $b_0, b_1, \ldots, b_n$ defining a \kl{bough} $B$,
  where all $b_i$ are \kl{marked nodes}, and $n = N_0 + 1$,
  we know that $B$ is a \kl{good bough} by assumption.
  As a consequence, there exists a \kl{compatible bough} $H$ of 
  same level as $B$ that can replace $B$ in $\t$
  by definition of \kl{good boughs}.
  The resulting tree $\t'$ is still \kl{forward Ramseyan} by definition of
  \kl{compatible boughs}, and the graph induced by leaves outside of $B$
  remains unchanged by \cref{fact:bough-replacement}.
  Let us now describe the marking on $\t'$:
  leaves of $B$ that were marked 
  remain marked in $H$, untouched leaves of $H$ become \kl{dummy},
  and if  
  $\BBlock[H]{h_i}{h_{i+1}}$ and $\BBlock[H]{h_{i+1}}{h_{i+2}}$
  are the two consecutive \kl{bough blocks} left untouched by the embedding
  witnessing that $B$ is a \kl{good bough},
  then we mark $h_{i}$ as a \kl{separating node},
  and all other nodes of these two \kl{bough blocks} as \kl{dummy nodes}.
  Then, we complete
  the marking by ensuring that the least common ancestor
  of two \kl{marked nodes} is also \kl{marked}. By definition, the graph
  induced by the \kl{marked leaves} remains unchanged by this operation.

  We repeat this procedure until there is no \kl{bough} of \kl{dimension}
  greater than $N_0$ whose \kl{backbone} is made of \kl{marked nodes}. This
  process terminates because every rewrite step decreases the multiset of
  \kl{dimensions} of \kl{boughs} whose \kl{backbone} is made of \kl{marked
  nodes}.

  
  To verify that $(\t', \spt', \marking)$ is \kl{well-marked},
  note that by construction,
  whenever we have \kl{marked nodes} $x$ and $y$ with a non-\kl{marked} node in between
  at level $k$, this configuration arises from a \kl{bough} replacement where
  the required witnesses $z_1, z_2, z_3$ exist by the definition of \kl{good boughs}.

  Assume towards a contradiction that there exists arbitrarily long 
  paths
  of non-\kl{dummy} nodes in the resulting tree $\t'$.
  Remark that any path that is too long contains 
  the \kl{backbone} of a \kl{bough} of \kl{dimension} greater than $N_0$
  by a Ramsey argument.
  By construction, every \kl{separating node} in such a
  \kl{bough backbone} is followed by  a \kl{dummy node}, so without loss of
  generality, one can assume that the \kl{bough backbone} is made solely
  of \kl{marked nodes}. But this contradicts the termination
  of the procedure.
  We have proven that there exists a bound on the length of paths
  of non-\kl{dummy} nodes, concluding the proof.
\end{proof}

\AP We conclude this section by proving one direction of
\cref{lem:good-boughs-wqo}. Assume that there exists a bound on the
\kl{dimension} of \kl{bad boughs}. By \cref{lem:important-nodes}, we can
transform any tree into a \kl{$L$-bounded} \kl{well-marked nested tree}. By
\cref{thm:marked-nested-trees-wqo}, the class of such trees is
\kl{well-quasi-ordered} under the \kl{gap-embedding} ordering. Furthermore, by
\cref{cor:gap-embedding-monotone}, the \kl{gap-embedding} ordering between
\kl{marked nested trees} preserves the interpretation: if $(\t_1, \spt_1,
\marking_1) \gemb (\t_2, \spt_2, \marking_2)$, then the graph interpreted from
$(\t_1, \spt_1, \marking_1)$ is an induced subgraph of the graph interpreted
from $(\t_2, \spt_2, \marking_2)$, when both are restricted to \kl{marked
leaves}. Since the interpretation restricted to \kl{marked leaves} coincides
with the original interpretation on $\t_1$ and $\t_2$ by
\cref{lem:important-nodes}, this establishes the desired order-preserving
property, and we conclude that the image of $\someInterp$ is
\kl{$\forall$-well-quasi-ordered} using \cref{fact:surjective-wqo}. 

\subsection{Obstructions}
\label{sec:obstructions}

\AP In this section, we will assume that there 
exist \kl{bad boughs} of arbitrarily large \kl{dimension}, and we will
leverage this to construct an infinite \kl{antichain} in the image of
$\someInterp$.

Our first remark is that there are only finitely many equivalence 
classes for \kl{boughs} of a given level $k$ with respect to \kl{compatibility}.
This is the content of the following \cref{lem:finitely-many-boughs}.


\begin{lemma}
  \label{lem:finitely-many-boughs}
  Let $k$ be a level. There exists a finite set of \kl{boughs} of level $k$
  such that every \kl{bough} of level $k$ is \kl{compatible} with one of them.
\end{lemma}
\begin{proof}
  To know if a \kl{bough} $B$ is \kl{compatible} with another \kl{bough} $B'$,
  it suffices to know the first \kl{idempotent} element labelling $B$ and $B'$,
  and for every \kl{split depth} $d$, 
  the possible values of monoid elements such that 
  when adding them in front it is still \kl{forward Ramseyan}.

  Indeed, when placing a \kl{bough} $B$ in a \kl{context} $C[\square]$,
  the only way that the tree $C[B]$ can fail to be \kl{forward Ramseyan}
  is that some ...

\end{proof}

\begin{lemma}
  \label{lem:finitely-many-contexts}
  There exists a finite set $\mathcal{F}$ of contexts $C[\square]$ such that
  for every bad bough $B$ of level $k$, there exists a context
  $C[\square] \in \mathcal{F}$ such that $B$ is a \kl{bad bough} in $C[B]$.
\end{lemma}
\begin{proof}
  We say that two contexts $C_1[\square]$ and $C_2[\square]$
  are equivalent if for every \kl{bough} $B$ of level $k$,
  $C[B]$ is \kl{forward Ramseyan} if and only if
\end{proof}


\begin{lemma}
  \label{lem:bad-bough-antichain}
  If there exist \kl{bad boughs} of arbitrarily large \kl{dimension},
  then there exists an infinite two-colored \kl{antichain} in the image of $\someInterp$.
\end{lemma}
\begin{proof}
  Assume towards a contradiction that the class of graphs 
  obtained as images of $\someInterp$ is \kl{$2$-well-quasi-ordered}.
  Let $\seqof{B_i}[i \geq 1]$ be an infinite sequence of \kl{bad boughs}. Without loss 
  of generality, one can assume that all \kl{boughs} $B_i$ have the same level $k$,
  and are pairwise \kl{compatible}, by \cref{lem:finitely-many-boughs}.

  Now, using \cref{lem:finitely-many-contexts}, one can extract one context 
  $C[\square]$ such that infinitely many \kl{boughs} $B_i$ are \kl{bad boughs}
  in $C[B_i]$. Finally, one can extract further to assume that 
  the \kl{dimension} of the \kl{boughs} $B_{i+1}$ 
  is greater than twice the number of leaves in $B_i$.

  Let us now color every graph $\someInterp(C[B_i])$ with two labels: $\top$
  to leaves belonging to $B_i$, and color $\bot$ to leaves belonging to
  $C[\square]$. Because we assume that the image of $\someInterp$ is
  \kl{$2$-well-quasi-ordered}, we can extract an infinite increasing
  subsequence, and assume that for all $i < j$, there exists an embedding $f_{i,j}
  \colon \someInterp(C[B_i]) \to \someInterp(C[B_j])$ that preserves labels.
  Let us remark that for all $i < j$, the map 
  $f_{i,j}$ acts as a permutation when restricted to nodes labelled $\bot$,
  that is of size the number of leaves in $C[\square]$.
  By a Ramsey argument, we can therefore assume that this permutation is the identity
  for all $i < j$.
  Finally, consider any two indices $i < j$. 

  Because the \kl{dimension} of $B_j$ is greater than twice the number of leaves
  in $B_i$, there must be some \kl{block} in $B_j$ that is left untouched by $f_{i,j}$.
  Furthermore, we know that $f_{i,j}$ is the identity on leaves
  belonging to $C[\square]$. Hence, the map $f_{i,j}$
  witnesses that $B_i$ is a \kl{good bough} in $C[B_i]$, which is a contradiction.
\end{proof}

