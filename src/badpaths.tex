\clearpage
\section{Bad Patterns in Images of Interpretations}
\label{sec:bad-patterns}

In this section, we will isolate combinatorial obstructions to being
\kl{2-well-quasi-ordered} in classes of graphs that are images of \kl{simple
MSO interpretations} from trees. We will take the same notations as in
\cref{sec:ramseyan}. 

\begin{definition}
    \label{ramseyan-branch:def}
    Let $\aTree$ be a tree and $\spt$ be a \kl{forward Ramseyan split} of height $N$.
    A \intro{bough of level $k$} in $\aTree$ is an infix of a branch of $\aTree$
    such that its maximal and minimal elements have level $k$,
    and such that 
    all elements of the bough have level greater or equal to $k$.

    The \intro{dimension of a bough} is the number of elements of level $k$
    in the \kl{bough}.
\end{definition}

\AP Let $B$ be a \kl{bough of level $k$} in $\aTree$. Given two nodes $b_1$ and
$b_2$ in $B$ such that $\spt(b_1) = \spt(b_2) = k$, we define the $\aTree_{b_1:b_2}$
as the set of nodes $x$ in $\aTree$ such that $b_1 \treeleq x$ and $\neg (b_2
\treeleq x)$. For every leaf $x \in \aTree_{b_0: b_n}$ where $n$ is the
\kl{dimension of the bough} $B$, we define $\Bt(x)$ to be the least ancestor of
$x$ that belongs to $B$. Similarly, we define $\Bl(x)$ to be the least element
of $B$ that is greater or equal to $\Bt(x)$, and $\Br(x)$ to be the greatest
element of $B$ that is less or equal to $\Bt(x)$. To every leaf $x$ of $B(\aTree)$,
we can therefore associate the following values in $M$:
$\BtL(x) \defined
\tlbl{\t}{\Bt(x)}{x}$, $\BlL(x) \defined \tlbl{\aTree}{\Bl(x)}{\Bt(x)}$,
$\BrL(x) \defined \tlbl{\t}{\Bt(x)}{\Br(x)}$,
and $\BrootL(x) \defined \tlbl{\aTree}{\treeRoot}{\Bl(x)}$. We call this tuple 
the \intro{bough type} of the leaf $x$ with respect to the \kl{bough} $B$.
We refer to
\cref{type-of-a-leaf-in-branch:fig} for an illustration of the type of a leaf
with respect to a given \kl{bough} $B$. We also refer to
\cref{partitionning-a-graph:fig} for an illustration of the resulting partition
of the tree $\aTree$ with respect to a given \kl{bough} $B$.

\begin{figure}
    \centering
    \begin{tikzpicture}[
        branch/.style={
            color=Prune,
            inner sep=0pt,
            minimum size=4pt,
            fill,
            circle
        },
        inner/.style={
            color=A1,
            inner sep=0pt,
            minimum size=4pt,
            draw,
            circle
        },
        staredge/.style={
            color=A1,
            ->,
            dashed
        },
        root/.style={
            color=Prune,
        },
        leaf/.style={
            color=Prune,
        },
        monoid/.style={
            color=A2
        },
        ]
        % first draw the branch 
        \node[root]   (root) at (-2,0) {$\treeRoot$};
        \node[branch] (b0) at (0,0) {};
        \node[branch] (b1) at (4,0) {};
        \node[inner]  (t)  at (2,0) {};
        \node[leaf]   (x)  at (3.5,-2) {$x$};

        \node[above=0.1cm of b0] {$\Bl(x)$};
        \node[above=0.1cm of t]  {$\Bt(x)$};
        \node[above=0.1cm of b1] {$\Br(x)$};

        \draw[staredge] (root) -- (b0);
        \draw[staredge] (b0) -- node[monoid, midway, below] {$\BlL(x)$} (t);
        \draw[staredge] (t)  -- node[monoid, midway, below] {$\BrL(x)$} (b1);
        \draw[staredge] (b1) -- (5,0);

        \draw[staredge] (t)  -- node[monoid, midway, below left] {$\BtL(x)$} (x);
    \end{tikzpicture}
    \caption{The type of a leaf with respect to a given \kl{bough} $B$.}
    \label{type-of-a-leaf-in-branch:fig}
\end{figure}

\begin{figure}
    \centering
    \resizebox{0.9\linewidth}{!}{
    \begin{tikzpicture}[
        localType/.style={
            color=C3,
            thick
        },
        branchProj/.style = {
            color=Prune,
            inner sep=0pt,
            minimum size=4pt,
            fill,
            circle
        },
        leaf/.style={
            color=Prune,
        },
        ]
        \draw (0,0) rectangle (8,2);
        \node (root) at (-1.2,2) {$\treeRoot$};
        \foreach \x in {0,1,2,3,4} {
            \coordinate (n\x) at ({ 2 * \x},0);
            \coordinate (pb\x) at ({ 2 * \x},2);
            \node[branchProj] (b\x) at (pb\x) {};
            \node (lb\x) at ({ 2 * \x},2.4) {$b_{\x}$};
            \draw (n\x) -- (b\x);
        }
        \draw[dashed,<-] (b0) -- (root);
        \draw[dashed] (b4) -- (9,2.5);
        \draw[dashed] (b4) -- (9,1.5);

        \foreach[count=\x] \y in {0,1,2,3} {
            \node (E\x) at ({ 2 * \x - 1},2.6) {$e_{\y}$};
            \draw[->,thick] (b\y) to[bend left=40] (b\x);
            \node (T\x) at ({ 2 * \x - 1},-0.6) {$T_{b_{\y}:b_{\x}}$};
        }
        
        \node (x) at (3,0.2)  {$x$};
        \node (y) at (7,0.2)  {$y$};
        \node[branchProj] (tx) at (3,2) {};
        \node[branchProj] (ty) at (7,2) {};

        \draw[localType, <-] (x)  -- 
        node[midway, left] {$\BtL(x)$} (tx);
        (tx);
        \draw[localType, ->] (tx) -- 
        node[midway, below] {$\BrL(x)$}
        (b2);
        \draw[localType, <-] (tx) -- 
        node[midway, below] {$\BlL(x)$}
        (b1);

        \draw[localType, <-] (y)  -- 
        node[midway, left] {$\BtL(y)$} 
        (ty);
        \draw[localType, ->] (ty) -- 
        node[midway, below] {$\BrL(y)$}
        (b4);
        \draw[localType, <-] (ty) --
        node[midway, below] {$\BlL(y)$}
        (b3);
    \end{tikzpicture}
    }
    \caption{Partitionning a branch of a tree using a \kl{bough}. To compute
    the presence of an edge between $x$ and $y$ in the resulting graph, 
    it is sufficient to know the values of
    $\tlbl{\t}{b_2}{b_3}$, and the respective \kl{bough types} of $x$ and $y$.}
    \label{partitionning-a-graph:fig}
\end{figure}

\AP In the rest of the paper, we will ofter rely on some tree surgery operation
consisting in replacing a \kl{bough} in a tree $\t$ by some other \kl{bough}.
Two boughs $B$ and $B'$ are \intro{compatible} if they start with the same
idempotent value, and if for every tree $\aTree$ that contains $B$, one can
replace $B$ by $B'$ to obtain a new tree $\aTree'$ that remains \kl{forward
Ramseyan}, and conversely for every tree $\aTree'$ that contains $B'$. The
following \cref{fact:bough-replacement} states that one can safely perform such
a replacement without modifying the part of the graph represented by the tree
$\t$ outside of the \kl{bough}.

\begin{fact}
  \label{fact:bough-replacement}
  Let $\t$ be a tree with a \kl{bough} $B$ of level $k$, 
  and let $H$ be a \kl{bough} that is \kl{compatible}
  with $B$. Let $\t'$ be the tree obtained by replacing $B$ by $H$ in $\t$.
  Then, the subraph of $\someInterp(\t)$ induced by the leaves 
  outside of $B$ is isomorphic (using the identity map) 
  to the subgraph of $\someInterp(\t')$
  induced by the leaves outside of $H$.
\end{fact}

\AP Now that we have an understanding of what happens when replacing a
\kl{bough} in a fixed tree, let us study what happens when we do the opposite
operation, i.e., when we chance the ``context'' in which a \kl{bough} is
placed. 

% TODO:
% - a context is a tree with a bough removed, given by a triple (Troot, Tleft, Tright) where
%   where Troot is the tree above the bough with a distinguished leaf where to plug the bough,
%   and Tleft and Tright are the trees below the bough.
% 
% - the context type of a leaf in a context C is given by the values of 
%   paths from the leaf to the root of Tleft resp Tright


Let us define a \intro{context} $C[\square]$ to be a pair of a tree $A$
with a distinguished hole $\square$ that can be filled by another tree, and a
tree $B$ that can be plugged into this hole. Given a tree $T$ with a hole, one
can place it inside a context $C[\square]$ to obtain a new tree $C[T]$ that is
defined by plugging $T$ into the hole of $A$, and plugging $B$ into the hole of
$T$.


\begin{definition}
    \label{good-bough:def}
    Let $B$ be a \kl{bough} of level $k$ in a tree $\t$.
    We say that $B$ is a \intro{good bough} if, 
    there exists a \kl{compatible} \kl{bough} $H$ of level $k$,
    and a map $h \colon \someInterp(\t{[B]}) \to \someInterp(\t{[H]})$
    such that:
    \begin{enumerate}
        \item $h$ is an embedding of graphs,
        \item $h$ is the identity map on leaves outside of $B$,
        \item there exists a block in $H$ that is left untouched 
          by $h$.
    \end{enumerate}
\end{definition}

\AP Therefore, the main combinatorial obstacle to understanding whether or not
a class is \kl{labelled-well-quasi-ordered} is to understand the behaviour of
pairs of nodes $x$ and $y$ that are \kl{dependent}. This is the motivation
for the following construction that studies the behaviour of
consecutive \kl{$k$-neighbourhoods} in a tree.
\subsection{Gap Embedding(s)}

In order to compare two trees equipped with \kl{forward Ramseyan splits}, we
will design an ordering that respects the tree structure as well as the splits.
This is usually called the \intro{gap embedding} relation, originally defined
by Dershowitz and Tzameret \cite{DERSHOWITZ200380}. It was noticed by Freund
\cite{FREU20} that this ordering can be understood as a nested version of
Kruskal's Tree Theorem, and this will be our point of view in the following,
since we will need to slightly adapt the definition to our setting.

\AP Let us consider some $k \in \set{1, \dots, N}$. We will write
$\Trees{\Sigma,k}{X}(Y)$ for the class of trees with labels in $X \times \set{k,
\ldots, N}$ on the internal nodes, labels in $\Sigma$ on the edges, with a
distinguished root, and such that leaves are labelled in $Y$.
It is immediate that $\Trees{\Sigma,k-1}{X}(Y)$ 
is the least fixed point of the following operator:
\begin{equation*}
  F \colon Z \mapsto Y + \Trees{\Sigma,k}{X}(Z) \quad .
\end{equation*}

It was shown in \cite{LOPEZ23} that any such inductive definition gives 
rise to a \kl{well-quasi-order}, obtained by 





Hence, we are in the opposite situation as with the \kl{composition ordering}:
we have a \kl{well-quasi-order} on trees, but we do not know whether the
interpretation $\someInterp$ is order-preserving from trees (with a choice of
split) to graphs. However, when inserting new nodes, we are often able to state
that the value of $\tlbl{\t}{x}{y}$ is preserved provided that $x$ and $y$ are
sufficiently far apart in the \kl{forward Ramseyan split} of the tree, as
stated in the following lemma.

Problems may arise when $x$ and $y$ are too close in the \kl{forward Ramseyan
split}, for instance because they are \kl{one-separated at level $k$}, or
because one inserts new nodes in the first and last part of the path from $x$
to $y$. A typical example of such a situation is illustrated in the case of a
monoid $M$ with two elements $\set{a, b}$ such that $a^2 = b$ and $b$ is
absorbing. Then, replacing an edge $x \treelt y$ labelled with $a$ by two or more
edges will change the value of $\tlbl{\t}{x}{y}$ from $a$ to $b$.

% TODO start with the positive part !
%
% Lemma: one can extend trees with dummy nodes such that all k-neighbourhoods of
% ``importand nodes'' are of bounded size.
%
% Lemma: one can define a quasi-ordering on trees with important nodes that 
% is wqo, and only touches dummy nodes.
%
% Conclude using Lemma 12
%

% Now: the bad part.
% Assume that there are arbitrarily long bad boughs.
% 
% Lemma 1: we can assume that the context is fixed, because 
% there are essentially finitely many contexts.
%
% Lemma 2: we can then color the graphs obtained by fixing the context in
% one color. By 2-WQO, we would obtain a pair of graphs where one embeds
% into the other, but this gives an embedding of bad boughs, which is a contradiction.
%
