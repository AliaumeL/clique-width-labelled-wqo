%!TEX root = ../cwqo.tex
\section{Bad Patterns In Tree Decompositions}
\label{sec:bad-patterns}

We will make heavy usage of the results of \cite{COLC07} on trees.

\subsection{From Interpretation to Monoids}

\AP We use unranked ordered trees $T$ with labels taken from a finite alphabet
$\Sigma$ placed on the nodes of the trees. These trees are denoted using
$\Trees{\Sigma}$. In a tree, the root is denoted by $\treeRoot$, and the set of
leaves is denoted by $\Leaves{T}$. Given two nodes $x$ and $y$ in a tree $T$,
their least common ancestor is denoted by $\lca(x,y)$. Given two nodes $x,y$
such that $y$ is an ancestor of $x$, the notation $T[x:y]$ denotes the
\emph{word} in $\Sigma^*$ obtained by reading the labels of the nodes on the
path from $x$ to $y$ in $T$. 

\begin{lemma}
    \label{interpretation-to-monoid:lem}
    Let $\Sigma$ be a finite alphabet, and $I$ be an $\MSO$ interpretation
    from $\Trees{\Sigma}$ to $\Graphs$.
    There exists a finite monoid $M$, a morphism
    $\mu \colon \Sigma \to M$,
    and a subset $P \subseteq M^3$ such that
    \begin{align*}
        &\forall T \in \Trees{\Sigma},
        \forall x,y \in \Leaves{T}, \\
        &\varphi(x,y) = \top \\
        &\iff \\
        &(\mu(T[x \colon \lca(x,y)]), 
          \mu(T[\lca(x,y) \colon \treeRoot]), 
          \mu(T[y \colon \lca(x,y)])) \in P
    \end{align*}
\end{lemma}

\begin{figure}
    \centering
    \todo[inline]{Draw the tree and how we compute the existence of an edge}
    \caption{The interpretation of a tree using a monoid and an accepting part.}
    \label{interpretation-to-monoid:fig}
\end{figure}

\AP This combinatorial description of the interpretation in terms of a monoid
allows us to introduce the notion of \emph{edge labelled tree}. We use the
notation $\Trees[M]{\Sigma}$ for the trees where the vertices are labelled by
elements of $\Sigma$ and the edges are labelled by elements of $M$. To simplify
notations, we will allow ourselves to write $\mu[T](x,y)$ for $\mu(T[x \colon
y])$ and omit $T$ when it is clear from the context.

\todo[inline]{Explain how the new tree interpretation works, using 
    the so-called \emph{composition ordering} on trees labelled by
    a finite monoid}

\begin{figure}
    \centering
    \todo[inline]{Draw the composition ordering on trees}
    \caption{The composition ordering on trees}
    \label{composition-ordering:fig}
\end{figure}

\subsection{Forward Ramseyan Splits and Bad Branches}

\todo[inline]{Recall that in the case of words, there exists a factorisation theorem
    due to Simon, explain that this will be an equivalent for trees.}

\AP
A \intro{split of height $N$} in a tree $T$ is a mapping $s$ from the nodes
of $T$ to $\set{1, \dots, N}$. Given a split, two elements $x \treelt y$ such
that $s(x) = s(y) = k$ are \intro{$k$-neighbours} if $s(z) \geq k$ for all
$z$ in the path from $x$ to $y$.
A split $s$ is \intro{forward Ramseyan} if for every $k = 1, \dots, N$ and
and every $x, y, x', y'$ in the same class of $k$-neighbourhood with $x \treelt
y$ and $x' \treelt y'$, we have
\[
    s(x,y) = s(x,y) s(x',y')
\]
where the product is the product in the monoid $M$.
So in particular, $s(x, y)$ is an idempotent, but $s(x, y)$ and $s(x', y')$ may be
different idempotents.

\todo[inline]{
    Explain that using the \emph{gap embedding relation}, we almost
    have a monotone surjective map! But not quite because of
    adjacent idempotent nodes. + drawing
}

\begin{figure}
    \centering
    \todo[inline]{A gap embedding relation between two small trees where we
    break an edge because of the separation of two idempotent nodes.}
    \caption{The gap embedding relation vs the composition ordering}
    \label{gap-embedding-composition-diff:fig}
\end{figure}

\begin{definition}
    \label{ramseyan-branch:def}
    Let $T$ be a tree and $s$ be a \kl{forward Ramseyan split} of height $N$.
    A branch $B$ has depth $k$ if its root has depth $k$ in $T$,
    and all nodes of $B$ have depth at most $k$.
    The $k$-length of a branch $B$ of depth $k$ is the number of nodes in $B$
    at depth $k$.
\end{definition}

\AP Let $B$ be a branch of depth $k$ in $T$. We define $B(T)$ the complete
subtree of $T$ rooted at the root of $B$. A branch $B$ of depth $k$, defines a
partition of the nodes in $B(T)$ as follows: for every node $x$ of $T$, we
define $B_l(x)$ be the least ancestor of $x$ that belongs to $B$ and has depth
$k$, $B(x)$ be the least ancestor of $x$ that belongs to $B$, and $B_r(x)$ be
the $k$-neighbour of $B_l(x)$ in $B$ if it exists, and the last node of $B$
otherwise. To every leaf $x$ of $B(T)$, we associate a triple $(\mu(x, B(x)),
\mu(x,B(x)), \mu(B_l(x), B(x)), \mu(B(x), B_r(x)))$.

\begin{definition}
    \label{good-branch:def}
    Let $\Sigma$ be a finite alphabet, $M$ be a finite monoid, 
    $T$ be a tree in $\Trees[M]{\Sigma}$, $s$ be a 
    \kl{forward Ramseyan split} of height $N$ in $T$,
    and $B$ be a branch of depth $k$ in $T$.
    The branch $B$ is a \intro{good branch} if, for every element $m \in M$,
    there exists a branch $H$ of depth $k$ in some other tree $T'$ in $\Trees[M]{\Sigma}$
    and a function $h \colon B(T) \to H(T')$ such that
    \begin{itemize}
        \item The subgraph generated by $m B(T)$ embeds into the subgraph generated by $m H(T')$ via $h$.
        \item The image of the root of $B$ by $h$ is the root of $H$.
        \item The image of the last element of $B$ by $h$ the last element of $H$.
        \item The map $h$ respects the $3$-types of the leaves.
    \end{itemize}
\end{definition}

\begin{lemma}
    Assume that there are no bounds on the $k$-length of bad branches, then 
    the class of images is not $(3 \times \card{M}^3)$-well-quasi-ordered.
\end{lemma}
\begin{proof}
    Let $\seqof{B_i}$ be an infinite sequence of bad branches of strictly increasing $k$-length.
    This means that for every $i$ there exists an $m_i \in M$ such that
    no branch $H$ can contain $m B_i(T)$. By extracting a subsequence, we can assume that $m$ is constant.
    Extracting more, we can assume that the $k$-length of $B_i$ is greater than three times the
    number of nodes in $B_{i-1}$.

    Look at the generated graphs, colored by the $3$-types of the leaves plus
    distinguishing colors for the root and last elements of the branches. Assume by contradiction that
    there exists an embedding from $T_i$ to $T_j$ for some $i < j$. Then,
    this is also an embedding between $B_i(T_i)$ and $B_i(T_j)$, and
    shows that $B_i(T_i)$ is a \kl{good branch} which is a contradiction.
\end{proof}

\begin{lemma}
    Let $T$ be a tree and let $B$ be a good branch. Then, one can embed     
    $B$ into three copies of itself.
\end{lemma}
\begin{proof}
    TODO. Be careful about idempotents but it should be ok.
\end{proof}

\begin{lemma}
    Assume that there is a bound $N_0$ on the length of bad branches, then
    one can embed any tree $T$ into a tree $T'$ with a ramseyan split of height $XXX$
    such that all idempotent nodes are "distance insensitive".
\end{lemma}
\begin{proof}
    This is done by grouping the $k$-neighbours of the tree $T$ into buckets of length $N_0 + 1$, and then
    splitting those buckets. 
\end{proof}

\begin{lemma}
    The gap ordering associated to the ramseyan split of height $XXX$ is a well-quasi-ordering.
\end{lemma}
\begin{proof}
    Trivial.
\end{proof}

\begin{theorem}
    Assume that there is a bound $N_0$ on the length of bad branches, then
    the class of images is wqo-well-quasi-ordered for induced substructures.
\end{theorem}
\begin{proof}
    This is a direct consequence of the fact that the idempotent nodes are distance insensitive.
    The gap embedding immediately proves that we have the same forward Ramseyan split,
    with the same "local types" for every edges.
\end{proof}


\begin{proofof}{effective-image:thm}[main]
    Test
\end{proofof}

TODOs:
\begin{itemize}
    \item Define the simon's factorisation theorem for trees.
    \item Check that we can assume that "all products are nice" and not just the ones
        appearing in the decomposition.
    \item Define "branch decompositions" and the "type of nodes" in a branch decomposition.
    \item Define a "bad branch"
    \item Prove that arbitrarily large bad branches violate WQO.
    \item Prove that "good branches" can be embedded into three copies of themselves
    \item Define the "good expansion" of a tree $T$ by expanding all good branches.
    \item Update the tree-decomposition of this good expansion to add \emph{non-sibling}
        idempotent nodes.
    \item Prove that the good expansions of a tree $T$ are well-quasi-ordered using a 
        suitable gap embedding.
    \item Conclude that the class of images is WQO.
\end{itemize}
